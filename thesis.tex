%%%%%%%%%%%%%%%%%%%%%%%%%%%%%%%%%%%%%%%%%%%%%%%%%%%%%%%%%%%%%%%%%%%%%%%%%%%%%%%%
%%%%%%%%%%%%%%%%%%   Vorlage für eine Abschlussarbeit   %%%%%%%%%%%%%%%%%%%%%%%%
%%%%%%%%%%%%%%%%%%%%%%%%%%%%%%%%%%%%%%%%%%%%%%%%%%%%%%%%%%%%%%%%%%%%%%%%%%%%%%%%

% Erstellt von Maximilian Nöthe, <maximilian.noethe@tu-dortmund.de>
% ausgelegt für lualatex und Biblatex mit biber

% Kompilieren mit
% lualatex dateiname.tex
% biber dateiname.bcf
% lualatex dateiname.tex
% lualatex dateiname.tex
% oder einfach mit:
% make

\documentclass[
  tucolor,
  BCOR=12mm,     % 12mm binding corrections, adjust to fit your binding
  DIV = 10,      % standart 10
  parskip=half,  % new paragraphs start with half line vertical space
  open=any,      % chapters start on both odd and even pages
  cleardoublepage=plain,  % no header/footer on blank pages
]{tudothesis}


% Warning, if another latex run is needed
\usepackage[aux]{rerunfilecheck}

% just list chapters and sections in the toc, not subsections or smaller
\setcounter{tocdepth}{2}

%------------------------------------------------------------------------------
%------------------------------ Sprache und Schrift: --------------------------
%------------------------------------------------------------------------------
\usepackage{fontspec}
\defaultfontfeatures{Ligatures=TeX}  % -- becomes en-dash etc.

% german language
\usepackage{polyglossia}
\setdefaultlanguage{german}

% for english abstract and english titles in the toc
\setotherlanguages{english}

% intelligent quotation marks, language and nesting sensitive
\usepackage[autostyle]{csquotes}

% microtypographical features, makes the text look nicer on the small scale
\usepackage{microtype}

%------------------------------------------------------------------------------
%------------------------ Für die Matheumgebung--------------------------------
%------------------------------------------------------------------------------

\usepackage{amsmath}
\usepackage{amssymb}
\usepackage{mathtools}
\usepackage{braket}

% Enable Unicode-Math and follow the ISO-Standards for typesetting math
\usepackage[
  math-style=ISO,
  bold-style=ISO,
  sans-style=italic,
  nabla=upright,
  partial=upright,
]{unicode-math}
\setmathfont{Latin Modern Math}

% nice, small fracs for the text with \sfrac{}{}
\usepackage{xfrac}


%------------------------------------------------------------------------------
%---------------------------- Numbers and Units -------------------------------
%------------------------------------------------------------------------------

\usepackage[
  locale=DE,
  separate-uncertainty=true,
  per-mode=symbol-or-fraction,
]{siunitx}
\sisetup{math-micro=\text{µ},text-micro=µ}

%------------------------------------------------------------------------------
%-------------------------------- tables  -------------------------------------
%------------------------------------------------------------------------------

\usepackage{booktabs}       % stellt \toprule, \midrule, \bottomrule

%------------------------------------------------------------------------------
%-------------------------------- graphics -------------------------------------
%------------------------------------------------------------------------------

\usepackage{graphicx}
\usepackage{grffile}

% allow figures to be placed in the running text by default:
\usepackage{scrhack}
\usepackage{float}
\floatplacement{figure}{htbp}
\floatplacement{table}{htbp}

% keep figures and tables in the section
\usepackage[section, below]{placeins}


%------------------------------------------------------------------------------
%---------------------- customize list environments ---------------------------
%------------------------------------------------------------------------------

\usepackage{enumitem}

%------------------------------------------------------------------------------
%------------------------------ Bibliographie ---------------------------------
%------------------------------------------------------------------------------

\usepackage[
  backend=biber,   % use modern biber backend
  autolang=hyphen, % load hyphenation rules for if language of bibentry is not
                   % german, has to be loaded with \setotherlanguages
                   % in the references.bib use langid={en} for english sources
]{biblatex}
\addbibresource{references.bib}  % die Bibliographie einbinden
\DefineBibliographyStrings{german}{andothers = {{et\,al\adddot}}}

%------------------------------------------------------------------------------
%------------------------------ Sonstiges: ------------------------------------
%------------------------------------------------------------------------------

\usepackage[pdfusetitle,unicode,linkbordercolor=tugreen]{hyperref}
\usepackage{bookmark}
\usepackage[shortcuts]{extdash}

%------------------------------------------------------------------------------
%-------------------------    Angaben zur Arbeit   ----------------------------
%------------------------------------------------------------------------------

\author{Ismael Abu-Nada}
\title{\LaTeX-Dokumentenklasse und Vorlage für Abschlussarbeiten an der TU Dortmund}
\date{2018}
\birthplace{Dortmund}
\chair{Lehrstuhl für Theoretische Physik II}
\division{Fakultät Physik}
\thesisclass{Bachelor of Science}
\submissiondate{}
\firstcorrector{Prof.~Dr.~Erstgutachter}
\secondcorrector{Prof.~Dr.~Zweitgutachter}

% tu logo on top of the titlepage
\titlehead{\includegraphics[height=1.5cm]{logos/tu-logo.pdf}}

\begin{document}
\frontmatter
%\input{content/hints.tex}
\maketitle

% Gutachterseite
\makecorrectorpage

% hier beginnt der Vorspann, nummeriert in römischen Zahlen
\thispagestyle{plain}
\iffalse
\section*{Kurzfassung}
Hier steht eine Kurzfassung der Arbeit in deutscher Sprache inklusive der Zusammenfassung der
Ergebnisse.
Zusammen mit der englischen Zusammenfassung muss sie auf diese Seite passen.

\section*{Abstract}
\begin{english}
The abstract is a short summary of the thesis in English, together with the German summary it has to fit on this page.
\end{english}
\fi

\section*{Kurzfassung}
%In deser Arbeit wird von den Wellenfunktionen des
In dieser Arbeit wird mit den Wellenfunktionen des periodisch getriebenen harmonischen Oszillators das Floquet-Theorem überprüft und für den Fall, dass ein weiterer ungetriebener Oszillator angekoppelt wird, die Wellenfunktionen hergeleitet.
Es werden diverse Erwartungswerte beider Systeme bestimmt und für die gekoppelten Oszillatoren visualisiert.
Dabei werden die Quasienergien der Floquet-Theorie verwendet, welche für eine beliebige und eine sinusförmige Treibkraft des einzelnen getriebenen Oszillators bestimmt werden.
Außerdem wird das Problem des einzelnen getriebenen Oszillators in die 2. Quantisierung übersetzt, wonach ein weiteres Mal die Erwartungswerte berechnet werden.

%\section*{Abstract}

\tableofcontents

\mainmatter
% Hier beginnt der Inhalt mit Seite 1 in arabischen Ziffern
\chapter{Einleitung}
\iffalse
Hier folgt eine kurze Einleitung in die Thematik der Bachelorarbeit.
Die Einleitung muss kurz sein, damit die vorgegebene Gesamtlänge der
Arbeit von 25 Seiten nicht überschritten wird.
Die Beschränkung der Seitenzahl sollte man ernst nehmen,
da Überschreitung zu Abzügen in der Note führen kann.
Um der Längenbeschränkung zu genügen, darf auch nicht an der Schriftgröße,
dem Zeilenabstand oder dem Satzspiegel (bedruckte Fläche der Seite) manipuliert werden.



nur wenige systeme exakt loesbar in qm harm oszi, wasserstoff. bei zeitabh hamiltonop keine stat schroed glgl  (eigenwertglg), gelingt aber bei periodischen mit floqet thoerie
\fi

\chapter{Floquet-Theorie}


%\section{Floquet-Theorie}
  Die Floquet-Theorie \cite{haengi} ist ein nützliches Werkzeug zur der Lösung von quantenmechanischen Systemen, welche durch einen zeitlich periodischen Hamilton-Operator
  \begin{equation}
    H(t) = H(t+T) \; ,
  \end{equation}
  mit der Periode $T$, beschrieben werden.

  Das Floquet-Theorem besagt, dass bei einem solchen System, die Lösungen $\Psi_n(x,t)$ der Schrödinger-Gleichung
  \begin{equation}
    \text{i}\hbar\frac{\partial}{\partial t}\Psi_n(x,t) = H(t)\Psi_n(x,t)
    \label{schroedinger}
  \end{equation}
  in Ortsdarstellung die Form
  \begin{equation}
    \Psi_n(x,t) = \text{e}^{-\frac{i}{\hbar}\epsilon_nt}\Phi_n(x,t)
    \label{floquet_theorem}
  \end{equation}
  haben.
  Hierbei sind $\Phi_n(x,t) = \Phi_n(x,t+T)$ $T$-periodische Funktionen, die sogenannten Floquet-Moden, und $\epsilon_n$ die zugehörigen reellen Quasienergien, wobei diese Bezeichnungen gewählt wurden aufgrund der Parallele zu den Bloch-Moden und Quasiimpulsen des Bloch-Theorems \cite{haengi}.
  Das Floquet-Theorem kann damit als "Bloch-Theorem in der Zeit" aufgefasst werden \cite{sherly}.

  Durch Einsetzen dieses Ansatzes für die Wellenfunktionen (\ref{floquet_theorem}) in die Schrödingergleichung (\ref{schroedinger}) erhalten wir
  \begin{equation}
    \epsilon_n \Phi_n(x,t) = \left(H(t)-\text{i}\hbar\frac{\partial}{\partial t}\right)\Phi_n(x,t) = \cal{H}(t) \Phi_n(x,t) \; .
    \label{eigenwertproblem}
  \end{equation}
  Die Lösung der Schrödinger-Gleichung konnte somit auf die Lösung eines Eigenwertproblems für den neuen Operator $\cal{H}(t)$ zurückgeführt werden \cite{sherly}.

  Die hermitischen Operatoren $H(t)$ und $\cal H(t)$ operieren auf dem Hilbertraum $\cal{L}^2 \otimes \cal{T}$.
  Dabei ist $\cal{L}^2$ der Raum der quadratintegrablen Funktionen und $\cal{T}$ der Raum der auf $[0,T]$ integrablen Funktionen, da die Operatoren $T$-periodisch sind \cite{haengi}.
  Nach dem Spektralsatz bilden die Eigenfunktionen $\Phi_n(x,t)$ von $\cal{H}(t)$ eine Orthogonalbasis von $\cal{L}^2 \otimes \cal{T}$, welche auf eine Orthonormalbasis normiert werden kann, wodurch wir das Skalarprodukt definieren k"onnen als:
  %Es ergibt sich
  \begin{align}
    \begin{split}
    \braket{\braket{\Phi_n(x,t)|\Phi_m(x,t)}} &= \frac{1}{T} \int_0^T \braket{\Phi_n(x,t)|\Phi_n(x,t)} \text{d}t \\
    &= \frac{1}{T} \int_0^T \int_{-\infty}^{\infty} \Phi_n^*(x,t)\Phi_m(x,t) \: \text{d}x\text{d}t = \delta_{n,m} \; .
    \label{skalarprodukt_einzelner}
    \end{split}
  \end{align}


%war vorher subsection
 \section{\texorpdfstring{Zeitlich gemittelter Erwartungswert der Energie $\bar{H}_n$}{Zeitlich gemittelter Erwartungswert der Energie bar{H}_n}}

    Da $H(t)$ nicht zeitlich konstant ist, sind auch dessen Erwartungswerte, die Energien des Systems, zeitabhängig.
    Ein direkter Vorteil der Floquet-Theorie liegt darin, dass sich die durchschnittliche Energie
    \begin{equation}
      \bar{H}_n  = \braket{\braket{\Psi_n(x,t)|H(t)|\Psi_m(x,t)}}
      %\label{mittleres_H}
    \end{equation}
    des $n$-ten Zustandes $\Psi_n(x,t)$ leicht über die Quasienergien $\epsilon_n$ berechnen lässt, ohne explizit Integrale zu lösen.

    Dazu ersetzen wir $H(t)$ mit Hilfe von (\ref{eigenwertproblem}).
    Außerdem unterscheiden sich die Floquet-Moden $\Phi_n(x,t)$ und die Wellenfunktionen $\Psi_n(x,t)$ nur durch eine komplexe Phase, daher sind deren Skalarprodukte identisch.
    Weiterhin benutzen wir (\ref{eigenwertproblem}), dass die Floquet-Moden Eigenfunktionen von $\cal{H}(t)$ sind:
    \begin{align}
      \begin{split}
      \bar{H}_n  &= \braket{\braket{\Phi_n(x,t)|\cal{H}(t)+\text{i}\hbar\frac{\partial}{\partial t}|\Phi_n(x,t)}} \\
      &=\epsilon_n \braket{\braket{\Phi_n(x,t)|\Phi_n(x,t)}} + \braket{\braket{\Phi_n(x,t)|\text{i}\hbar\frac{\partial}{\partial t}|\Phi_n(x,t)}} \\
      &= \epsilon_n + \braket{\braket{\Phi_n(x,t)|\text{i}\hbar\frac{\partial}{\partial t}|\Phi_n(x,t)}} \; .
    \end{split}
    \end{align}
    Nähere Betrachtung zeigt, dass
    \begin{equation}
      \text{i}\hbar\frac{\partial}{\partial t} = -\omega \frac{\partial \cal{H}(t)}{\partial \omega}
    \end{equation}
    gilt \cite{haengi}.
    Da für $\epsilon_n$ und $\cal H(t)$ mit (\ref{eigenwertproblem}) eine Eigenwertgleichung vorliegt, kann das Hellman-Feynman-Theorem angewendet werden \cite{hellmann}.
    Dieses gibt eine Verbindung zwischen den Ableitungen der Eigenwerte und der Ableitung des Hamilton-Operators an.
    In unserem Fall erhalten wir dadurch
    \begin{equation}
      \frac{\partial \epsilon_n}{\partial \omega} = \braket{\braket{\Phi_n(x,t)|\frac{\partial \cal H(t)}{\partial \omega}|\Phi_n(x,t)}} \; .
    \end{equation}
    Damit folgt \cite{haengi}:
    \begin{equation}
      \bar{H}_n = \epsilon_n - \omega\frac{\partial \epsilon_n}{\partial \omega} \; .
      \label{mittleres_H}
    \end{equation}

    %\newpage






























  



\chapter{Berechnung von Erwartungswerten}
\label{3}
  In diesem Abschnitt werden die zeitabhängigen Erwartungswerte des periodisch getriebenen Oszillators für den Impuls, den Ort und die Energie berechnet, so wie das zeitliche Mittel des Erwartungswertes der Energie.
  Dabei werden die Erwartungswerte für alle Zustände $\Psi_n(x,t)$ auf die bekannten Erwartungswerte des Standard-Oszillators zurückzuführt, womit wir eine Art "2. Quantisierung für die Erwartungswerte" haben.

  Die Treibkraft $S(t)=S(t+T)$ ist nicht weiter festgelegt und geht stets über die klassische Lösung $\zeta(t)$ (\ref{dgl_zeta}) in die Erwartungswerte ein.

  Um die Erwartungswerte des getriebenen Oszillators zu erhalten, werden wir dessen Berechnnung auf die Berechnung der bekannten Erwartungswerte des ungetriebenen Oszillators zurückführen.
  Dafür nutzen wir aus, dass die Wellenfunktionen und die Floquet-Moden nach (\ref{gesamtlsg_einzelner}) und (\ref{floquet_moden_einzelner}) einem um $\zeta(t)$ verschobenen ungetriebnen Oszillator mit einer zusätzlichen komplexen Phase entsprechen.
  Da eine komplexe Phase beim Bilden des Betrages wegfällt, gilt
  \begin{equation}
    \Psi_n^*(x,t)\Psi_n(x,t) = |\Psi_n(x,t)|^2 = |\Phi_n(x,t)|^2 = |\Psi_{n,\text{ung}}(x-\zeta(t),t)|^2 = |\Psi_{n,\text{ung}}(y,t)|^2 \; ,
    \label{betrag_einzelner}
  \end{equation}
  wobei $\Psi_{n,\text{ung}}(y,t)$ die Zustände des ungetriebenen Oszillators bezeichnen.

  Bei dem Integral von $-\infty$ bis $\infty$ ist eine konstante Verschiebung der Variable irrelevant.
  Außerdem sind die $\Psi_{n,\text{ung}}(y,t)$ auf dem Raum normiert, daher sind auch die $\Psi_n(x,t)$ auf $x \in (-\infty,\infty)$ normiert, wie bereits am Ende von Kapitel \ref{lsg_einzelner} erwähnt.

  Eine Visualisierung von Erwartungswerten folgt in Kapitel (\ref{erwartungswerte_gekoppelt}), weil es sich bei den Erwartungswerten der zwei gekoppelten Oszillatoren genau um die Summe aus den Erwartungswerten von zwei gesonderten Oszillatoren handelt, wie sich dort zeigen wird.

  \section{Zeitabhängige Erwartungswerte des Ortes}
  %INTERPRETATION ; SCHWINGT KLASSICH USW
    Zuerst betrachten wir den Erwartungswert $\braket{x}_n$ des Ortsoperators $x$ für den $n$-ten Zustand $\Psi_n(x,t)$ des getriebenen Oszillators, welcher per Definition gegeben ist durch
    \begin{equation}
      \braket{x}_n = \int_{-\infty}^{\infty} \Psi_n^*(x,t)x\Psi_n(x,t) \: \text dx
      = \int_{-\infty}^{\infty} |\Psi_n(x,t)|^2 x \: \text dx \; .
    \end{equation}
    Mit der Substitution $y=x-\zeta(t)$ und unter Verwendung von (\ref{betrag_einzelner}) und der Normierung der ungetriebenen Oszillator-Funktionen gelangen wir zu
    \begin{align}
      \begin{split}
        \braket{x}_n &= \int_{-\infty}^{\infty} |\Psi_{n,\text{ung}}(x-\zeta(t),t)|^2 x \: \text dx
        = \int_{-\infty}^{\infty} |\Psi_{n,\text{ung}}(y,t)|^2 (y+\zeta(t)) \: \text dy \\
        &= \braket{y}_{n,\text{ung}} + \zeta(t) = \zeta(t) \; .
        \label{erwartungswert_x_einzelner}
      \end{split}
    \end{align}
    Da die Erwartungswerte $\braket{y}_{n,\text{ung}}$ für den Ort des ungetriebenen Oszillators verschwinden, ist der Erwartungswert jedes Zuständes des getriebenen Oszillators genau $\zeta(t)$.

    Das bedeutet, der Erwartungswert $\braket{x}_n$ für den Ort eines beliebigen Zustandes des quantenmechanischen getriebenen Oszillators entspricht genau der Lösung $\zeta(t)$ der Bewegungsgleichung des klassischen getriebenen Oszillators.
    Der Erwartungswert hat deshalb sehr große Amplituden, wenn die Treibfrequenz $\omega$ von $S(t)$ in der Nähe der Eigenfrequenz des Systems $\omega_0$ liegt (\ref{zeta_bel_kraft}), was der klassischen Anschauung entspricht, weil wir ein ungedämpftes System betrachten.
    Der Erwartungswert des ungetriebenen Oszillators, welcher im Ursprung liegt, ist somit zeitabhängig periodisch um $\zeta(t)=\zeta(t+T)$ verschoben, mit $T=2\pi/\omega_0$.
    Dies entspricht dem Ergebnis, wie es auch bei einer konstanten Verschiebung der Ortsvariable/-operator $x$ erhalten wird, z.B. beim Oszillator im zusätzlichen konstanten elektrischen Feld, nur das die Verschiebung hier die zeitabhängige klassische Lösung ist.

    Da $\braket{x}_n=\zeta(t)$ die klassische Bewegungsgleichung
    \begin{equation}
      m\ddot{\braket{x}}_n + m\omega_0^2\braket{x}_n - S(t) = 0
    \end{equation}
    erfüllt, ist ergänzend das Ehrenfest-Theorem bestätigt.

    Für den Erwartungswert von $x^2$ ergibt sich komplett analog
    \begin{align}
      \braket{x^2}_n &= \int_{-\infty}^{\infty} |\Psi_n(x,t)|^2 x^2 \: \text dx
      = \int_{-\infty}^{\infty} |\Psi_{n,\text{ung}}(y,t)|^2 (y+\zeta(t))^2 \: \text dy \notag \\
      &= \braket{y^2}_{n,\text{ung}} + 2\braket{y}_{n,\text{ung}} + \zeta^2(t)
      = s^2(2n+1) + \zeta^2(t) \; ,
    \end{align}
    zusammen mit der charakteristischen Länge des ungetriebenen Oszillators
    \begin{equation}
      s = \sqrt{\frac{\hbar}{2m\omega_0}} \; .
      \label{charak_laenge}
    \end{equation}
    Der Erwartungswert ist nun um $\zeta^2(t)$ verschoben.

    Insgesamt können alle Erwartungswerte $\braket{x^m}_n$ auf diese Art mit den bekannten ungetriebenen Erwartungswerten berechnet werden.
    Ab $\braket{x^3}_n$ kommt es wegen $(y+\zeta(t))^3$ aber im Allgemeinen nicht nur zu Verschiebungen bzw. zusätzlichen $\zeta(t)$-Termen, mit demselben Exponenten wie die ungetriebenen Erwartungswerte, weshalb wir den Ewartungswert $\braket{x^m}_n$ nicht einfacher schreiben können als den linearen Erwartungswert des Klammerausdruckes mit dem binomischen Lehrsatz zu vereinfachen:
    \begin{equation}
      \braket{x^m}_n = \braket{(y+\zeta(t))^m}_{n,\text{ung}} = \sum_{j=0}^m \begin{pmatrix} m \\ j \\ \end{pmatrix} \braket{y^{m-j}}_{n,\text{ung}}\zeta^j(t) \; .
    \end{equation}
    In obiger Formel ist
    \begin{equation}
      \begin{pmatrix} m \\ j \\ \end{pmatrix} = \frac{m!}{j!(m-j)!}
    \end{equation}
    der Binomial-Koeffizient.
    Hiernach setzt sich der Erwartungswert $\braket{x^m}_n$ aus allen $\zeta(t)...\zeta^m(t)$ zusammen, die mit dem Binomial-Koeffizienten und der jeweiligen Potenz von $\braket{y}_{n,\text{ung}}$ gewichtet werden.


    %\newpage





  \section{Zeitabhängige Erwartungswerte des Impulses}
    Um den Erwartungswert $\braket{p}_n$ des Impulsoperators für den Zustand $\Psi_n(x,t)$ des getriebenen Oszillators zu ermitteln, nutzen wir gleichermaßen die Beziehung (\ref{betrag_einzelner}) und unterdessen ein geschicktes Anwenden der Produktregel, um das Problem erneut auf den ungetriebenen Oszilaltor zu beschränken.

    Wieder wählen wir den Variablenwechsel zur neuen Variable $y=x-\zeta(t)$, weil der Impulsoperator $p=p_x$ in der neuen Koordinate identisch ist, wie schon am Anfang von Kapitel \ref{lsg_einzelner} angedeutet, was sich leicht mit der Kettenregel zeigen lässt:
    \begin{equation}
      -\text i\hbar \frac{\partial}{\partial x} = -\text i\hbar \frac{\partial}{\partial y}\frac{\partial y}{\partial x} = -\text i\hbar \frac{\partial}{\partial y} \cdot 1 \; , \quad \text{d.h.} \quad p_x = p_y \; .
    \end{equation}
    Weiterhin können wir den $y$-unabhängigen Teil der komplexen Phasen in den Wellenfunktionen (\ref{gesamtlsg_einzelner}) sofort am Operator vorbeiziehen und zu 1 zusmmenfassen, daher schreiben wir $\braket{p}_n$ als
    \begin{align}
      \braket{p}_n &= \int_{-\infty}^{\infty} \Psi^*_n(x,t) p \Psi_n(x,t) \; \text dx
      = \int_{-\infty}^{\infty} \Psi^*_n(y+\zeta(t),t) p_y \Psi_n(y+\zeta(t),t) \: \text dy \notag\\
      &= \int_{-\infty}^{\infty} N_nO_n(y)\text e^{\frac{-m\omega_0}{2\hbar}y^2}\text e^{\frac{-\text i}{\hbar}m\dot{\zeta}(t)y} p_y N_nO_n(y)\text e^{\frac{-m\omega_0}{2\hbar}y^2}\text e^{\frac{+\text i}{\hbar}m\dot{\zeta}(t)y} \: \text dy \; .
      \label{spaste}
    \end{align}
    Mit der Produktregel spalten wir die Anwendung des Impulsoperators auf den rechten Teil auf in die Anwendung auf die ungetriebene Oszillatorfunktion (in der neuen Koordinate $y$) und die Anwendung auf die verbleibende komplexe Phase:
    \begin{align}
      &p_y N_nO_n(y)\text e^{\frac{-m\omega_0}{2\hbar}y^2}\text e^{\frac{+\text i}{\hbar}m\dot{\zeta}(t)y} = \notag\\
      &\left(p_y N_nO_n(y)\text e^{\frac{-m\omega_0}{2\hbar}y^2}\right) \text e^{\frac{+\text i}{\hbar}m\dot{\zeta}(t)y}
      + N_nO_n(y)\text e^{\frac{-m\omega_0}{2\hbar}y^2} \left(p_y \text e^{\frac{+\text i}{\hbar}m\dot{\zeta}(t)y}\right) \notag\\
      &= \text e^{\frac{+\text i}{\hbar}m\dot{\zeta}(t)} \left[p_yN_nO_n(y)\text e^{\frac{-m\omega_0}{2\hbar}y^2} + N_nO_n(y)\text e^{\frac{m\omega_0}{2\hbar}y^2}m\dot{\zeta}(t) \right] \; .
      \label{spaste2}
    \end{align}
    Hiernach haben wir effektiv erreicht, dass wir den Impulsoperator an der $y$-abhängigen Phase vorbei ziehen konnten.
    Damit ist es möglich die verbliebenen Phasen in (\ref{spaste}) ebenso zu 1 zu vereinfachen.
    Es verbleiben noch im ersten Teil der eckigen Klammer die Anwendung des Impulsoperators auf die Wellenfunktionen des ungetriebenen Oszillators und im zweiten Teil die Multiplikation dieser mit einem ortsunabhängigen Term.
    Zusammen mit dem übrigen linken Teil aus (\ref{spaste2}), welcher nach dem Wegfallen der komplexen Phase jetzt auch der ungetriebenen Wellenfunktion entspricht, ergeben sich dadurch die Erwartungswerte für $p_y$ und den Term $m\dot{\zeta}(t)$, welcher zeitabhängig ist.
    %Zusammen mit dem, nun identischen, übrigen komplex konjugiertem Teil aus (\ref{spaste2}) verbleiben nur noch mit im ersten Teil der der eckigen Klammer die Anwendung des Impulsoperators auf die Wellenfunktion des ungetriebenen Oszillators bzw

    Schließlich können wir den Erwartungswert für den Impuls in Abhängigkeit der bekannten Erwartungswerte des Standard-Oszillators hinschreiben:
    \begin{align}
      \braket{p}_n = \braket{p_y}_{n,\text{ung}} + m\dot{\zeta}(t) = m\dot{\zeta}(t)
    \end{align}

    Wieder erfüllt der Erwartungswert die Bewegungsgleichung des klassischen Oszillators:
    \begin{equation}
      \ddot{\braket{p}}_n + \omega_0^2\braket{p}_n - \dot{S}(t) = 0 \; .
    \end{equation}

    Wie an (\ref{spaste2}) zu sehen, ist es darüber hinaus einfach den Erwartungswert $\braket{p^m}_n$ für einen beliebigen Exponenten $m \in \mathbb{N}$ zu evaluieren.
    Wir können auf dem gleichen Weg die Produktregel anwenden, und erhalten wieder den ungetrieben Erwartungswert $\braket{p^m_y}_n$ aber diesmal zusätzlich die $m$-fache Anwendung des Operators auf die komplexe Phase, was wir sofort verallgemeinert aufschreiben können, da sich die komplexe Phase beim Ableiten nur reproduziert, weil immer der ortsunabhängige Term $m\dot{\zeta(t)}$ übrig bleibt.
    Es folgt:
    \begin{equation}
      \braket{p^m}_n = \braket{p^m_y}_{n,\text{ung}} + (m\dot{\zeta}(t))^m \; .
    \end{equation}
    Speziell für $m=2$ gilt:
    \begin{equation}
      \braket{p^2}_n = \braket{p^2_y}_{n,\text{ung}} + (m\dot{\zeta}(t))^2 = m^2\omega^2_0 s^2(2n+1) + m^2\dot{\zeta}^2(t) \; .
    \end{equation}


  \section{Heisenberg-Unschärferelation}
    Mit den zuvor errechneten Erwartungswerten $\braket{x}_n,\braket{x^2}_n,\braket{p}_n,\braket{p^2}_n$ wird hier die Unschärfe des periodisch getriebenen Oszillators aufgestellt.
    Diese ist
    \begin{align}
      \Delta x_n\Delta p_n &= \sqrt{\braket{x^2}_n-\braket{x}^2_n}\sqrt{\braket{p^2}_n-\braket{p}^2_n} \notag\\
      &= \sqrt{s^2(2n+1)+\zeta^2(t)-\zeta^2(t)}\sqrt{m^2\omega^2_0s^2(2n+1)+m^2\zeta^2(t)-m^2\zeta^2(t)} \notag\\
      &= \frac{\hbar}{2}(2n+1) \; .
    \end{align}
    Die Unschärfe ist also gleich der des ungetriebenen Oszillators, welche zeitunabhängig ist und für den Grundzustand $n=0$ minimal ist.

    Dieses Ergebnis ist sinvoll in Anbetracht der Tatsache, dass die (zeitabhängige) Verschiebung der Wellenfunktionen im Raum $\zeta(t)$ (\ref{gesamtlsg_einzelner}) die Unschärfe nicht ändern sollte.
    Die Wellenfunktionen haben allerdings noch den weiteren Unterschied der orts- und zeitabhängigen komplexen Phase, verglichen mit den Wellenfunktionen des Standard-Oszillators.
    Diese spielt zwar keine  Rolle bei den Erwartungswerten für den Ort, bei den Impulserwartungswerten aber schon.
    Dennoch war zu erwarten, dass sich auch beim Impuls eine Standardabweichung $\Delta p_n$ wie beim ungetriebenen Oszillator ergibt, so wie bei $\Delta x_n$, da $\braket{p}_n$ im Wesentlichen die zeitliche Ableitung von $\braket{x}_n$ ist.
    Demnach folgte die gleiche Unschärfe.

  %  MEHR SCHREIEN ????????

\iffalse
  \subsection{Erwartungswerte der Energie}
    Hier werden wir den normalen zeitabhängigen Erwartungswert der Energie $\braket{H(t)}_n$ aufstellen.
    Weiterhin wird der zeitlich, über eine Periode $T$, gemittelten Erwartungswert $\bar H_n$ mittels Formel (\ref{mittleres_H}) für eine beliebige Treibkraft bestimmt, indem wir die Quasienergien $\epsilon_n$ benutzen, welche wir in Kapitel \ref{espsilon_bel_kraft} für eine allgemeine periodische Treibkraft $S(t)$, in Abhängigkeit deren  Fourier-Koeffizienten, bestimmt haben.
\fi

  \section{Zeitabhängiger Erwartungswert der Energie}
    Mit denselben Erwartungswerten für den Ort und den Impuls, wie sie bei der Heisenber-Unschärferelation von Nöten sind, wird im Folgenden der normale Erwartungswert der Energie bzw. des Hamilton-Operators $\braket{H(t)}_n$ aufgestellt, es muss nur die Linearität des Erwartungswertes verwendet werden:
    \begin{align}
      \braket{H(t)}_n &= \braket{\frac{p^2}{2m} + \frac{1}{2}m\omega_0^2x^2 - S(t)x}_n
      =\frac{1}{2m}\braket{p^2}_n + \frac{1}{2}m\omega_0^2\braket{x^2}_n - S(t)\braket{x}_n \notag\\
      &= \frac{1}{2m}\braket{p^2_y}_{n,\text{ung}} + \frac{1}{2}m\omega_0^2\braket{y}_{n,\text{ung}} + \frac{1}{2}m\dot{\zeta}^2(t) + \frac{1}{2}m\omega_0^2\zeta^2(t) - S(t)\zeta(t) \notag\\
      &= \hbar\omega_0\left(n+\frac{1}{2}\right) - L(\dot{\zeta},\zeta,t) + m\dot{\zeta}^2(t) = E_n - L(\dot{\zeta},\zeta,t) + m\dot{\zeta}^2(t) \; .
      %hier en in extra zeile mit satz definieren evt
    \end{align}
    Wir finden wieder den Erwartungswert des ungetriebenen Oszillators, die $E_n$, mit einer zeitabhängigen Verschiebung, in welcher die Lagrange-Funktion des klassischen Problems (\ref{lagrange_zeta}) mit dem Restterm identifiziert werden kann.
    %Die Amplitude des Erwartungswertes kann wie erwartet sher groß werden, wenn die Treibfrequenz $\omega$ von $S(t)$ nah an der Eigenfrequenz des Systems $\omega_0$ liegt,
  %  WELCHE PERIODE, SCHWANKT

  \section{Durchschnittlicher Erwartungswert der Energie}
    Hier wird der zeitlich, über eine Periode $T$, gemittelte Erwartungswert $\bar H_n$ mittels Formel (\ref{mittleres_H}) für eine beliebige Treibkraft bestimmt, indem wir die Quasienergien $\epsilon_n$ benutzen, welche wir in Kapitel \ref{epsilon_bel_kraft} für eine allgemeine periodische Treibkraft $S(t)$, in Abhängigkeit deren Fourier-Koeffizienten, bestimmt haben.
    %Um den zeitlich gemittelten Erwartungswert der Energie $\bar H_n$ für eine beliebige periodisch Treibende Kraft $S(t)=S(T)$ zu berechnen,

    Unter der Annahme, dass die Summe in $\epsilon_n$ (\ref{epsilon_bel_kraft}) konvergiert, weil es sich im Wesentlichen um die Summe über $1/j^2$ handelt, leiten wir nach $\omega$ ab, indem wir die Ableitung und Summe vertauschen, dann haben wir:
    \begin{align}
      \bar H_n = \epsilon_n - \omega\frac{\partial}{\partial \omega}\epsilon_n
      = E_n - \sum_{j=-\infty}^{\infty} \left[ \frac{c_jc_{-j}}{2m(\omega_0^2-j^2\omega^2)}\left( 1-\frac{2j^2\omega^2}{(\omega_0^2-j^2\omega^2)}\right) \right] \; .
    \end{align}

    Für den expliziten Fall der Sinus-Treibkraft $S(t)=A\sin(\omega t)$ folgt:
    \begin{equation}
      \bar H_n = E_n - \frac{A}{4m(\omega_0^2-\omega^2)}\left(1-\frac{2\omega^2}{(\omega_0^2-\omega^2)}\right) \; .
    \end{equation}
    Für $\omega_0^2>\omega^2$ erhalten wir in diesem Beispiel demnach einen Anstieg des Energiemittelwertes gegenüber den Quasienergien, andernfalls eine Absenkung.



















%\section{\texorpdfstring{Zeitlich gemittelter Erwartungswert der Energie $\bar{H}_n$}{Zeitlich gemittelter Erwartungswert der Energie bar{H}_n}}

\chapter{Zwei gekoppelte getriebene harmonische Oszillatoren in der Quantenmechanik}


In diesem Teil der Arbeit wird mit Hilfe der aus (\ref{lsg_einzelner}) bekannten Lösung des einzelnen getrieben Oszillators, die Wellenfunktionen für ein System hergeleitet, dass aus zwei gekoppelten Oszillatoren $x_1$ und $x_2$ der gleichen Masse $m$ besteht, von denen einer mit der periodischen Kraft $S(t) = S(t+T)$ angetrieben wird.
Die Potentialkonstanten $k$ der beiden Oszillatoren sind ebenfalls identisch, die Kopplungskonstante $\kappa$ zwischen den Oszillatoren ist allerdings anders.
Der Hamilton-Operator dieses Systems kann direkt aus der klassichen Mechanik übernommen werden:
\begin{equation}
  H(t) = H(t+T) = \frac{p_1^2}{2m} + \frac{p_2^2}{2m} + \frac 1 2 kx_1^2 + \frac 1 2 kx_2^2 + \frac 1 2 \kappa(x_2-x_1)^2 - S(t)x_1 \; .
  \label{H_gekoppelt}
\end{equation}
Es werden auch die Erwartungswerte für  den Ort $\braket{x_{1,2}}_{n,l}$ und den Impuls $\braket{p_{1,2}}_{n,l}$, genauso wie der Erwartungswert der Energie $\braket{H}_{n,l}$ und dessen zeitliches Mittel $\bar{H}_{n,l}$ berechnet, indem auf die bekannten Erwartungswerte des einzelnen getriebenen Oszillators zurückgeführt wird.
Die Erwartungswerte werden auch graphisch dargestellt.

Zur Lösung des Systems wird eine unitäre Koordinatentransformation eingeführt, welche den Hamilton-Operator $H(x_1,x_2,p_1,p_2,t)$ zu zwei in den neuen Koordinaten unabhängigen Hamilton-Operatoren $H_+(x_+,p_+,t)$ und $H_-(x_-,p_-,t)$ mit effektiven Potentialkonstanten $k_+,k_-$ entkoppelt, welche je einen einzelnen (getriebenen) Oszillator beschreiben.
Dann ergeben sich die Wellenfunktionen leicht aus denen des einzelnen Oszillators.


%war vorher subsection
\section{Die Schrödinger-Gleichung mit unabhängigen Hamilton-Operatoren}
  Liegt ein Hamilton-Operator der Form
  \begin{equation}
    H = \sum_i H_i(x_i,p_i,t) \;,\; H_i : \cal H_i \rightarrow H_i
    \label{unabh_H}
  \end{equation}
  vor, führt der Ansatz
  \begin{equation}
    \Psi = \prod_i \Psi(x_i,p_i,t) \; , \; \Psi_i \in \cal H_i
    \label{lsg_unabh_H}
  \end{equation}
  auf unabhängige Schrödinger-Gleichungen für die einzelnen Wellenfunktionen \\ $\Psi_i(x_i,p_i,t)$, sodass \cite{online quelle}
  \begin{equation}
    \text i \hbar \frac{\partial}{\partial t}\Psi_i = \delta_{i,j}  H_j \Psi_i
    \label{schroedglg_unabh_H}
  \end{equation}
  erfüllt ist.
  Um dies schnell zu zeigen, schauen wir uns den Fall von zwei unabhängigen Operatoren an:
  \begin{align}
    \begin{split}
      \text i \hbar \frac{\partial}{\partial t} \Psi = H\Psi \iff \Psi_2 H_1 \Psi_1 + \Psi_1 H_2 \Psi_2 = \text i \hbar \left(\frac{\partial}{\partial t} \Psi_1\Psi_2 + \Psi_1\frac{\partial}{\partial t} \Psi_2 \right) \; .
    \end{split}
  \end{align}
  Da die Operatoren nur auf Funktionen wirken, die auf dem selben Raum definiert sind, kann man sie an der jeweils anderen Funktion vorbei ziehen.
  Wenn wir weiterhin durch unseren Ansatz $\Psi_1\Psi_2$ teilen, wird die Gleichung zu:
  \begin{equation}
    \frac{1}{\Psi_1}H_1\Psi_1 + \frac{1}{\Psi_2}H_2\Psi_2 = \text i \hbar \left(\frac{\frac{\partial}{\partial t} \Psi_1}{\Psi_1} + \frac{\frac{\partial}{\partial t} \Psi_2}{\Psi_2} \right) \; .
  \end{equation}
  Weil die linke und rechte Seite für alle unabhängigen $x_1,p_1,x_2,p_2$ gleich sein müssen, folgen die einzelnen Schrödinger-Gleichungen für die Wellenfunktionen $\Psi_1$ und $\Psi_2$:
  \begin{equation}
    \frac{1}{\Psi_1}H_1\Psi_1 = \frac{\frac{\partial}{\partial t} \Psi_1}{\Psi_1} \; , \; \frac{1}{\Psi_2}H_2\Psi_2 = \frac{\frac{\partial}{\partial t} \Psi_2}{\Psi_2} \; .
  \end{equation}
  Wie an diesem Beispiel leicht zu sehen, gilt Gleiches auch für beliebig viele unabhängige Hamilton-Operatoren, womit (\ref{schroedglg_unabh_H}) erfüllt ist und wir wissen, dass die Lösung $\Psi$ die Form (\ref{lsg_unabh_H}) hat.
  %nur EINE moegliche lsg hat die form psi, wenn wir einschraenken, dass psi so aussehen soll, wer weis ob es noch andere lsgen nicht dieser form gibt.



\section{Unitäre Variablenformation und allgemeine Lösung der Schrödinger-Gleichung}
  Um für unseren Hamilton-Operator $H(x_1,x_2,p_1,p_2,t)$ (\ref{H_gekoppelt}) eine entkoppelte Form $H(x_+,x_-,p_+,p_-,t)=H_+(x_+,p_+,t)+H_-(x_-,p_-,t)$ nach (\ref{unabh_H}) in den neuen Koordinaten/Variablen $x_+,x_-,p_+,p_-$ zu erhalten, wählen wir die unitären Koordinationtransformationen \cite{arxiv}
  \begin{equation}
    x_+ = \frac{1}{\sqrt{2}}(x_2+x_1) \;,\; x_-=\frac{1}{\sqrt{2}}(x_2-x_1) \;,
  \end{equation}
  welche aus den Normalmoden des klassischen Problems folgen, worauf in Kapitel () genauer eingegangen wird.
  Durch einfaches Umstellen folgen $x_1$ und $x_2$ in Abhängigkeit der neuen Koordinaten $x_+$ und $x_-$:
  \begin{equation}
    x_1=\frac{1}{\sqrt{2}}(x_+-x_-) \;,\; x_2=\frac{1}{\sqrt{2}}(x_++x_-) \; .
  \end{equation}
  Indem wir $x_1$ und $x_2$ so im Hamilton-Operator (\ref{H_gekoppelt}) ersetzen, formen wir den ortsabhängigen Teil um zu
  \begin{align}
    &\frac{1}{2}kx_1^2+\frac{1}{2}kx_2^2+\frac{1}{2}\kappa(x_2-x_1)^2-S(t)x_1= \notag\\
    &\frac{1}{2}kx_+^2+\frac{1}{2}(k+2\kappa)x_-^2-S(t)\frac{1}{\sqrt{2}}(x_+-x_-) \; .
  \end{align}
  Es tauchen nun keine Kopplungsterme $x_+x_-$ mehr auf, wie es in den alten Koordinaten der Fall war.

  Jetzt betrachten wir die neuen Impulse und überprüfen, dass es durch den Variablenwechsel nicht zu neuen Kopplungstermen in den $p_+$,$p_-$ kommt.
  Mit der Kettenregel folgt für die Ableitungen der Impulsoperatoren
  \begin{align}
    \frac{\partial}{\partial x_{\pm}} = \frac{\partial}{\partial x_1}\frac{\partial x_1}{\partial x_{\pm}} + \frac{\partial}{\partial x_2}\frac{\partial x_2}{\partial x_{\pm}}
    =\frac{\partial}{\partial x_1}\left(\pm\frac{1}{\sqrt{2}}\right)
    + \frac{\partial}{\partial x_2}\frac{1}{\sqrt{2}}
    = \frac{1}{\sqrt{2}}\left(\frac{\partial}{\partial x_2}\pm\frac{\partial}{\partial  x_1}\right) \;,
  \end{align}
  weshalb für die Impulsoperatoren identisch zu den Ortsoperatoren gilt
  \begin{equation}
    p_+ = \frac{1}{\sqrt{2}}(p_2+p_1) \;,\; p_-=\frac{1}{\sqrt{2}}(p_2-p_1) \; .
  \end{equation}
  Wir setzen erneut in unseren Hamilton-Operator (\ref{H_gekoppelt}) ein und erhalten für den impulsabhängigen Teil
  \begin{equation}
    \frac{p_1^2}{2m} + \frac{p_2^2}{2m} = \frac{p_+^2}{2m} + \frac{p_-^2}{2m}
  \end{equation}

  Wie erwartet bleibt die Summe der quadrierten Impulsoperatoren unverändert.

  Der gesamte Hamilton-Operator der zwei gekoppelten getrieben Oszillatoren ist in den neuen Variablen folglich
  \begin{align}
    H(t) &= H(t+T) = \frac{p_1^2}{2m} + \frac{p_2^2}{2m} + \frac 1 2 kx_1^2 + \frac 1 2 kx_2^2 + \frac 1 2 \kappa(x_2-x_1)^2 - S(t)x_1 \notag\\
    &= \frac{p_+^2}{2m}+\frac{1}{2}kx_+^2-\frac{1}{\sqrt{2}}S(t)x_+ \quad + \quad
    \frac{p_-^2}{2m}+\frac{1}{2}(k+2\kappa)x_-^2+\frac{1}{\sqrt{2}}S(t)x_- \notag\\
    &= \frac{p_+^2}{2m}+\frac{1}{2}k_+x_+^2-S_+(t)x_+ \quad + \quad
    \frac{p_-^2}{2m}+\frac{1}{2}k_-x_-^2-S_-(t)x_- \notag\\
    &= H_+(x_+,p_+,t) + H_-(x_-,p_-,t) \; .
  \end{align}
  Der Hamilton-Operator in den neuen Variablen $x_{\pm},p_{\pm}$ beschreibt demnach ein System aus zwei unabhängigen getriebenen harmonischen Oszillatoren mit neuen Potentialkonstanten $k_1=k$ und $k_2=k+2\kappa$ bzw. neuen Eigenfrequenzen
  \begin{equation}
    w_+=\sqrt{\frac{k}{m}} \quad\text{und}\quad \omega_-=\sqrt{\frac{k+2\kappa}{m}} \; .
  \end{equation}
  Die beiden Oszillatoren werden, wegen dem verschiedenen Vorzeichen von $S_+(t)=S_+(t+T)$ und $S_-(t)=S_-(t+T)$, periodisch aber phasenversetzt um $\pi$ mit der ursprünglichen Treibkraft $S(t)$ angetrieben, wobei diese mit dem Faktor $1/\sqrt{2}$ skaliert wird.

  Betrachten wir den komplizierter geglaubten Fall, dass beide Oszillatoren $x_1$ und $x_2$ des Systems in den alten Koordinaten getrieben sind, und zwar genau wie durch die neuen Koordinaten $x_+$ und $x_-$ vorgegeben, das heißt
  \begin{align}
      &H(t) = \frac{p_1^2}{2m} + \frac{p_2^2}{2m} + \frac 1 2 kx_1^2 + \frac 1 2 kx_2^2 + \frac 1 2 \kappa(x_2-x_1)^2 - S(t)(x_2+x_1) \notag\\
      \text{oder} \; &H(t) = \frac{p_1^2}{2m} + \frac{p_2^2}{2m} + \frac 1 2 kx_1^2 + \frac 1 2 kx_2^2 + \frac 1 2 \kappa(x_2-x_1)^2 - S(t)(x_2-x_1) \;,
  \end{align}
  liegt in den neuen Variablen ein vereinfachtes System vor, bei dem nur ein Oszillator $x_+$ oder $x_-$ getrieben ist.
  Es muss also in den klassischen Normalmoden getrieben werden, damit nach dem Variablenwechsel ein möglichst einfaches System mit nur noch einem getriebenen Oszillator vorliegt.

\chapter{Zusammenfassung und Ausblick}
\label{6}
In dieser Arbeit wurde das quantenmechanische Problem eines periodisch getriebenen harmonischen Oszillators näher betrachtet.
Nachdem die Floquet-Theorie in Grundzügen erklärt wurde, konnte das Floquet-Theorem anhand der Wellenfunktionen des getriebenen Oszillators verifiziert werden.
Mit einem Fourier-Reihen-Ansatz konnten die Quasienergien der Floquet-Theorie für eine beliebige periodische Treibkraft im System, in Abhängigkeit der Fourier-Koeffizienten der Kraft bestimmt werden.
Mit diesen kann unter anderem das zeitliche Mittel des Energie-Erwartungswertes berechnet werden, ohne diesen selbst zu kennen, wodurch ein einfacher erster Eindruck des Systems möglich ist.
Wir konnten z.\,B. frühzeitig feststellen, dass die Energie divergieren wird, wenn sich die Treibfrequenz nah der Oszillator-Eigenfrequenz befindet.

Nachdem die Erwartungswerte für den Ort, den Impuls und die Energie berechnet wurden, haben wir das System um einen weiteren angekoppelten Oszillator erweitert.
Die Lösung der Schrödinger-Gleichung gelang hier mit einer Koordinatentransformation, welche den Hamilton-Operator in eine Summe aus zwei unabhängigen Hamilton-Operatoren überführte.
Damit wussten wir, dass die Lösung durch ein Produkt der Wellenfunktionen in den neuen Koordinaten gegeben ist, was für beliebig viele Hamilton-Operatoren ebenso korrekt ist.
Die Form der Koordinaten-Transformation kann hierbei aus den klassischen Normalmoden gewonnen werden, welche wiederum durch das Diagonalisieren einer Matrix bei der Lösung des klassichen Problems erhalten werden.
Dies sollte ebenso für beliebig viele gekoppelte Oszillatoren funktionieren, womit die Wellenfunktionen für ein System aus $n$ gekoppelten Oszilaltoren genauso aufgestellt werden können.
Für ungleiche Massen und Potentialkonstanten ergeben sich im Allgemeinen aber kompliziertere Normalmoden und Transformationen.

Für die zwei gekoppelten getriebenen Oszillatoren haben wir, unter Ausnutzung der neuen Variablen und der Erwartungswerte des einzelnen Oszillators, ebenfalls die Erwartungswerte berechnet und zudem visualisiert.
Es ergab sich an vielen Stellen ein Verhalten der Erwatungswerte wie es aus der klassichen Mechanik zu erwarten war.

Zum Schluss der Arbeit wurde ein weiteres Mal der einzelne getriebene Oszillator behandelt.
Diesmal wurde das Problem in der 2. Quantisierung betrachtet.
Mit einem geeigneten Ansatz, welchen wir anhand der Lösung in Ortsdarstellung motiviert haben, haben wir das darstellungsunabhängige Wellenfuktion-Ket aufstellen können.
In der 2. Quantisierung haben wir damit die gleichen Erwartungswerte wie in der Ortsdarstellung erhalten.
Um mehrere gekoppelte Oszillatoren in der Besetzungszahldarstellung zu betrachten, kann die gleiche Transformation für die Operatoren gewählt werden wie zuvor in der Ortsdarstellung.
Weil die Operatoren beim Wellenfunktion-Ket nur im Exponenten vorkommen, führt dies sofort auf die Produktform.
%Der Pduktansatz funktioniert für beliebig viele unabhängige Operatoren.
%Die Koordinatentranformation


\appendix
% Hier beginnt der Anhang, nummeriert in lateinischen Buchstaben
%\chapter{Ein Anhangskapitel}

%Hier könnte ein Anhang stehen, falls Sie z.B. Code, Konstruktionszeichnungen oder Ähnliches mit in die Arbeit bringen wollen. Im Normalfall stehen jedoch alle Ihre Resultate im Hauptteil der Bachelorarbeit und ein Anhang ist überflüssig.
\iffalse
\section*{Danksagung}
Bedanken möchte ich mich an dieser Stelle bei allen, die mich bei der Entstehung dieser Arbeit unterstützt haben.
Mein Dank gebührt zuallererst Herrn Prof. Dr. Anders für die Vergabe der Bachelorarbeit und dafür, dass er sich trotz seines engen Zeitplans mindestens einmal pro Woche Zeit für mich genommen hat.
Ebenso möchte ich mich bei Herrn Prof. Dr.      für die Zweitkorrektur dieser Arbeit bedanken.
Mein Dank gilt weiterhin Frau Iris Kleinjohann die immer Zeit für meine Fragen hatte und mit ihrer tatkräftigen Betreuung zum Gelingen dieser Arbeit beigetragen hat.
Auch bei allen, die diese Bachelorarbeit Korrektur gelesen haben möchte ich mich bedanken, wobei vorallem Julian Hochhaus und Niko Salewski zu nennen sind.
Nicht zuletzt gilt mein Dank meiner Familie und meinen Freunden, für die Unterstützung während des Studiums.
\fi


\backmatter
\printbibliography

\cleardoublepage
\input{content/eid_versicherung.tex}
\end{document}
