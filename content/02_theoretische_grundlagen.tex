\chapter{Theoretische Grundlagen}


\section{Floquet-Theorie}
  Die Floquet-Theorie ist ein nützliches Werkzeug zur der Lösung von quantenmechanischen Systemen, welche durch einen zeitlich periodischen Hamilton-Operator
  \begin{equation}
    H(t) = H(t+T) \; ,
  \end{equation}
  mit der Periode $T$, beschrieben werden.

  Das Floquet-Theorem besagt, dass bei einem solchen System, die Lösungen $\Psi_n(x,t)$ der Schrödinger-Gleichung
  \begin{equation}
    \text{i}\hbar\frac{\partial}{\partial t}\Psi_n(x,t) = H(t)\Psi_n(x,t)
    \label{schroedinger}
  \end{equation}
  in Ortsdarstellung die Form
  \begin{equation}
    \Psi_n(x,t) = \text{e}^{-\frac{i}{\hbar}\epsilon_nt}\Phi_n(x,t)
    \label{floquet_theorem}
  \end{equation}
  haben.
  Hierbei sind $\Phi_n(x,t) = \Phi_n(x,t+T)$ $T$-periodische Funktionen, die sogenannten Floquet-Moden, und $\epsilon_n$ die zugehörigen reellen Quasienergien, wobei diese Bezeichnungen gewählt wurden aufgrund der Parallele zu den Bloch-Moden und Quasiimpulsen des Bloch-Theorems \cite{haengi}.
  Das Floquet-Theorem kann damit als "Bloch-Theorem in der Zeit" aufgefasst werden \cite{sherly}.

  Durch Einsetzen dieses Ansatzes für die Wellenfunktionen (\ref{floquet_theorem}) in die Schrödingergleichung (\ref{schroedinger}) erhalten wir
  \begin{equation}
    \epsilon_n \Phi_n(x,t) = \left(H(t)-\text{i}\hbar\frac{\partial}{\partial t}\right)\Phi_n(x,t) = \cal{H}(t) \Phi_n(x,t) \; .
    \label{eigenwertproblem}
  \end{equation}
  Die Lösung der Schrödinger Gleichung konnte somit auf die Lösung eines Eigenwertproblems für den neuen Operator $\cal{H}(t)$ zurückgeführt werden \cite{sherly}.

  Die hermitischen Operatoren $H(t)$ und $\cal H(t)$ operieren auf dem Hilbertraum $\cal{L}^2 \otimes \cal{T}$.
  Dabei ist $\cal{L}^2$ der Raum der quadratintegrablen Funktionen und $\cal{T}$ der Raum der auf $[0,T]$ integrablen Funktionen, da die Operatoren $T$-periodisch sind \cite{haenggi}.
  Nach dem Spektralsatz bilden die Eigenfunktionen $\Phi_n(x,t)$ von $\cal{H}(t)$ eine Orthogonalbasis von $\cal{L}^2 \otimes \cal{T}$, welche auf eine Orthonormalbasis normiert werden kann, wodurch wir das Skalarprodukt definieren k"onnen als:
  %Es ergibt sich
  \begin{align}
    \begin{split}
    \braket{\braket{\Phi_n(x,t)|\Phi_m(x,t)}} &= \frac{1}{T} \int_0^T \braket{\Phi_n(x,t)|\Phi_n(x,t)} \text{d}t \\
    &= \frac{1}{T} \int_0^T \int_{-\infty}^{\infty} \Phi_n^*(x,t)\Phi_m(x,t) \: \text{d}x\text{d}t = \delta_{n,m} \; .
    \label{skalarprodukt_einzelner}
    \end{split}
  \end{align}



  \subsection{\texorpdfstring{Zeitlich gemittelter Erwartungswert der Energie $\bar{H}_n$}{Zeitlich gemittelter Erwartungswert der Energie bar{H}_n}}

    Da $H(t)$ nicht zeitlich konstant ist, sind auch dessen Erwartungswerte, die Energien des Systems, zeitabhängig.
    Ein direkter Vorteil der Floquet-Theorie liegt darin, dass sich die durchschnittliche Energie
    \begin{equation}
      \bar{H}_n  = \braket{\braket{\Psi_n(x,t)|H(t)|\Psi_m(x,t)}}
      %\label{mittleres_H}
    \end{equation}
    des $n$-ten Zustandes $\Psi_n(x,t)$ leicht über die Quasienergien $\epsilon_n$ berechnen lässt, ohne explizit Integrale zu lösen.

    Dazu ersetzen wir $H(t)$ mit Hilfe von (\ref{eigenwertproblem}).
    Außerdem unterscheiden sich die Floquet-Moden $\Phi_n(x,t)$ und die Wellenfunktionen $\Psi_n(x,t)$ nur durch eine komplexe Phase, daher sind deren Skalarprodukte identisch.
    Weiterhin benutzen wir (\ref{eigenwertproblem}), dass die Floquet-Moden Eigenfunktionen von $\cal{H}(t)$ sind:
    \begin{align}
      \begin{split}
      \bar{H}_n  &= \braket{\braket{\Phi_n(x,t)|\cal{H}(t)+\text{i}\hbar\frac{\partial}{\partial t}|\Phi_n(x,t)}} \\
      &=\epsilon_n \braket{\braket{\Phi_n(x,t)|\Phi_n(x,t)}} + \braket{\braket{\Phi_n(x,t)|\text{i}\hbar\frac{\partial}{\partial t}|\Phi_n(x,t)}} \\
      &= \epsilon_n + \braket{\braket{\Phi_n(x,t)|\text{i}\hbar\frac{\partial}{\partial t}|\Phi_n(x,t)}} \; .
    \end{split}
    \end{align}

    Nähere Betrachtung zeigt, dass
    \begin{equation}
      \text{i}\hbar\frac{\partial}{\partial t} = -\omega \frac{\partial \cal{H}(t)}{\partial \omega}
    \end{equation}
    gilt \cite{haenggi}.
    Da für $\epsilon_n$ und $\cal H(t)$ mit (\ref{eigenwertproblem}) eine Eigenwertgleichung vorliegt, kann das Hellman-Feynman-Theorem angewendet werden \cite{hellmann online quelle}.
    Dieses gibt eine Verbindung zwischen den Ableitungen der Eigenwerte und der Ableitung des Hamilton-Operators an.
    In unserem Fall erhalten wir dadurch
    \begin{equation}
      \frac{\partial \epsilon_n}{\partial \omega} = \braket{\braket{\Phi_n(x,t)|\frac{\partial \cal H(t)}{\partial \omega}|\Phi_n(x,t)}} \; .
    \end{equation}
    Damit folgt \cite{haenggi}:
    \begin{equation}
      \bar{H}_n = \epsilon_n - \omega\frac{\partial \epsilon_n}{\partial \omega} \; .
      \label{mittleres_H}
    \end{equation}

  \newpage



  \section{Allgemeine Lösung der Schrödinger-Gleichung des getriebenen harmonischen Oszillators in der Quantenmechanik}
    \label{lsg_einzelner}
    Der Hamilton-Operator eines harmonischen Oszillators der Masse $m$, welcher mit einer beliebigen aber periodischen äußeren Kraft $S(t)=S(t+T)$ getrieben wird, hat die Form
    \begin{equation}
      H(t) = H(t+T) = \frac{p^2}{2m} + \frac{1}{2}m\omega_0^2x^2-S(t)x \; ,
    \end{equation}
    mit $\omega_0=\sqrt{k/m}$.
    Dieses System kann exakt gelöst werden, indem die Schrödinger-Gleichung durch einen Variablenwechsel und zwei unitäre Transformationen auf die bekannte Form des ungetriebenen Oszillators reduziert wird \cite{haenggi}.

    \textbf{1) Variablenwechsel}\\
    Für den neuen Ortsoperator bzw. die Ortsvariable wird eine zeitabhängige Verschiebung angesetzt:
    \begin{equation}
      x \rightarrow y=x-\zeta(t) \; ;
    \end{equation}
    Wie zu erwarten verändert sich der Impuls(operator) durch die Translation im Ort nicht, da $\zeta(t)$ bei der Ortsableitung wegfällt.
    %wobei der Impuls(operator) unverändert bleibt.
    %Der Impulsoperator bleibt dadurch unverändert.

    Mit der neuen Zeitableitung der Wellenfunktion
    \begin{equation}
      \text{i}\hbar \frac{\partial}{\partial t} \Psi(y(t),t) = \text{i}\hbar \dot{\Psi} -\dot{\zeta}\frac{\partial}{\partial y}\Psi(y(t),t)
    \end{equation}
    wird die Schrödinger-Gleichung zu:
    \begin{equation}
      \text i \hbar \dot{\Psi}(y,t) = \left[\text i \hbar \dot{\zeta}\frac{\partial}{\partial y}-\frac{\hbar^2}{2m}\frac{\partial^2}{\partial y^2}+\frac{1}{2}m\omega_0^2(y+\zeta)^2-(y+\zeta)S(t)\right]\Psi(y,t) \; .
      \label{schroedinger_einzeln_getrieben}
    \end{equation}

    \textbf{2) Unitäre Tranformation für $\Psi(y,t)$}\\
    Im Weiteren wählen wir die unitäre Transformation
    \begin{equation}
      \Psi(y,t) = \text e^{\frac{\text i}{\hbar}m\dot \zeta y}\Lambda(y,t) \; .
    \end{equation}
    Durch Einsetzen in die Schrödinger-Gleichung (\ref{schroedinger_einzeln_getrieben}) und Ausrechnen der Ableitungen erhalten wir
    \begin{align}
      \begin{split}
        &\text e^{\frac{\text i}{\hbar}m\dot \zeta y}(\text i \hbar \dot \Lambda(y,t) - my \ddot{\zeta}\Lambda(y,t)) =\\
         &\text e^{\frac{\text i}{\hbar}m\dot \zeta y} \left[\left(-m\dot \zeta^2 \Lambda(y,t) + \text i \hbar \dot \zeta \frac{\partial}{\partial y} \Lambda(y,t) \right) \right. \\
         &\left. + \left(\frac{1}{2}m\dot \zeta^2 \Lambda(y,t) - \text{i} \hbar \dot \zeta \frac{\partial}{\partial y} \Lambda(y,t) - \frac{\hbar}{2m}\frac{\partial^2}{\partial y^2} \Lambda(y,t)  \right)
        + \frac{1}{2}m\omega_0^2(y+\zeta)^2  + (y+\zeta)S(t)  \right] \; .
      \end{split}
    \end{align}
    \iffalse
    Durch Einsetzen in die Schrödinger-Gleichung und Ausrechnen der Ableitungen erhalten wir für die linke Seite von (\ref{schroedinger_einzeln_getrieben})
    \text e^{\frac{\text i}{\hbar}m\dot \zeta y}(\text i \hbar \dot \Lambda(y,t) - my \ddot{\zeta}\Lambda(y,t))
    \begin{equation}
    \end{equation}
    und für die rechte Seite
    \begin{align}
      \begin{split}
      \text e^{\frac{\text i}{\hbar}m\dot \zeta y} \left[\left(-m\dot \zeta^2 \Lambda(y,t) + \text i \hbar \dot \zeta \frac{\partial}{\partial y} \Lambda(y,t) \right) + \left(\frac{1}{2}m\dot \zeta^2 \Lambda(y,t) - \text{i} \hbar \dot \zeta \frac{\partial}{\partial y} \Lambda(y,t) - \frac{\hbar}{2m}\frac{\partial^2}{\partial y^2} \Lambda(y,t)  \right) \right. \\
      \left. + \left(\frac{1}{2}m\omega_0^2y^2\Lambda(y,t) + m\omega_0^2y\zeta\Lambda(y,t) + \frac{1}{2}m\omega_0^2 \zeta^2\Lambda(y,t)) \right) + \left(-yS(t)\Lambda(y,t) - \zeta S(t)\Lambda(y,t) \right) \right]
    \end{split}
    \end{align}
    \fi

    \newpage

    Indem auf beiden Seiten durch die Exponentialfunktion geteilt wird, bekommen wir eine Differentialgleichung für $\Lambda(y,t)$.
    Außerdem können durch geschicktes Umsortieren der Terme die Lagrange-Funktion $L(\zeta,\dot \zeta, t)$
    \begin{equation}
      L(\zeta,\dot \zeta, t) = \frac{1}{2}m\dot \zeta^2 - \frac{1}{2}m\omega_0^2\zeta^2 + S(t)\zeta
      \label{lagrange_zeta}
    \end{equation}
      sowie die Bewegungsgleichung des klassischen getriebenen harmonischen  Oszillators \cite{husimi}
      \begin{equation}
        m\ddot \zeta + m\omega_0^2\zeta - S(t) = 0
        \label{dgl_zeta}
      \end{equation}
    für die Verschiebung $\zeta(t)$ identifiziert werden.
    Die Differentialgleichung für $\Lambda(y,t)$ ist folglich
    \begin{equation}
      \text i \hbar \dot \Lambda(y,t) = \left[ \frac{-\hbar^2}{2m}\frac{\partial^2}{\partial y^2} + \frac{1}{2}m\omega_0^2y^2 + (m\ddot \zeta + m\omega_0^2y\zeta - S(t))y - L(\zeta,\dot \zeta, t) \right]\Lambda(y,t) \; .
      \label{dgl_lambda}
    \end{equation}
    Um die Gleichung zu vereinfachen, wählen wir $\zeta(t)$ nun so, dass es gerade die klassische Bewegungsgleichung erfüllt, der entsprechende Term in (\ref{dgl_lambda}) also verschwindet.
    Nur noch die Lagrange-Funktion unterscheidet diese Differential-Gleichung von der des ungetriebenen Oszillators.

    \textbf{3) Unitäre Transformation für $\Lambda(y,t)$}\\
    Zuletzt wählen wir den Ansatz
    \begin{equation}
      \Lambda(y,t) = \text e^{\frac{\text i}{\hbar}\int_0^tL \: \text d t'}\chi(y,t)
    \end{equation}
    für $\Lambda(y,t)$, um die Lagrange-Funktion in (\ref{dgl_lambda}) zu eliminieren.
    Dadurch wird diese Differential-Gleichung für $\Lambda(y,t)$ bzw. die ursprüngliche Schrödinger-Gleichung auf einen ungetriebenen Oszillator für $\chi(y,t)$ zurückgeführt:
    \begin{equation}
      \text i \hbar \dot \chi(y,t) = \left[ \frac{-\hbar^2}{2m}\frac{\partial^2}{\partial y^2} + \frac{1}{2}m\omega_0^2y^2 \right]\chi(y,t) \; .
    \end{equation}
    Das bedeutet die $\chi_n(y,t)$ sind die Wellenfunktionen des ungetriebenen Oszillators und die Gesamtlösung der Schrödinger-Gleichung des getriebenen Oszillators ist damit gegeben durch
    \begin{align}
      \begin{split}
      \Psi_n(x,t) &= \Psi_n(y=x-\zeta(t),t) \\
      &= N_nO_n\left(\sqrt{\frac{m\omega_0}{\hbar}}(x-\zeta(t))\right) \text e^{\frac{-m\omega_0}{2\hbar}(x-\zeta(t))^2} \\
      &\quad \cdot \text e^{\frac{\text i}{\hbar}\left(m\dot \zeta(t)(x-\zeta(t))-E_nt+\int_0^tL(\dot \zeta,\zeta,t')\:\text dt'\right)} \; , \\
      &\quad \quad n \in \mathbb{N}_0 \; .
    \end{split}
    \end{align}
    Dabei sind $O_n$ die Hermit-Polynome, $E_n = \hbar \omega_0(n+1/2)$ die bekannten Eigenenergien der Zustände des ungetriebenen Oszillators und
    \begin{equation}
      N_n = \left(\frac{m\omega_0}{\pi \hbar}\right) \frac{1}{\sqrt{2^nn!}}
    \end{equation}
    dessen Normierungsfaktoren.
    Die Lösungen des getriebenen Oszillators sind somit bezüglich des Skalarproduktes (\ref{skalarprodukt_einzelner}) normiert.

    Die Lösung entspricht damit einem, um die klassiche Lösung $\zeta(t)$ verschobenen, ungetriebenen Oszillator, mit einer zusätzlichen zeit- und ortsabhängigen komplexen Phase.
    Die treibende Kraft $S(t)$ geht in die klassiche Lösung $\zeta(t)$ und direkt, über das Wirkungsintegral, in die komplexe Phase ein.


    \subsection{\texorpdfstring{Identifizierung der Quasienergien $\epsilon_n$ und Floquet-Moden $\Phi_n(x,t)$}{Identifizierung der Quasienergien \epsilon_n und Floquet-Moden \Phi_n(x,t)}}
      Nach dem Floquet-Theorem für periodische Hamilton-Operatoren (\ref{floquet_theorem}) kann die Lösung des getriebenen Oszillators geschrieben werden als
      \begin{equation}
        \Psi_n(x,t) = \text e^{\frac{-\text i}{\hbar}\epsilon_nt}\Phi_n(x,t) \; ,
      \end{equation}
      mit $\Phi_n(x,t)=\Phi_n(x,t+T)$.
      Nun können wir alle $T$-periodischen Terme als $\Phi_n(x,t)$ identifizieren, und alle Terme im Exponenten, die linear in $t$ sind, als $-\text i\epsilon_n t/\hbar$ \cite{haenggi}.

      Alle Funktionen von $(x-\zeta(t))$ haben die Periode $T$, da $\zeta(T)$ als Lösung der klassichen Bewegungsgleichung mit $S(t)=S(t+T)$ die Periode $T$ hat. Das
      Ergebnis des Integrals über die Lagrange-Funktion kann nur $T$-periodisch oder linear in $t$ sein, deshalb sind die Quasienergien gegeben durch die $E_n$ und den linearen Teil des Integrals:
      \begin{equation}
        \epsilon_n = E_n - \frac{1}{T} \int_0^T L(\dot \zeta, \zeta, t) \: \text d t \; .
      \end{equation}
      Die Floquet-Moden sind demnach
      \begin{align}
        \begin{split}
          \Phi_n(x,t) &=
           N_nH_n\left(\sqrt{\frac{m\omega_0}{\hbar}}(x-\zeta(t))\right) \text e^{\frac{-m\omega_0}{2\hbar}(x-\zeta(t))^2} \\
          &\quad \cdot \text e^{\frac{\text i}{\hbar}\left(m\dot \zeta(t)(x-\zeta(t))+\int_0^tL(\dot \zeta,\zeta,t')\:\text d t-\frac{t}{T} \int_0^T L(\dot \zeta,\zeta,t)\:\text dt\: \right)} \; , \\
          &\quad \quad n \in \mathbb{N}_0 \; .
        \end{split}
      \end{align}


    \subsection{Quasienergien für eine sinusoidiale Treibkraft}
      Hier wird ein Beispiel einer treibenden Kraft diskutiert \cite{haenggi}:
      \begin{equation}
        S(t) = S(t+T) = A\sin(\omega t),
      \end{equation}
      wobei $T=2\pi / \omega$ ist.
      Setzen wir die allgemeine homogene Lösung gleich null, wird die Lösung $\zeta(t)$ der klassischen Bewegungsgleichung (\ref{dgl_zeta}) zu \cite{mads}
      \begin{equation}
        \zeta(t) = \frac{A\sin(\omega t)}{m(\omega_0^2 - \omega^2)} \; .
      \end{equation}
      Das Berechnen des Wirkungsintegrals und anschließendes Identifizieren des linearen Anteils liefert die Quasienergien für die gegebene Kraft:
      \begin{equation}
        \epsilon_n  = \hbar \omega_0\left(n+\frac{1}{2}\right) - \frac{A}{4m(\omega_0^2-\omega^2)} \;.
      \end{equation}
      Wie zu erkennen streben die Quasienergien, und somit die mittlere Energie des Systems (\ref{mittleres_H}), gegen Unendlich, wenn sich die Treibfrequenz $\omega$ nahe der Eigenfrequenz des Oszillators $\omega_0 $ befindet.
      Da wir einen getriebenen Oszillator ohne Dämpfung betrachten, war dieses Ergebnis zu erwarten \cite{mads}.



























  b
