\chapter{Theoretische Grundlagen}


\section{Floquet Theorie und Floquet'sches Theorem}
  Die Floquet Theorie ist ein n"utzliches Werkzeug zur der L"osung von quantenmechanischen Systemen, welche durch einen zeitlich periodischen Hamilton-Operator
  \begin{equation}
    H(t) = H(t+T) \; ,
  \end{equation}
  mit der Periode $T$, beschrieben werden.

  Das Floquet'sche Theorem besagt, dass bei einem solchen Hamoltionian, die L"osungen $\Psi_n(x,t)$ der Schr"odinger-Gleichung
  \begin{equation}
    \text{i}\hbar\frac{\partial}{\partial t}\Psi_n(x,t) = H(t)\Psi_n(x,t)
    \label{schroedinger}
  \end{equation}
  in Ortsdarstellung die Form
  \begin{equation}
    \Psi_n(x,t) = \text{e}^{-\frac{i}{\hbar}\epsilon_nt}\Phi_n(x,t)
    \label{floquet_theorem}
  \end{equation}
  haben.
  Hierbei sind die $\Phi_n(x,t) = \Phi_n(x,t+T)$ $T$-periodische Funktionen, die sogenannten Floquet-Moden, und $\epsilon_n$ die zugeh"origen reellen Quasienergien, wobei diese Bezeichnungen gew"ahlt wurden aufgrund der Parallele zu den Bloch-Moden und Quasiimpulsen des Bloch-Thoerems \cite{haenlagrangei}.
  Das Floquet-Theorem kann damit als "Bloch-Theorem in der Zeit" aufgefasst werden.

  Durch Einsetzen dieses Ansatzes f"ur die Wellenfunktionen (\ref{floquet_theorem}) in die Schr"odingergleichung (\ref{schroedinger}) erhalten wir
  \begin{equation}
    \epsilon_n \Phi_n(x,t) = (H(t)-\text{i}\hbar\frac{\partial}{\partial t})\Phi_n(x,t) = \cal{H}(t) \Phi_n(x,t) \; .
    \label{eigenwertproblem}
  \end{equation}
  Die L"osung der Schr"odinger Gleichung konnte somit auf die L"osung eines Eigenwertproblems f"ur den neuen hermitischen Operator $\cal{H}(t)$ zur"uckgef"uhrt werden \cite{sherly}.

  $\cal{H}(t)$ bzw. $H(t)$ ist hermitisch und operiert auf dem Hilbertraum $\cal{L}^2 \otimes \cal{T}$.
  $\cal{L}^2$ ist der Raum der quadratintegrablen Funktionen und $\cal{T}$ der Raum der auf $[0,T]$ integrablen Funktionen, da die Operatoren $T$ periodisch sind \cite{haenggi}.
  Nach dem Spekralsatz bilden die Eigenfunktionen $\Phi_n(x,t)$ von $\cal{H}(t)$ eine Orthogonalbasis von $\cal{L}^2 \otimes \cal{T}$, welche auf eine Orthonormalbasis normiert werden kann:
  %Es ergibt sich
  \begin{equation}
    \braket{\braket{\Phi_n(x,t)|\Phi_m(x,t)}} = \frac{1}{T} \int_0^T \int_{-\infty}^{\infty} \Phi_n^*(x,t)\Phi_m(x,t) \: \text{d}x\text{d}t = \delta_{n,m} \; .
    \label{skalarprodukt_einzelner}
  \end{equation}



  \subsection{\texorpdfstring{Zeitlich gemittelter Erwartungswert der Energie $\bar{H}_n$}{Zeitlich gemittelter Erwartungswert der Energie bar{H}_n}}

    Da $H(t)$ nicht zeitlich konstant ist, sind auch seine Observablen, die Energien des Systems, zeitabh"angig.
    Ein direkter Vorteil der Floquet-Theorie liegt darin, dass die durchschnittliche Energie
    \begin{equation}
      \bar{H}_n  = \braket{\braket{\Psi_n(x,t)|H(t)|\Psi_m(x,t)}}
      %\label{mittleres_H}
    \end{equation}
    des $n$-ten Zustandes $\Psi_n(x,t)$ leicht "uber die Quasienergien $\epsilon_n$ berechnen l"asst, ohne explizit Integrale zu l"osen.

    Dazu ersetzen wir $H(t)$ mit Hilfe von (\ref{eigenwertproblem}), au"serdem unterscheiden sich die Floquet-Moden $\Phi_n(x,t)$ und die Wellenfunktionen $\Psi_n(x,t)$ nur durch eine komplexe Phase, daher sind deren Skalarprodukte identisch.
    Weiterhin benutzen wir (\ref{eigenwertproblem}), dass die Floquet-Moden Eigenfunktionen von $\cal{H}(t)$ sind:
    \begin{align}
      \begin{split}
      \bar{H}_n  &= \braket{\braket{\Phi_n(x,t)|\cal{H}(t)+\text{i}\hbar\frac{\partial}{\partial t}|\Phi_n(x,t)}} \\
      &=\epsilon_n \braket{\braket{\Phi_n(x,t)|\Phi_n(x,t)}} + \braket{\braket{\Phi_n(x,t)|\text{i}\hbar\frac{\partial}{\partial t}|\Phi_n(x,t)}} \\
      &= \epsilon_n + \braket{\braket{\Phi_n(x,t)|\text{i}\hbar\frac{\partial}{\partial t}|\Phi_n(x,t)}} \; .
    \end{split}
    \end{align}

    N"ahere Betrachtung zeigt, dass
    \begin{equation}
      \text{i}\hbar\frac{\partial}{\partial t} = -\omega \frac{\partial \cal{H}(t)}{\partial \omega}
    \end{equation}
    gilt \cite{haenggi}.
    Da f"ur $\epsilon_n$ und $\cal H(t)$ mit (\ref{eigenwertproblem}) eine Art station"are Schr"oginger-Gleichung vorliegt, kann das Hellman-Feynman Theorem angewendet werden \cite{hellmann online quelle}.
    Dieses gibt eine Verbindung zwischen den Ableitungen der Eigenwerte und der Ableitung des Hamiltom-Operators.
    In unserem Fall erhalten wir dadurch
    \begin{equation}
      \frac{\partial \epsilon_n}{\partial \omega} = \braket{\braket{\Phi_n(x,t)|\frac{\partial \cal H(t)}{\partial \omega}|\Phi_n(x,t)}}
    \end{equation}
    Damit folgt \cite{haenggi}:
    \begin{equation}
      \bar{H}_n = \epsilon_n - \omega\frac{\partial \epsilon_n}{\partial \omega} \; .
      \label{mittleres_H}
    \end{equation}

  \newpage



  \section{Allgemeine L"osung der Schr"odinger-Gleichung des getriebenen harmonischen Oszillators in der Quantenmechanik}
    Der Hamilton-Operator eines harmonischen Oszillators der Masse $m$, welcher mit einer beliebigen aber periodischen "au"seren Kraft $S(t)=S(t+T)$ getrieben wird, hat die Form
    \begin{equation}
      H(t) = H(t+T) = \frac{p^2}{2m} + \frac{1}{2}m\omega_0^2x^2-S(t)x \; ,
    \end{equation}
    mit $\omega_0=\sqrt{k/m}$.
    Dieses System kann exakt gel"ost werden, indem die Schr"odinger-Gleichung durch einen Variablenwechsel und zwei unit"are Transformationen auf die bekannte Form des ungetriebenen Oszillators reduziert wird \cite{haenggi}.

    \textbf{1) Variablenwechsel}\\
    F"ur den neuen Ortsoperator bzw. die Ortsvariable wird eine zeitabh"angige Verschiebung angesetzt:
    \begin{equation}
      x \rightarrow y=x-\zeta(t) \; ;
    \end{equation}
    Wie zu erwarten ver"andert sich der Impuls(operatur) durch die Translation im Ort nicht, da $zeta(t)$ bei der Ortsableitung wegf"allt.
    %wobei der Impuls(operator) unver"andert bleibt.
    %Der Impulsoperator bleibt dadurch unver"andert.

    Mit der neuen Zeitableitung der Wellenfunktion
    \begin{equation}
      \text{i}\hbar \frac{\partial}{\partial t} \Psi(y(t),t) = \text{i}\hbar \dot{\Psi} -\dot{\zeta}\frac{\partial}{\partial y}\Psi(y(t),t)
    \end{equation}
    wird die Schr"odinger-Gleichung zu:
    \begin{equation}
      \text i \hbar \dot{\Psi}(y,t) = \left[\text i \hbar \dot{\zeta}\frac{\partial}{\partial y}-\frac{\hbar^2}{2m}\frac{\partial^2}{\partial y^2}+\frac{1}{2}m\omega_0^2(y+\zeta)^2-(y+\zeta)S(t)\right]\Psi(y,t) \; .
      \label{schroedinger_einzeln_getrieben}
    \end{equation}

    \textbf{2) Unit"are Trafe f"ur $\Psi(y,t)$}\\
    Im Weiteren w"ahlen wir die unit"are Transformation
    \begin{equation}
      \Psi(y,t) = \text e^{\frac{\text i}{\hbar}m\dot \zeta y}\Lambda(y,t) \; .
    \end{equation}
    Durch Einsetzen in die Schr"odinger-Gleichung (\ref{schroedinger_einzeln_getrieben}) und Ausrechnen der Ableitungen erhalten wir
    \begin{align}
      \begin{split}
        \text e^{\frac{\text i}{\hbar}m\dot \zeta y}(\text i \hbar \dot \Lambda(y,t) - my \ddot{\zeta}\Lambda(y,t))  \quad\quad \quad \quad \quad \quad \quad \quad \quad \quad\quad\quad\quad\quad \quad \quad \quad\quad\quad\quad\\
         = \text e^{\frac{\text i}{\hbar}m\dot \zeta y} \left[\left(-m\dot \zeta^2 \Lambda(y,t) + \text i \hbar \dot \zeta \frac{\partial}{\partial y} \Lambda(y,t) \right) + \left(\frac{1}{2}m\dot \zeta^2 \Lambda(y,t) - \text{i} \hbar \dot \zeta \frac{\partial}{\partial y} \Lambda(y,t) - \frac{\hbar}{2m}\frac{\partial^2}{\partial y^2} \Lambda(y,t)  \right) \right. \\
        \left. + \frac{1}{2}m\omega_0^2(y+\zeta)^2  + (y+\zeta)S(t)  \right] \; .
      \end{split}
    \end{align}
    \iffalse
    Durch Einsetzen in die Schr"odinger-Gleichung und Ausrechnen der Ableitungen erhalten wir f"ur die linke Seite von (\ref{schroedinger_einzeln_getrieben})
    \text e^{\frac{\text i}{\hbar}m\dot \zeta y}(\text i \hbar \dot \Lambda(y,t) - my \ddot{\zeta}\Lambda(y,t))
    \begin{equation}
    \end{equation}
    und f"ur die rechte Seite
    \begin{align}
      \begin{split}
      \text e^{\frac{\text i}{\hbar}m\dot \zeta y} \left[\left(-m\dot \zeta^2 \Lambda(y,t) + \text i \hbar \dot \zeta \frac{\partial}{\partial y} \Lambda(y,t) \right) + \left(\frac{1}{2}m\dot \zeta^2 \Lambda(y,t) - \text{i} \hbar \dot \zeta \frac{\partial}{\partial y} \Lambda(y,t) - \frac{\hbar}{2m}\frac{\partial^2}{\partial y^2} \Lambda(y,t)  \right) \right. \\
      \left. + \left(\frac{1}{2}m\omega_0^2y^2\Lambda(y,t) + m\omega_0^2y\zeta\Lambda(y,t) + \frac{1}{2}m\omega_0^2 \zeta^2\Lambda(y,t)) \right) + \left(-yS(t)\Lambda(y,t) - \zeta S(t)\Lambda(y,t) \right) \right]
    \end{split}
    \end{align}
    \fi

    \newpage

    Indem aud beiden Seiten durch die Exponentialfunktion geteilt wird, erhalten wir eine Differentialgleichung f"ur $\Lambda(y,t)$.
    Au"serdem k"onnen durch geschicktes Umsortieren der Terme die Lagrange-Funktion $L(\zeta,\dot \zeta, t)$
    \begin{equation}
      L(\zeta,\dot \zeta, t) = \frac{1}{2}m\dot \zeta^2 - \frac{1}{2}m\omega_0^2\zeta^2 + S(t)\zeta
      \label{lagrange_zeta}
    \end{equation}
      sowie die Bewegungsgleichung des klassischen getriebenen harmonischen  Oszillators \cite{husimi}
      \begin{equation}
        m\ddot \zeta + m\omega_0^2\zeta - S(t) = 0
        \label{dgl_zeta}
      \end{equation}
    f"ur die Verschiebung $\zeta(t)$ identifiziert werden.
    Die Differentialgleichung f"ur $\Lambda(y,t)$ ist
    \begin{equation}
      \text i \hbar \dot \Lambda(y,t) = \left[ \frac{-\hbar^2}{2m}\frac{\partial^2}{\partial y^2} + \frac{1}{2}m\omega_0^2y^2 + (m\ddot \zeta + m\omega_0^2y\zeta - S(t))y - L(\zeta,\dot \zeta, t) \right]\Lambda(y,t) \; .
      \label{dgl_lambda}
    \end{equation}
    Um die Gleichung zu vereinfachen, w"ahlen wir $\zeta(t)$ nun so, dass es gerade die klassische Bewegungsgleichung erf"ullt, der entsprechende Term in (\ref{dgl_lambda}) also verschwindet.
    Nur noch die Lagrangge-Funktion unterscheidet diese Differential-Gleichung von der des ungetriebenen Oszillators.

    \textbf{3) Unit"are Trafe f"ur $\Lambda(y,t)$}\\
    Zuletzt w"ahlen wir den Ansatz
    \begin{equation}
      \Lambda(y,t) = \text e^{\frac{\text i}{\hbar}\int_0^tL \: \text d t'}\chi(y,t)
    \end{equation}
    f"ur $\Lambda(y,t)$, um die Lagrange-Funktion in (\ref{dgl_lambda}) zu elilminieren.
    Dadurch wird die Differential-Gleichung f"ur $\Lambda(y,t)$ bzw. die urspr"ungliche Schr"odinger-Gleichung auf einen ungetriebenen Oszillator f"ur $\chi(y,t)$ reduziert:
    \begin{equation}
      \text i \hbar \dot \chi(y,t) = \left[ \frac{-\hbar^2}{2m}\frac{\partial^2}{\partial y^2} + \frac{1}{2}m\omega_0^2y^2 \right]\chi(y,t) \; .
    \end{equation}
    Das bedeutet die $\chi_n(y,t)$ sind die Wellenfunktionen des ungetriebenen Oszillators und die Gesamtl"osung der Schro"dinger-Gleichung des getriebenen Oszillators ist damit gegeben durch
    \begin{align}
      \begin{split}
      \Psi_n(x,t) &= \Psi_n(y=x-\zeta(t),t) \\
      &= N_nH_n\left(\sqrt{\frac{m\omega_0}{\hbar}}(x-\zeta(t))\right) \text e^{\frac{-m\omega_0}{2\hbar}(x-\zeta(t))^2} \text e^{\frac{\text i}{\hbar}\left(m\dot \zeta(t)(x-\zeta(t))-E_nt+\int_0^tL(\dot \zeta,\zeta,t')\:\text dt'\right)} \; , \\
      &n \in \mathbb{N}_0 \; .
    \end{split}
    \end{align}
    Dabei sind $H_n$ die Hermit-Polynome, $E_n = \hbar \omega_0(n+1/2)$ die bekannten Eigenenergien des ungetriebenen Oszillators und
    \begin{equation}
      N_n = \left(\frac{m\omega_0}{\pi \hbar}\right) \frac{1}{\sqrt{2^nn!}}
    \end{equation}
    dessen Normierungsfaktoren.
    Die L"osungen des getriebenen Oszillators somit sind bez"uglich des Skalarproduktes (\ref{skalarprodukt_einzelner}) normiert.

    Die L"osung entspricht damit einem ,um die klassiche L"osung $\zeta(t)$ verschobenen, ungetriebenen Oszillator, mit einer zus"atlichen zeit- und ortsabh'angigen komplexen Phase.
    Die treibende Kraft $S(t)$ geht in die klassiche L"osung $\zeta(t)$ und direkt, "uber das Wirkungsintegral, in die komplexe Phase mit ein.


    \subsection{\texorpdfstring{Identifizierung der Quasienergien $\epsilon_n$ und Floquet-Moden $\Phi_n(x,t)$}{Identifizierung der Quasienergien \epsilon_n und Floquet-Moden \Phi_n(x,t)}}
      Nach dem Floquet'schen-Theorem f"ur periodische Hamilton-Operatoren (\ref{floquet_theorem}) kann die L"osung des getriebenen Oszillators geschrieben werden als
      \begin{equation}
        \Psi_n(x,t) = \text e^{\frac{-\text u}{\hbar}\epsilon_nt}\Phi_n(x,t) \; ,
      \end{equation}
      mit $\Phi_n(x,t)=\Phi_n(x,t+T)$.
      Nun k"onnen wir alle $T$-periodischen Terme als $\Phi_n(x,t)$ identifizieren, und alle Terme im Exponenten die linear in $t$ sind als $\frac{-\text i}{\hbar}\epsilon_n t$ \cite{haenggi}.

      Alle Funktionen von $(x-\zeta(t))$ haben die Periode $T$, da $\zeta(T)$ als L"osung der klassichen Bewegungsgleichung mit $S(t)=S(t+T)$ die Periode $T$ hat.
      Das Ergebnis des Integrals "uber die Lagrange Funktion kann nur $T$-periodisch oder linear in $t$ sein, deshalb sind die Quasienergien gegeben durch die $E_n$ und den linearen Teil des Integrals:
      \begin{equation}
        \epsilon_n = E_n - \frac{1}{T} \int_0^T L(\dot \zeta, \zeta, t) \: \text d t \; .
      \end{equation}
      Die Floquet-Moden sind
      \begin{align}
        \begin{split}
          &\Phi_n(x,t) \\
          &= N_nH_n\left(\sqrt{\frac{m\omega_0}{\hbar}}(x-\zeta(t))\right) \text e^{\frac{-m\omega_0}{2\hbar}(x-\zeta(t))^2} \text e^{\frac{\text i}{\hbar}\left(m\dot \zeta(t)(x-\zeta(t))+\int_0^tL(\dot \zeta,\zeta,t')\:\text d t-\frac{t}{T} \int_0^T L(\dot \zeta,\zeta,t)\:\text dt\: \right)} \; .
        \end{split}
      \end{align}


    \subsection{Quasienergien f'ur eine monochromatische Treibkraft}
      Hier wird ein Beispiel einer treibenden Kraft diskutiert;
      \begin{equation}
        S(t) = S(t+T) = Asin(\omega t),
      \end{equation}
      wobei $T=2\pi / \omega$ ist.
      Setzen wir die allgemeine homogene L"osung gleich null, wird die L"osung $\zeta(t)$ der klassischen Bewegungsgleichung (\ref{dgl_zeta}) zu \cite{mads}
      \begin{equation}
        \zeta(t) = \frac{A\sin(\omega t)}{m(\omega_0^2 - \omega^2)} \; .
      \end{equation}
      Das Berechnen des Wirkungsintegrals und anschlie"sendes Identifizieren des linearen Anteils liefert die Quasienergien f"ur die gegebene Kraft:
      \begin{equation}
        \epsilon_n  = \hbar \omega_0(n+\frac{1}{2}) - \frac{A}{4m(\omega_0^2-\omega^2)} \;.
      \end{equation}
      Wie zu erkennen streben die Quasienergien, und somit die mittlere Energie des Systems (\ref{mittleres_H}), gegen Unendlich, wenn sich die Treibfrequenz $\omega$ nahe der Eigenfrequentz $\omega_0 $ des Oszillators befindet.
      Da wir einen getriebenen Oszillator ohne D"ampfung betrachten, war dieses Ergebnis zu erwarten \cite{mads}.



























  b
