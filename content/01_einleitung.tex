\chapter{Einleitung}
\iffalse
Hier folgt eine kurze Einleitung in die Thematik der Bachelorarbeit.
Die Einleitung muss kurz sein, damit die vorgegebene Gesamtlänge der
Arbeit von 25 Seiten nicht überschritten wird.
Die Beschränkung der Seitenzahl sollte man ernst nehmen,
da Überschreitung zu Abzügen in der Note führen kann.
Um der Längenbeschränkung zu genügen, darf auch nicht an der Schriftgröße,
dem Zeilenabstand oder dem Satzspiegel (bedruckte Fläche der Seite) manipuliert werden.



nur wenige systeme exakt loesbar in qm harm oszi, wasserstoff. bei zeitabh hamiltonop keine stat schroed glgl  (eigenwertglg), gelingt aber bei periodischen mit floqet thoerie
\fi

Nur wenige System k"onnen in der Quantenmechanik exakt analytisch gel"ost werden.
Bei Systemen wie dem harmonischen Oszillator oder dem Wasserstoff-Atom, welche durch einen zeitunabh"angigen Hamilton-Operator beschrieben werden, wird zur die L"osung der Schr"odinger-Gleichung einen Produktansatz gew"ahlt, wonach haupts"achlich eine Eigenwertgleichung f"ur den Hamilton-Operator gel"ost werden muss.
Wenn der Hamilton-Operator allerdings zeitabh"angig ist, funktioniert dieser Ansatz nicht und es muss im Allgemeinen numerisch im Rahmen der zeitabh"angigen St"orungstheorie gerechnet werden.
Werden jedoch Probleme betrachtet wie ein einzelnes Atom oder Spin in einem oszillierenden "au"seren Feld (\cite{haengi},\cite{sherly}), bei denen der Hamilton-Operator periodisch in der Zeit ist, kann die L"osung unter Verwendung der Floquet-Theorie mit einem neuen Ansatz wieder auf eine Eigenwertgleichung gebracht werden, f"ur einen neuen zeitabh"angigen Operator.

In dieser Bachelorarbeit wird der periodisch getriebene harmonische Oszillator thematisiert.
In Kap. \ref{2} wird nach einer kurzen Einf"uhrung in die Floquet-Theorie die L"osung der Schr"odinger-Gleichung f"ur den getriebenen Oszillator in Ortsdarstellung dargestellt.
Anhand der Wellenfunktionen wird das Floquet-Theorem best"atigt und im Weiteren die Quasienergien f"ur eine beliebige Treibkraft und ein Beispiel einer sinusf"ormigen Treibkraft bestimmt.
Daraufhin werden in Kap. \ref{3} diverse Ewartungswerte des getriebenen Oszillators ermittelt.
Die Ergebnisse aus Kap. \ref{2} und \ref{3} werden danach genutzt um die L"osung sowie die Erwartungswerte eines Systems aus zwei gekoppelten getriebenen harmonischen Oszillatoren zu bestimmen,
Im letzten Teil der Arbeit (Kap. \ref{5}) werden die Zust"ande des einzelnen getriebenen Oszillators in der Besetzungszahldarstellung aufgestellt und erneut Erwartungswerte berechnet.
