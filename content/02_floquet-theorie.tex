\chapter{Floquet-Theorie des getriebenen harmonischen Oszillators}
Der Hamilton-Operator eines quantenmechanischen harmonischen Oszillators der Masse $m$, welcher mit einer beliebigen aber periodischen äußeren Kraft $S(t)=S(t+T)$ getrieben wird, hat die Form
\begin{equation}
  H(t) = H(t+T) = \frac{p^2}{2m} + \frac{1}{2}m\omega_0^2x^2-S(t)x \; ,
  \label{H_einzelner}
\end{equation}
mit $\omega_0=\sqrt{k/m}$ und der Potentialkonstante $k$.

Im Folgenden werden vorerst die allgemeinen Grundzüge der Floquet-Theorie erläutert, indem die entprechenden Teile aus \cite{haengi} und \cite{sherly} reproduziert wird.

Dann wird Schrödinger-Gleichung (\ref{H_einzelner}) für eine beliebige Treibkraft $S(t)$ gelöst, wobei die Ergebnisse aus \cite{haengi},\cite{husimi} und \cite{mads} reproduziert werden.
Weiterhin werden die Floquet-Moden $\Phi_n(x,t)$ identifiziert, genauso wie die Quasienergien $\epsilon_n$, welche für eine beliebige periodische Treibkraft und explizit für eine sinusförmige Treibkraft bestimmt werden.
\iffalse
Danach werden wir die Ewartungswerte $\braket{x}_n,\braket{x^2}_n,\braket{p}_n,\braket{p^2}_n$ und damit die Unschärfe berechnen, indem wir die bekannten Erwartungswerte des ungetriebenen Oszillators benutzen.
Ebenso werden wir den zeitabhängigen und gemittelten Erwartungswert der Energie $\braket{H}_n$ und $\bar H_n$ berechnen.
\fi

\section{Grundlagen}
  Die Floquet-Theorie \cite{haengi} ist ein nützliches Werkzeug zur der Lösung von quantenmechanischen Systemen, welche durch einen zeitlich periodischen Hamilton-Operator
  \begin{equation}
    H(t) = H(t+T) \; ,
  \end{equation}
  mit der Periode $T$, beschrieben werden.

  Das Floquet-Theorem besagt, dass bei einem solchen System, die Lösungen $\Psi_n(x,t)$ der Schrödinger-Gleichung
  \begin{equation}
    \text{i}\hbar\frac{\partial}{\partial t}\Psi_n(x,t) = H(t)\Psi_n(x,t)
    \label{schroedinger}
  \end{equation}
  in Ortsdarstellung die Form
  \begin{equation}
    \Psi_n(x,t) = \exp\left(-\frac{i}{\hbar}\epsilon_nt\right)\Phi_n(x,t)
    \label{floquet_theorem}
  \end{equation}
  haben.
  Hierbei sind $\Phi_n(x,t) = \Phi_n(x,t+T)$ $T$-periodische Funktionen, die sogenannten Floquet-Moden, und $\epsilon_n$ die zugehörigen reellen Quasienergien, wobei diese Bezeichnungen gewählt wurden aufgrund der Parallele zu den Bloch-Moden und Quasiimpulsen des Bloch-Theorems \cite{haengi}.
  Das Floquet-Theorem kann damit als "Bloch-Theorem in der Zeit" aufgefasst werden \cite{sherly}.

  Durch Einsetzen dieses Ansatzes für die Wellenfunktionen (\ref{floquet_theorem}) in die Schrödingergleichung (\ref{schroedinger}) erhalten wir
  \begin{equation}
    \epsilon_n \Phi_n(x,t) = \left(H(t)-\text{i}\hbar\frac{\partial}{\partial t}\right)\Phi_n(x,t) = \cal{H}(t) \Phi_n(x,t) \; .
    \label{eigenwertproblem}
  \end{equation}
  Die Lösung der Schrödinger-Gleichung kann somit auf die Lösung eines Eigenwertproblems für den neuen Operator $\cal{H}(t)$ zurückgeführt werden \cite{sherly}.

  Die hermitischen Operatoren $H(t)$ und $\cal H(t)$ operieren auf dem Hilbertraum $\cal{L}^2 \otimes \cal{T}$.
  Dabei ist $\cal{L}^2$ der Raum der quadratintegrablen Funktionen und $\cal{T}$ der Raum der auf $[0,T]$ integrablen Funktionen, da die Operatoren $T$-periodisch sind \cite{haengi}.
  Nach dem Spektralsatz bilden die Eigenfunktionen $\Phi_n(x,t)$ von $\cal{H}(t)$ eine Orthogonalbasis von $\cal{L}^2 \otimes \cal{T}$, welche auf eine Orthonormalbasis normiert werden kann, wodurch wir das Skalarprodukt definieren können als \cite{haengi}:
  %Es ergibt sich
  \begin{align}
    \begin{split}
    \braket{\braket{\Phi_n(x,t)|\Phi_m(x,t)}} &= \frac{1}{T} \int_0^T \braket{\Phi_n(x,t)|\Phi_n(x,t)} \text{d}t \\
    &= \frac{1}{T} \int_0^T \int_{-\infty}^{\infty} \Phi_n^*(x,t)\Phi_m(x,t) \: \text{d}x\text{d}t = \delta_{n,m} \; .
    \label{skalarprodukt_einzelner}
    \end{split}
  \end{align}


%war vorher subsection
 \subsection{\texorpdfstring{Zeitlich gemittelter Erwartungswert der Energie $\bar{H}_n$}{Zeitlich gemittelter Erwartungswert der Energie bar{H}_n}}

    Da $H(t)$ nicht zeitlich konstant ist, sind auch dessen Erwartungswerte, die Energien des Systems, zeitabhängig.
    Ein direkter Vorteil der Floquet-Theorie liegt darin, dass sich die durchschnittliche Energie
    \begin{equation}
      \bar{H}_n  = \braket{\braket{\Psi_n(x,t)|H(t)|\Psi_n(x,t)}}
      %\label{mittleres_H}
    \end{equation}
    des $n$-ten Zustandes $\Psi_n(x,t)$ leicht über die Quasienergien $\epsilon_n$ berechnen lässt, ohne explizit Integrale zu lösen.

    Dazu ersetzen wir $H(t)$ mit Hilfe von (\ref{eigenwertproblem}).
    Außerdem unterscheiden sich die Floquet-Moden $\Phi_n(x,t)$ und die Wellenfunktionen $\Psi_n(x,t)$ nur durch eine komplexe Phase, daher sind deren Skalarprodukte identisch.
    Weiterhin benutzen wir (\ref{eigenwertproblem}), dass die Floquet-Moden Eigenfunktionen von $\cal{H}(t)$ sind:
    \begin{align}
      \begin{split}
      \bar{H}_n  = \braket{\braket{\Phi_n(x,t)|\cal{H}(t)+\text{i}\hbar\frac{\partial}{\partial t}|\Phi_n(x,t)}}
    %  &=\epsilon_n \braket{\braket{\Phi_n(x,t)|\Phi_n(x,t)}} + \braket{\braket{\Phi_n(x,t)|\text{i}\hbar\frac{\partial}{\partial t}|\Phi_n(x,t)}} \\
      = \epsilon_n + \braket{\braket{\Phi_n(x,t)|\text{i}\hbar\frac{\partial}{\partial t}|\Phi_n(x,t)}} \; .
    \end{split}
    \end{align}
    Nähere Betrachtung zeigt, dass
    \begin{equation}
      \text{i}\hbar\frac{\partial}{\partial t} = -\omega \frac{\partial \cal{H}(t)}{\partial \omega}
    \end{equation}
    gilt \cite{haengi}.
    Da für $\epsilon_n$ und $\cal H(t)$ mit (\ref{eigenwertproblem}) eine Eigenwertgleichung vorliegt, kann das Hellman-Feynman-Theorem angewendet werden \cite{hellmann}.
    Dieses gibt eine Verbindung zwischen den Ableitungen der Eigenwerte und der Ableitung des Hamilton-Operators an.
    In unserem Fall erhalten wir dadurch
    \begin{equation}
      \frac{\partial \epsilon_n}{\partial \omega} = \braket{\braket{\Phi_n(x,t)|\frac{\partial \cal H(t)}{\partial \omega}|\Phi_n(x,t)}} \; .
    \end{equation}
    Damit folgt \cite{haengi}:
    \begin{equation}
      \bar{H}_n = \epsilon_n - \omega\frac{\partial \epsilon_n}{\partial \omega} \; .
      \label{mittleres_H}
    \end{equation}

    %\newpage
\iffalse
    \chapter{Getriebener harmonischer Oszillator in der Quantenmechanik}
      Der Hamilton-Operator eines harmonischen Oszillators der Masse $m$, welcher mit einer beliebigen aber periodischen äußeren Kraft $S(t)=S(t+T)$ getrieben wird, hat die Form
      \begin{equation}
        H(t) = H(t+T) = \frac{p^2}{2m} + \frac{1}{2}m\omega_0^2x^2-S(t)x \; ,
        \label{H_einzelner}
      \end{equation}
      mit $\omega_0=\sqrt{k/m}$ und der Potentialkonstante $k$.
      Im Folgenden wird vorerst die Schrödinger-Gleichung für eine beliebige Treibkraft $S(t)$ gelöst.
      Weiterhin werden die Floquet-Moden $\Phi_n(x,t)$ identifiziert, genauso wie die Quasienergien $\epsilon_n$, welche für eine beliebige periodische Treibkraft und explizit für eine sinusförmige Treibkraft bestimmt werden.

      Danach werden wir die Ewartungswerte $\braket{x}_n,\braket{x^2}_n,\braket{p}_n,\braket{p^2}_n$ und damit die Unschärfe berechnen, indem wir die bekannten Erwartungswerte des ungetriebenen Oszillators benutzen.
      Ebenso werden wir den zeitabhängigen und gemittelten Erwartungswert der Energie $\braket{H}_n$ und $\bar H_n$ berechnen.
\fi


    \section{Allgemeine Lösung der Schrödinger-Gleichung}
      \label{lsg_einzelner}
      Das System mit Hamilton-Operator (\ref{H_einzelner}) kann exakt gelöst werden, indem die Schrödinger-Gleichung durch einen Variablenwechsel und zwei unitäre Transformationen auf die bekannte Form des ungetriebenen Oszillators reduziert wird \cite{haengi}.

      \textbf{1) Unitärer Variabletransformation}\\
      Für den neuen Ortsoperator bzw. die Ortsvariable wird eine zeitabhängige Verschiebung angesetzt:
      \begin{equation}
        x \rightarrow y=x-\zeta(t) \; .
      \end{equation}
      Wie zu erwarten verändert sich der Impuls(operator) durch die Translation im Ort nicht, da $\zeta(t)$ bei der Ortsableitung wegfällt.
      %wobei der Impuls(operator) unverändert bleibt.
      %Der Impulsoperator bleibt dadurch unverändert.

      Mit der neuen Zeitableitung der Wellenfunktion
      \begin{equation}
        \text{i}\hbar \frac{\partial}{\partial t} \Psi(y(t),t) = \text{i}\hbar \dot{\Psi} -\dot{\zeta}\frac{\partial}{\partial y}\Psi(y(t),t)
      \end{equation}
      wird die Schrödinger-Gleichung zu:
      \begin{equation}
        \text i \hbar \dot{\Psi}(y,t) = \left[\text i \hbar \dot{\zeta}\frac{\partial}{\partial y}-\frac{\hbar^2}{2m}\frac{\partial^2}{\partial y^2}+\frac{1}{2}m\omega_0^2(y+\zeta)^2-(y+\zeta)S(t)\right]\Psi(y,t) \; .
        \label{schroedinger_einzeln_getrieben}
      \end{equation}

      \textbf{2) Unitäre Tranformation für $\Psi(y,t)$}\\
      Im Weiteren wählen wir die unitäre Transformation
      \begin{equation}
        \Psi(y,t) = \exp\left(\frac{\text i}{\hbar}m\dot \zeta y\right)\Lambda(y,t) \; .
      \end{equation}
      Durch Einsetzen in beide Seiten der Schrödinger-Gleichung (\ref{schroedinger_einzeln_getrieben}) und Ausrechnen der Ableitungen erhalten wir
      \begin{align}
        \begin{split}
          &\exp\left(\frac{\text i}{\hbar}m\dot \zeta y\right)(\text i \hbar \dot \Lambda(y,t) - my \ddot{\zeta}\Lambda(y,t)) =\\
           &\exp\left(\frac{\text i}{\hbar}m\dot \zeta y\right) \left[\left(-m\dot \zeta^2 \Lambda(y,t) + \text i \hbar \dot \zeta \frac{\partial}{\partial y} \Lambda(y,t) \right) \right. \\
           &\left. + \left(\frac{1}{2}m\dot \zeta^2 \Lambda(y,t) - \text{i} \hbar \dot \zeta \frac{\partial}{\partial y} \Lambda(y,t) - \frac{\hbar}{2m}\frac{\partial^2}{\partial y^2} \Lambda(y,t)  \right)
          + \frac{1}{2}m\omega_0^2(y+\zeta)^2  + (y+\zeta)S(t)  \right] \; .
        \end{split}
      \end{align}
      \iffalse
      Durch Einsetzen in die Schrödinger-Gleichung und Ausrechnen der Ableitungen erhalten wir für die linke Seite von (\ref{schroedinger_einzeln_getrieben})
      \text e^{\frac{\text i}{\hbar}m\dot \zeta y}(\text i \hbar \dot \Lambda(y,t) - my \ddot{\zeta}\Lambda(y,t))
      \begin{equation}
      \end{equation}
      und für die rechte Seite
      \begin{align}
        \begin{split}
        \text e^{\frac{\text i}{\hbar}m\dot \zeta y} \left[\left(-m\dot \zeta^2 \Lambda(y,t) + \text i \hbar \dot \zeta \frac{\partial}{\partial y} \Lambda(y,t) \right) + \left(\frac{1}{2}m\dot \zeta^2 \Lambda(y,t) - \text{i} \hbar \dot \zeta \frac{\partial}{\partial y} \Lambda(y,t) - \frac{\hbar}{2m}\frac{\partial^2}{\partial y^2} \Lambda(y,t)  \right) \right. \\
        \left. + \left(\frac{1}{2}m\omega_0^2y^2\Lambda(y,t) + m\omega_0^2y\zeta\Lambda(y,t) + \frac{1}{2}m\omega_0^2 \zeta^2\Lambda(y,t)) \right) + \left(-yS(t)\Lambda(y,t) - \zeta S(t)\Lambda(y,t) \right) \right]
      \end{split}
      \end{align}
      \fi

      %\newpage

      Indem auf beiden Seiten durch die Exponentialfunktion geteilt wird, bekommen wir eine Differentialgleichung für $\Lambda(y,t)$.
      Außerdem können durch Umsortieren der Terme die Lagrange-Funktion $L(\zeta,\dot \zeta, t)$
      \begin{equation}
        L(\zeta,\dot \zeta, t) = \frac{1}{2}m\dot \zeta^2 - \frac{1}{2}m\omega_0^2\zeta^2 + S(t)\zeta \; ,
        \label{lagrange_zeta}
      \end{equation}
        sowie die Bewegungsgleichung des klassischen getriebenen harmonischen  Oszillators \cite{husimi}
        \begin{equation}
          m\ddot \zeta + m\omega_0^2\zeta - S(t) = 0
          \label{dgl_zeta}
        \end{equation}
      für die Verschiebung $\zeta(t)$ identifiziert werden.
      Die Differentialgleichung für $\Lambda(y,t)$ ist folglich
      \begin{equation}
        \text i \hbar \dot \Lambda(y,t) = \left[ \frac{-\hbar^2}{2m}\frac{\partial^2}{\partial y^2} + \frac{1}{2}m\omega_0^2y^2 + (m\ddot \zeta + m\omega_0^2y\zeta - S(t))y - L(\zeta,\dot \zeta, t) \right]\Lambda(y,t) \; .
        \label{dgl_lambda}
      \end{equation}
      Um die Gleichung zu vereinfachen, wählen wir $\zeta(t)$ nun so, dass es gerade die klassische Bewegungsgleichung erfüllt, der entsprechende Term in (\ref{dgl_lambda}) also verschwindet.
      Nur noch die Lagrange-Funktion unterscheidet diese Differential-Gleichung von der des ungetriebenen Oszillators.

      \textbf{3) Unitäre Transformation für $\Lambda(y,t)$}\\
      Zuletzt wählen wir den Ansatz
      \begin{equation}
        \Lambda(y,t) = \exp\left(\frac{\text i}{\hbar}\int_0^tL \: \text d t'\right)\chi(y,t)
      \end{equation}
      für $\Lambda(y,t)$, um die Lagrange-Funktion in (\ref{dgl_lambda}) zu eliminieren.
      Dadurch wird die Differential-Gleichung für $\Lambda(y,t)$ bzw. die ursprüngliche Schrödinger-Gleichung auf einen ungetriebenen Oszillator für $\chi(y,t)$ zurückgeführt:
      \begin{equation}
        \text i \hbar \dot \chi(y,t) = \left[ \frac{-\hbar^2}{2m}\frac{\partial^2}{\partial y^2} + \frac{1}{2}m\omega_0^2y^2 \right]\chi(y,t) \; .
      \end{equation}
      Das bedeutet die $\chi_n(y,t)$ sind die Wellenfunktionen des ungetriebenen Oszillators und die Gesamtlösung der Schrödinger-Gleichung des getriebenen Oszillators ist damit gegeben durch
      \begin{align}
        \begin{split}
        \Psi_n(x,t) &= \Psi_n(y=x-\zeta(t),t) \\
        &= N_nO_n\left(\sqrt{\frac{m\omega_0}{\hbar}}(x-\zeta(t))\right) \exp\left[\frac{-m\omega_0}{2\hbar}(x-\zeta(t))^2\right] \\
        &\quad \cdot \exp\left[\frac{\text i}{\hbar}\left(m\dot \zeta(t)(x-\zeta(t))-E_nt+\int_0^tL(\dot \zeta,\zeta,t')\:\text dt'\right)\right] \; ,
        \; n \in \mathbb{N}_0 \; .
        \label{gesamtlsg_einzelner}
      \end{split}
      \end{align}
      Dabei sind $O_n$ die Hermit-Polynome, $E_n = \hbar \omega_0(n+1/2)$ die bekannten Eigenenergien der Zustände des ungetriebenen Oszillators und
      \begin{equation}
        N_n = \left(\frac{m\omega_0}{\pi \hbar}\right) \frac{1}{\sqrt{2^nn!}}
      \end{equation}
      dessen Normierungsfaktoren.
      Die Lösungen des getriebenen Oszillators sind auf $x \in (-\infty, \infty)$ normiert und somit ist ebenfalls Gleichung (\ref{skalarprodukt_einzelner}) mit dem Skalarprodukt erfüllt.

      Die Lösung entspricht damit einem, um die klassische Lösung $\zeta(t)$ verschobenen, ungetriebenen Oszillator, mit einer zusätzlichen zeit- und ortsabhängigen komplexen Phase.
      Die treibende Kraft $S(t)$ geht in die klassische Lösung $\zeta(t)$ und, über das Wirkungsintegral, in die komplexe Phase ein.



    %war vorher subsection
      \section{\texorpdfstring{Identifizierung der Quasienergien $\epsilon_n$ und der Floquet-Moden $\Phi_n(x,t)$}{Identifizierung der Quasienergien epsilon_n und der Floquet-Moden Phi_n(x,t)}}
        Nach dem Floquet-Theorem für periodische Hamilton-Operatoren (\ref{floquet_theorem}) kann die Lösung des getriebenen Oszillators geschrieben werden als
        \begin{equation}
          \Psi_n(x,t) = \exp\left(\frac{-\text i}{\hbar}\epsilon_nt\right)\Phi_n(x,t) \; ,
        \end{equation}
        mit $\Phi_n(x,t)=\Phi_n(x,t+T)$.
        Nun können wir alle $T$-periodischen Terme als $\Phi_n(x,t)$ identifizieren, und alle Terme im Exponenten, die linear in $t$ sind, als $-\text i\epsilon_n t/\hbar$ \cite{haengi}.

        Alle Funktionen von $(x-\zeta(t))$ haben die Periode $T$, da $\zeta(T)$ als Lösung der klassischen Bewegungsgleichung mit $S(t)=S(t+T)$ die Periode $T$ \,hat. Das
        Ergebnis des Integrals über die Lagrange-Funktion kann nur $T$-periodisch oder linear in $t$ sein, deshalb sind die Quasienergien gegeben durch die $E_n$ und den linearen Teil des Integrals \cite{haengi}:
        \begin{equation}
          \epsilon_n = E_n - \frac{1}{T} \int_0^T L(\dot \zeta, \zeta, t) \: \text d t \; .
          \label{ep}
        \end{equation}
        Die Floquet-Moden sind demnach
        \begin{align}
          \begin{split}
            \Phi_n(x,t) &=
             N_nO_n\left(\sqrt{\frac{m\omega_0}{\hbar}}(x-\zeta(t))\right) \exp\left[\frac{-m\omega_0}{2\hbar}(x-\zeta(t))^2\right] \\
            &\quad \cdot \exp\left[\frac{\text i}{\hbar}\left(m\dot \zeta(t)(x-\zeta(t))+\int_0^tL(\dot \zeta,\zeta,t')\:\text d t-\frac{t}{T} \int_0^T L(\dot \zeta,\zeta,t)\:\text dt\: \right)\right] \; ,
            \;n \in \mathbb{N}_0 \; .
            \label{floquet_moden_einzelner}
          \end{split}
        \end{align}


      \subsection{Quasienergien für eine sinusoidiale Treibkraft}
      \label{epsilon_sinuskraft}
        Hier wird ein Beispiel einer treibenden Kraft diskutiert: %\cite{haengi}:
        \begin{equation}
          S(t) = S(t+T) = A\sin(\omega t),
        \end{equation}
        wobei $T=2\pi / \omega$ ist.
        Setzen wir die allgemeine homogene Lösung gleich null, wird die Lösung $\zeta(t)$ der klassischen Bewegungsgleichung (\ref{dgl_zeta}) zu \cite{mads}
        \begin{equation}
          \zeta(t) = \frac{A\sin(\omega t)}{m(\omega_0^2 - \omega^2)} \;
          \label{zeta_sinuskraft}.
        \end{equation}
        Das Berechnen des Wirkungsintegrals und anschließendes Identifizieren des linearen Anteils liefert die Quasienergien für die gegebene Kraft \cite{mads}:
        \begin{equation}
          \epsilon_n  = \hbar \omega_0\left(n+\frac{1}{2}\right) - \frac{A}{4m(\omega_0^2-\omega^2)} \;.
        \end{equation}
        Wie zu erkennen streben die Quasienergien, und somit die mittlere Energie des Systems (\ref{mittleres_H}), gegen Unendlich, wenn sich die Treibfrequenz $\omega$ nahe der Eigenfrequenz des Oszillators $\omega_0 $ befindet.
        Da wir einen getriebenen Oszillator ohne Dämpfung betrachten, war dieses Ergebnis zu erwarten.
        %\cite{mads}.
        %HIER KANN MAN NOCH AUSFUEHRLICH INSCHREIBEN MIT MADS

    %\chapter{Ergebnisse}

    %war vorher section
    \subsection{Quasienergien für eine beliebige periodische Treibkraft $S(t)$}
      \label{epsilon_bel_kraft}
      Um die Quasienergien bei einer beliebigen periodisch treibenden Kraft auf elegante Weise zu bestimmen, und ohne ein Wirkungsintegral berechnen zu müssen, setzen wir eine komplexe Fourier-Reihe an:
      \begin{equation}
        S(t) = \sum_{j=-\infty}^\infty c_j \text e^{\text ij\omega t}, \; c_j = \frac{1}{T} \int_0^T S(t) \text e^{-ij\omega t} \: \text d t \; .
      \end{equation}
      Die $c_j$ haben die Einheit einer Kraft.
\iffalse
      Für $\zeta(t)$ wählen wir jetzt ebenfalls einen Reihenansatz:
      \begin{equation}
        \zeta(t) = \sum_{j=-\infty}^\infty d_j \text e^{\text ij\omega t} \; .
      \end{equation}
\fi
      Für $\zeta(t)$ wählen wir jetzt den gleichen Reihenansatz mit anderen Koeffizienten $d_j$.
      Wir betrachten wieder nur die inhomogene klassische Bewegungsgleichung und bestimmen die $d_j$, indem wir einsetzen und einen Koeffizientenvergleich der $d_j$ und $c_j$ machen.
      Es zeigt sich, dass
      \begin{equation}
        \zeta(t) = \sum_{j=-\infty}^\infty \frac{c_j}{m(\omega_0^2-j^2\omega^2)} \text e^{ij\omega t}
        \label{zeta_bel_kraft}
      \end{equation}
      gilt, womit sich die Lagrange-Funktion (\ref{lagrange_zeta}) ergibt zu:
      \begin{align}
        \begin{split}
          L &= \frac{1}{2}m\dot \zeta^2 - \frac{1}{2}m\omega_0^2\zeta^2 + S(t)\zeta \notag\\
           &=\sum_j \sum_l \left[ \frac{-\omega^2}{2m} \frac{jc_j}{\omega_0^2-j^2\omega^2} \frac{lc_l}{\omega_0^2-l^2\omega^2}
           -\frac{\omega_0^2}{2m} \frac{c_j}{\omega_0^2-j^2\omega^2} \frac{c_l}{\omega_0^2-l^2\omega^2} \right. \notag\\
            &\left. \quad + \frac{1}{m} \frac{c_jc_l}{\omega_0^2-j^2\omega^2} \right] \text e^{\text i(j+l)\omega t}  \; , \; j,l \in \mathbb{Z} \; .
         \end{split}
       \end{align}
      \iffalse
      \begin{align}
        \begin{split}
          L &= \frac{1}{2}m\dot \zeta^2 - \frac{1}{2}m\omega_0^2\zeta^2 + S(t)\zeta \\
           &=\frac{-\omega^2}{2m} \sum_j \sum_l \frac{jc_j}{\omega_0^2-j^2\omega^2} \frac{lc_l}{\omega_0^2-l^2\omega^2} \text e^{\text i(j+l)\omega t}\\
           &\quad-\frac{\omega_0^2}{2m} \sum_j \sum_l \frac{c_j}{\omega_0^2-j^2\omega^2} \frac{c_l}{\omega_0^2-l^2\omega^2} \text e^{\text i(j+l)\omega t}\\
           &\quad + \frac{1}{m} \sum_j \sum_l \frac{c_jc_l}{\omega_0^2-j^2\omega^2} \text e^{\text i(j+l)\omega t}\; , \; j,l \in \mathbb{Z} \; .
         \end{split}
       \end{align}
       \fi

       Nun identifizieren wir alle Terme der Lagrange-Funktion, welche nach der Ausführung des Wirkungsintegrals linear in der Zeit $t$ sind, ohne dieses explizit zu berechnen.
       Denn wir wissen, dass die Exponentialterme, welche sich beim Integrieren nach der Zeit reproduzieren, periodisch in der Zeit sind.
       Daher werden nur die konstanten Terme, welche entstehen, wenn die Exponentialterme wegfallen, eine lineare Abhängigkeit aufweisen.
       Die Exponentialterme werden 1, wenn $j=-l$ ist.
       Infolgedessen wird die Doppelsumme, die bei der Quadrierung entstanden ist, wieder zur Einzelsumme.
       Wir fassen den linearen Teil des Wirkungsintegrals zusammen:
       \begin{align}
         \begin{split}
           \frac{1}{T} \int_0^T L \: \text d t
           &= \sum_j \left[\frac{\omega^2}{2m} \frac{j^2c_jc_{-j}}{(\omega_0^2-j^2\omega^2)^2}
           - \frac{\omega_o^2}{2m} \frac{c_jc_{-j}}{(\omega_0^2-j^2\omega^2)^2} \right.
           \left. +\frac{1}{m} \frac{c_jc_{-j}}{\omega_0^2-j^2\omega^2}
           \right]\\
           %&= \sum_j \left[\frac{1}{2m} \frac{c_jc_{-j}(j^2\omega^2-\omega_0^2)}{(\omega_0^2-j^2\omega^2)^2} + \frac{1}{m} \frac{c_jc_{-j}}{\omega_0^2-j^2\omega^2} \right] \\
           &= \sum_j \frac{c_jc_{-j}}{2m(\omega_0^2-j^2\omega^2)} = \frac{c_0^2}{2m\omega_0^2} + \sum_{j} \frac{c_j^2}{m(\omega_0^2-j^2\omega^2)}
         \end{split}
       \end{align}

       Die Quasienergien $\epsilon_n$ sind hiernach:
       \begin{align}
         \begin{split}
           \epsilon_n &= \hbar \omega_0\left(n+\frac{1}{2}\right) - \sum_j \frac{c_jc_{-j}}{2m(\omega_0^2-j^2\omega^2)} \; .
         \end{split}
       \end{align}
       Interessanterweise kommt es in obiger Formel nicht nur bei $\omega = \omega_0$ zu einer Singularität, wie bei der Beispielkraft $S(t) = A\sin(\omega t)$, sondern bei allen $\omega = \omega_0 / j$.
       %WORAN LIEGT DAS, BEI KOEFFIZIENTEENVEGL ZU BESTIMMUNG VON ZETA GUCKEN.



       %\subsection{Beispiele für S(t) dreieck rechteck delta}
