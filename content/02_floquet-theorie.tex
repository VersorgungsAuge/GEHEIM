\chapter{Floquet-Theorie}


%\section{Floquet-Theorie}
  Die Floquet-Theorie ist ein nützliches Werkzeug zur der Lösung von quantenmechanischen Systemen, welche durch einen zeitlich periodischen Hamilton-Operator
  \begin{equation}
    H(t) = H(t+T) \; ,
  \end{equation}
  mit der Periode $T$, beschrieben werden.

  Das Floquet-Theorem besagt, dass bei einem solchen System, die Lösungen $\Psi_n(x,t)$ der Schrödinger-Gleichung
  \begin{equation}
    \text{i}\hbar\frac{\partial}{\partial t}\Psi_n(x,t) = H(t)\Psi_n(x,t)
    \label{schroedinger}
  \end{equation}
  in Ortsdarstellung die Form
  \begin{equation}
    \Psi_n(x,t) = \text{e}^{-\frac{i}{\hbar}\epsilon_nt}\Phi_n(x,t)
    \label{floquet_theorem}
  \end{equation}
  haben.
  Hierbei sind $\Phi_n(x,t) = \Phi_n(x,t+T)$ $T$-periodische Funktionen, die sogenannten Floquet-Moden, und $\epsilon_n$ die zugehörigen reellen Quasienergien, wobei diese Bezeichnungen gewählt wurden aufgrund der Parallele zu den Bloch-Moden und Quasiimpulsen des Bloch-Theorems \cite{haengi}.
  Das Floquet-Theorem kann damit als "Bloch-Theorem in der Zeit" aufgefasst werden \cite{sherly}.

  Durch Einsetzen dieses Ansatzes für die Wellenfunktionen (\ref{floquet_theorem}) in die Schrödingergleichung (\ref{schroedinger}) erhalten wir
  \begin{equation}
    \epsilon_n \Phi_n(x,t) = \left(H(t)-\text{i}\hbar\frac{\partial}{\partial t}\right)\Phi_n(x,t) = \cal{H}(t) \Phi_n(x,t) \; .
    \label{eigenwertproblem}
  \end{equation}
  Die Lösung der Schrödinger Gleichung konnte somit auf die Lösung eines Eigenwertproblems für den neuen Operator $\cal{H}(t)$ zurückgeführt werden \cite{sherly}.

  Die hermitischen Operatoren $H(t)$ und $\cal H(t)$ operieren auf dem Hilbertraum $\cal{L}^2 \otimes \cal{T}$.
  Dabei ist $\cal{L}^2$ der Raum der quadratintegrablen Funktionen und $\cal{T}$ der Raum der auf $[0,T]$ integrablen Funktionen, da die Operatoren $T$-periodisch sind \cite{haenggi}.
  Nach dem Spektralsatz bilden die Eigenfunktionen $\Phi_n(x,t)$ von $\cal{H}(t)$ eine Orthogonalbasis von $\cal{L}^2 \otimes \cal{T}$, welche auf eine Orthonormalbasis normiert werden kann, wodurch wir das Skalarprodukt definieren k"onnen als:
  %Es ergibt sich
  \begin{align}
    \begin{split}
    \braket{\braket{\Phi_n(x,t)|\Phi_m(x,t)}} &= \frac{1}{T} \int_0^T \braket{\Phi_n(x,t)|\Phi_n(x,t)} \text{d}t \\
    &= \frac{1}{T} \int_0^T \int_{-\infty}^{\infty} \Phi_n^*(x,t)\Phi_m(x,t) \: \text{d}x\text{d}t = \delta_{n,m} \; .
    \label{skalarprodukt_einzelner}
    \end{split}
  \end{align}


%war vorher subsection
 \section{\texorpdfstring{Zeitlich gemittelter Erwartungswert der Energie $\bar{H}_n$}{Zeitlich gemittelter Erwartungswert der Energie bar{H}_n}}

    Da $H(t)$ nicht zeitlich konstant ist, sind auch dessen Erwartungswerte, die Energien des Systems, zeitabhängig.
    Ein direkter Vorteil der Floquet-Theorie liegt darin, dass sich die durchschnittliche Energie
    \begin{equation}
      \bar{H}_n  = \braket{\braket{\Psi_n(x,t)|H(t)|\Psi_m(x,t)}}
      %\label{mittleres_H}
    \end{equation}
    des $n$-ten Zustandes $\Psi_n(x,t)$ leicht über die Quasienergien $\epsilon_n$ berechnen lässt, ohne explizit Integrale zu lösen.

    Dazu ersetzen wir $H(t)$ mit Hilfe von (\ref{eigenwertproblem}).
    Außerdem unterscheiden sich die Floquet-Moden $\Phi_n(x,t)$ und die Wellenfunktionen $\Psi_n(x,t)$ nur durch eine komplexe Phase, daher sind deren Skalarprodukte identisch.
    Weiterhin benutzen wir (\ref{eigenwertproblem}), dass die Floquet-Moden Eigenfunktionen von $\cal{H}(t)$ sind:
    \begin{align}
      \begin{split}
      \bar{H}_n  &= \braket{\braket{\Phi_n(x,t)|\cal{H}(t)+\text{i}\hbar\frac{\partial}{\partial t}|\Phi_n(x,t)}} \\
      &=\epsilon_n \braket{\braket{\Phi_n(x,t)|\Phi_n(x,t)}} + \braket{\braket{\Phi_n(x,t)|\text{i}\hbar\frac{\partial}{\partial t}|\Phi_n(x,t)}} \\
      &= \epsilon_n + \braket{\braket{\Phi_n(x,t)|\text{i}\hbar\frac{\partial}{\partial t}|\Phi_n(x,t)}} \; .
    \end{split}
    \end{align}

    Nähere Betrachtung zeigt, dass
    \begin{equation}
      \text{i}\hbar\frac{\partial}{\partial t} = -\omega \frac{\partial \cal{H}(t)}{\partial \omega}
    \end{equation}
    gilt \cite{haenggi}.
    Da für $\epsilon_n$ und $\cal H(t)$ mit (\ref{eigenwertproblem}) eine Eigenwertgleichung vorliegt, kann das Hellman-Feynman-Theorem angewendet werden \cite{hellmann online quelle}.
    Dieses gibt eine Verbindung zwischen den Ableitungen der Eigenwerte und der Ableitung des Hamilton-Operators an.
    In unserem Fall erhalten wir dadurch
    \begin{equation}
      \frac{\partial \epsilon_n}{\partial \omega} = \braket{\braket{\Phi_n(x,t)|\frac{\partial \cal H(t)}{\partial \omega}|\Phi_n(x,t)}} \; .
    \end{equation}
    Damit folgt \cite{haenggi}:
    \begin{equation}
      \bar{H}_n = \epsilon_n - \omega\frac{\partial \epsilon_n}{\partial \omega} \; .
      \label{mittleres_H}
    \end{equation}

    %\newpage






























  b
