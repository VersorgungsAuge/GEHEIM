\chapter{Zwei gekoppelte getriebene harmonische Oszillatoren}
\label{4}

In diesem Teil der Arbeit werden mit Hilfe der aus (\ref{lsg_einzelner}) bekannten Lösung des einzelnen getrieben Oszillators die Wellenfunktionen für ein System hergeleitet, das aus zwei gekoppelten Oszillatoren $x_1$ und $x_2$ der gleichen Masse $m$ besteht, von denen einer mit der periodischen Kraft $S(t) = S(t+T)$ angetrieben wird.
Die Potentialkonstanten $k$ der beiden Oszillatoren sind ebenfalls identisch, aber im Allgemeinen unterschiedlich zur Kopplungskonstante $\kappa$ zwischen den Oszillatoren.
Der Hamilton-Operator dieses Systems kann direkt aus der Hamilton-Funktion  der klassischen Mechanik übernommen werden:
\begin{equation}
  H(t) = H(t+T) = \frac{p_1^2}{2m} + \frac{p_2^2}{2m} + \frac 1 2 kx_1^2 + \frac 1 2 kx_2^2 + \frac 1 2 \kappa(x_2-x_1)^2 - S(t)x_1 \; .
  \label{H_gekoppelt}
\end{equation}
Es werden auch die Erwartungswerte für  den Ort $\braket{x_{1,2}}_{n,l}$ und den Impuls $\braket{p_{1,2}}_{n,l}$, genauso wie der Erwartungswert der Energie $\braket{H}_{n,l}$ und dessen zeitliches Mittel $\overline{H}_{n,l}$ berechnet, indem auf die bekannten Erwartungswerte des einzelnen getriebenen Oszillators zurückgeführt wird.
Die Erwartungswerte werden zudem graphisch dargestellt.

Zur Lösung des Systems wird eine unitäre Koordinatentransformation eingeführt, welche den Hamilton-Operator $H(x_1,x_2,p_1,p_2,t)$ zu zwei in den neuen Koordinaten unabhängigen Hamilton-Operatoren $H_+(x_+,p_+,t)$ und $H_-(x_-,p_-,t)$ mit effektiven Potentialkonstanten $k_+,k_-$ entkoppelt, welche je einen einzelnen (getriebenen) Oszillator beschreiben.
Dann ergeben sich die Wellenfunktionen aus denen des einzelnen Oszillators.




%war vorher subsection
\section{Die Schrödinger-Gleichung mit unabhängigen Hamilton-Operatoren}
\label{kap:schroedglg_unabh_H}
  Liegt ein Hamilton-Operator der Form
  \begin{equation}
    H = \sum_i H_i(x_i,p_i,t) \;,\; H_i : \cal H_i \rightarrow H_i
    \label{unabh_H}
  \end{equation}
  vor, führt der Ansatz
  \begin{equation}
    \Psi = \prod_i \Psi(x_i,p_i,t) \; , \; \Psi_i \in \cal H_i
    \label{lsg_unabh_H}
  \end{equation}
  auf unabhängige Schrödinger-Gleichungen für die einzelnen Wellenfunktionen \\ $\Psi_i(x_i,p_i,t)$, sodass %\cite{online quelle}
  \begin{equation}
    \text i \hbar \frac{\partial}{\partial t}\Psi_i = \delta_{i,j}  H_j \Psi_i
    \label{schroedglg_unabh_H}
  \end{equation}
  erfüllt ist.
  Um dies zu zeigen, schauen wir uns den Fall von zwei unabhängigen Operatoren an:
  \begin{align}
    \begin{split}
      \text i \hbar \frac{\partial}{\partial t} \Psi = H\Psi \iff \Psi_2 H_1 \Psi_1 + \Psi_1 H_2 \Psi_2 = \text i \hbar \left(\Psi_2\frac{\partial}{\partial t} \Psi_1 + \Psi_1\frac{\partial}{\partial t} \Psi_2 \right) \; .
    \end{split}
  \end{align}
  Da die Operatoren nur auf Funktionen wirken, die auf dem selben Raum definiert sind, können sie mit der jeweils anderen Funktion vertauscht werden.
  Wenn wir weiterhin durch unseren Ansatz $\Psi_1\Psi_2$ dividieren, wird die Gleichung zu:
  \begin{equation}
    \frac{1}{\Psi_1}H_1\Psi_1 + \frac{1}{\Psi_2}H_2\Psi_2 = \text i \hbar \left(\frac{\frac{\partial}{\partial t} \Psi_1}{\Psi_1} + \frac{\frac{\partial}{\partial t} \Psi_2}{\Psi_2} \right) \; .
  \end{equation}
  Die linke und rechte Seite müssen für alle unabhängigen $x_1,p_1,x_2,p_2$ gleich sein, dadurch folgen die einzelnen Schrödinger-Gleichungen für die Wellenfunktionen $\Psi_1$ und $\Psi_2$:
  \begin{equation}
    \frac{1}{\Psi_1}H_1\Psi_1 = \text i\hbar\frac{\frac{\partial}{\partial t} \Psi_1}{\Psi_1} \; , \; \frac{1}{\Psi_2}H_2\Psi_2 =\text i\hbar \frac{\frac{\partial}{\partial t} \Psi_2}{\Psi_2} \; .
  \end{equation}
  Wie an diesem Beispiel zu sehen, gilt Gleiches auch für beliebig viele unabhängige Hamilton-Operatoren, womit (\ref{schroedglg_unabh_H}) erfüllt ist und wir wissen, dass die Lösung $\Psi$ die Form (\ref{lsg_unabh_H}) hat.
  %nur EINE moegliche lsg hat die form psi, wenn wir einschraenken, dass psi so aussehen soll, wer weis ob es noch andere lsgen nicht dieser form gibt.



\section{Unitäre Variablentransformation und allgemeine Lösung der Schrödinger-Gleichung}
  Um für unseren Hamilton-Operator $H(x_1,x_2,p_1,p_2,t)$ (\ref{H_gekoppelt}) eine entkoppelte Form \\
  $H(x_+,x_-,p_+,p_-,t)=H_+(x_+,p_+,t)+H_-(x_-,p_-,t)$ nach (\ref{unabh_H}) in den neuen Koordinaten/Variablen $x_+,x_-,p_+,p_-$ zu erhalten, wählen wir die unitären Koordinatentransformationen~\cite{arxiv}
  \begin{equation}
    x_+ = \frac{1}{\sqrt{2}}(x_2+x_1) \;,\; x_-=\frac{1}{\sqrt{2}}(x_2-x_1) \;,
    \label{koord_trafo_x}
  \end{equation}
  welche durch die Eigenmoden/Eigenvektoren des klassischen Problems motiviert sind.
  Dessen Normierung stellt sicher, dass die Transformation unitär ist.
  %worauf in Kapitel () genauer eingegangen wird.
  Durch einfaches Umstellen folgen $x_1$ und $x_2$ in Abhängigkeit der neuen Koordinaten $x_+$ und $x_-$:
  \begin{equation}
    x_1=\frac{1}{\sqrt{2}}(x_+-x_-) \;,\; x_2=\frac{1}{\sqrt{2}}(x_++x_-) \; .
  \end{equation}
  Indem wir $x_1$ und $x_2$ im Hamilton-Operator (\ref{H_gekoppelt}) ersetzen, formen wir den ortsabhängigen Teil um zu
  \begin{align}
    &\frac{1}{2}kx_1^2+\frac{1}{2}kx_2^2+\frac{1}{2}\kappa(x_2-x_1)^2-S(t)x_1 \notag\\
    &=\frac{1}{2}kx_+^2+\frac{1}{2}(k+2\kappa)x_-^2-S(t)\frac{1}{\sqrt{2}}(x_+-x_-) \; .
  \end{align}
  Es tauchen nun keine Kopplungsterme $\propto x_+x_-$ mehr auf, wie es in den alten Koordinaten der Fall war.

  Jetzt betrachten wir die neuen Impulse und überprüfen, dass es durch den Variablenwechsel nicht zu neuen Kopplungstermen $\propto p_+$$p_-$ kommt.
  Mit der Kettenregel folgt für die Ableitungen der Impulsoperatoren
  \begin{align}
    \frac{\partial}{\partial x_{\pm}} = \frac{\partial}{\partial x_1}\frac{\partial x_1}{\partial x_{\pm}} + \frac{\partial}{\partial x_2}\frac{\partial x_2}{\partial x_{\pm}}
    =\frac{\partial}{\partial x_1}\left(\pm\frac{1}{\sqrt{2}}\right)
    + \frac{\partial}{\partial x_2}\frac{1}{\sqrt{2}}
    = \frac{1}{\sqrt{2}}\left(\frac{\partial}{\partial x_2}\pm\frac{\partial}{\partial  x_1}\right) \;,
  \end{align}
  weshalb für die Impulsoperatoren, analog zu den Ortsoperatoren, gilt
  \begin{equation}
    p_+ = \frac{1}{\sqrt{2}}(p_2+p_1) \;,\; p_-=\frac{1}{\sqrt{2}}(p_2-p_1) \; .
    \label{koord_trafo_p}
  \end{equation}
  Wir setzen erneut in unseren Hamilton-Operator (\ref{H_gekoppelt}) ein und erhalten für den impulsabhängigen Teil:
  \begin{equation}
    \frac{p_1^2}{2m} + \frac{p_2^2}{2m} = \frac{p_+^2}{2m} + \frac{p_-^2}{2m} \; .
    \label{koord_trafo_p^2}
  \end{equation}

  %Wie erwartet bleibt die Summe der quadrierten Impulsoperatoren unverändert.

  Der gesamte Hamilton-Operator der zwei gekoppelten getrieben Oszillatoren ist in den neuen Variablen folglich
  \begin{align}
    H(t) &= H(t+T) = \frac{p_1^2}{2m} + \frac{p_2^2}{2m} + \frac 1 2 kx_1^2 + \frac 1 2 kx_2^2 + \frac 1 2 \kappa(x_2-x_1)^2 - S(t)x_1 \notag\\
    &= \frac{p_+^2}{2m}+\frac{1}{2}kx_+^2-\frac{1}{\sqrt{2}}S(t)x_+  +
    \frac{p_-^2}{2m}+\frac{1}{2}(k+2\kappa)x_-^2+\frac{1}{\sqrt{2}}S(t)x_- \notag\\
    &= \frac{p_+^2}{2m}+\frac{1}{2}k_+x_+^2-S_+(t)x_+  +
    \frac{p_-^2}{2m}+\frac{1}{2}k_-x_-^2-S_-(t)x_- \notag\\
    &= H_+(x_+,p_+,t) + H_-(x_-,p_-,t) \;,\; H_\pm:\cal H_\pm \rightarrow \cal H_\pm \; .
    \label{gekoppelte_H_entkoppelt}
  \end{align}
  Der Hamilton-Operator in den neuen Variablen $x_{\pm}/p_{\pm}$ beschreibt demnach ein System aus zwei unabhängigen getriebenen harmonischen Oszillatoren mit neuen Potentialkonstanten $k_1=k$ und $k_2=k+2\kappa$ bzw. neuen Eigenfrequenzen
  \begin{equation}
    w_+=\sqrt{\frac{k}{m}} \quad\text{und}\quad \omega_-=\sqrt{\frac{k+2\kappa}{m}} \; .
    \label{neue_eigenfrequenzen}
  \end{equation}
  Die beiden Oszillatoren werden, wegen dem verschiedenen Vorzeichen von $S_+(t)=S_+(t+T)$ und $S_-(t)=S_-(t+T)$, periodisch aber phasenversetzt um $\pi$ mit der ursprünglichen Treibkraft $S(t)$ angetrieben, wobei diese mit dem Faktor $1/\sqrt{2}$ skaliert wird.

Betrachten wir den Fall, dass beide Oszillatoren $x_1$ und $x_2$ des Systems in den alten Koordinaten getrieben sind, und zwar genau wie durch die neuen Koordinaten $x_+$ und $x_-$ vorgegeben, das heißt
  \begin{align}
      &H(t) = \frac{p_1^2}{2m} + \frac{p_2^2}{2m} + \frac 1 2 kx_1^2 + \frac 1 2 kx_2^2 + \frac 1 2 \kappa(x_2-x_1)^2 - S(t)(x_2+x_1) \notag\\
      \text{oder} \; &H(t) = \frac{p_1^2}{2m} + \frac{p_2^2}{2m} + \frac 1 2 kx_1^2 + \frac 1 2 kx_2^2 + \frac 1 2 \kappa(x_2-x_1)^2 - S(t)(x_2-x_1) \;,
  \end{align}
  liegt in den neuen Variablen ein vereinfachtes System vor, bei dem nur ein Oszillator $x_+$ oder $x_-$ getrieben ist.
  Es muss also in den klassischen Normalmoden getrieben werden, damit nach dem Variablenwechsel ein System mit nur noch einem getriebenen Oszillator vorliegt.

  \textbf{Lösung der Schrödinger-Gleichung}

  Mit den Ergebnissen dieses und des vorherigen Abschnittes können wir nun die Wellenfunktion $\Psi_{n,l}$ des Systems zweier gekoppelter getriebener Oszillatoren (\ref{H_gekoppelt}), für eine allgemeine Treibkraft $S(t)=S(t+T)$, in den neuen, entkoppelten Koordinaten $x_+$ und $x_-$, nach Formel (\ref{schroedglg_unabh_H}) schreiben.
  Durch Ersetzen der neuen Variablen nach Formel (\ref{koord_trafo_x}) und (\ref{koord_trafo_p^2}) ist auch die Wellenfunktion in den ursprünglichen Variablen bekannt:
  \begin{align}
     &\Psi_{n,l}(x_+,x_-,t) = \Psi_{+,n}(x_+,t)\Psi_{-,l}(x_-,t) \notag\\
     &=N_{n,+}H_n\left(\sqrt{\frac{m\omega_+}{\hbar}}(x_+-\zeta_+(t))\right) \exp\left(\frac{-m\omega_+}{2\hbar}(x_+-\zeta_+(t))^2\right)\notag\\
     &\quad \cdot \exp\left[\frac{\text i}{\hbar}\left(m\dot \zeta(t)(x_+-\zeta_+(t))-E_{+,n}t+\int_0^tL_+\:\text dt'\right)\right] \quad\quad\quad\notag\\
     &\quad \cdot N_{l,-}H_l\left(\sqrt{\frac{m\omega_-}{\hbar}}(x_--\zeta_-(t))\right) \exp\left(\frac{-m\omega_-}{2\hbar}(x_--\zeta_-(t))^2\right)\notag\\
     &\quad \cdot \exp\left[\frac{\text i}{\hbar}\left(m\dot \zeta(t)(x_--\zeta_-(t))-E_{-,l}t+\int_0^tL_-\:\text dt'\right)\right] \;,\; \Psi_\pm \in \cal H_\pm \notag\\
    &= N_{+,n}N_{-,l} H_n\left(\sqrt{\frac{m\omega_+}{\hbar}}\left(\frac{1}{\sqrt{2}}(x_2+x_1)-\zeta_+(t)\right)\right)H_l\left(\sqrt{\frac{m\omega_-}{\hbar}}\left(\frac{1}{\sqrt{2}}(x_2-x_1)-\zeta_-(t)\right)\right) \notag\\
    &\quad\cdot \exp\left[\frac{-m}{2\hbar}\left(\omega_+\left(\frac{1}{\sqrt{2}}(x_2+x_1)-\zeta_+(t)\right)^2 + \omega_-\left(\frac{1}{\sqrt{2}}(x_2-x_1)-\zeta_-(t)\right)^2\right)\right] \notag\\
    &\quad\cdot \exp\left[\frac{\text i}{\hbar} \left(m\zeta_+(t)\left(\frac{1}{\sqrt{2}}(x_2+x_1)-\zeta_+(t)\right) + m\zeta_-(t)\left(\frac{1}{\sqrt{2}}(x_2-x_1)-\zeta_-(t)\right)\right) \right] \notag\\
    &\quad\cdot \exp \left[\frac{\text i}{\hbar}\left(-(E_{+,n}+E_{-,l})+\int_0^tL_++L_-\:\text dt'\right)\right] \notag\\
    &=\Psi_{n,l}(x_1,x_2,t) \; , \; n,l \in \mathbb N_0 \; .
  \end{align}
  %HIER GUCKEN, OB SO AUFSCHREIBEN OK (GEHT UEBER RAND) UND EVTL AUCH IN X1 X2 AUSGESCHRIEBEN (WAHRSCHEINLICH SCHON, DANN AUCH MIT (t) UEBERALL)
  Die klassischen Lösungen $\zeta_\pm(t)$ und die Lagrange-Funktionen $L_\pm$ sind wie bei der Lösung des einzelnen Oszillators bestimmt über (\ref{dgl_zeta}) und (\ref{lagrange_zeta}).
  Genauso sind die Normierungskonstanten $N_{+,n}/N_{-,l}$ und ungetriebenen Eigenenergien $E_{+,n}/E_{-,l}$ wie beim einzelnen getriebenen Oszillator definiert.
  Nur die Eigenfrequenz $\omega_0$ wird durch die neuen Eigenfrequenzen $\omega_+/\omega_-$ (\ref{neue_eigenfrequenzen}) ersetzt .

  Wie in Abschnitt \ref{kap:schroedglg_unabh_H} beschrieben, ist $\Psi$ in den neuen Variablen das Produkt aus den unabhängigen Wellenfunktionen $\Psi_+$ und $\Psi_-$, welche Lösungen der einzelnen Schrödinger-Gleichungen mit $H_+(t)/H_-(t)$ sind. Dadurch ist $\Psi$ auch auf dem Raum, das heißt auf $x_\pm \in (-\infty,\infty)$, normiert, wie am Anfang von Kapitel (\ref{3}) gezeigt wurde:
  \begin{align}
    &\int_{-\infty}^{\infty}\int_{-\infty}^{\infty} |\Psi_{n,l}(x_1,x_2,t)|^2 \: \text dx_1\text dx_2
    =\int_{-\infty}^{\infty}\int_{-\infty}^{\infty} |\Psi_{n,l}(x_+,x_-,t)|^2 \: \text dx_+\text dx_- \notag\\
    &=\int_{-\infty}^{\infty} |\Psi_{+,n}(x_+,t)|^2 \: \text dx_+ \int_{-\infty}^{\infty} |\Psi_{-,l}(x_-,t)|^2 \: \text dx_-
    = 1 \cdot 1 = 1 \; .
    \label{norm_gekoppelt}
  \end{align}

  In den ursprünglichen Koordinaten sind Kopplungsterme $\propto x_1x_2$ in der Wellenfunktion vorhanden.
  Durch die Hermit-Polynome $H_n/H_l$ treten hierbei im Allgemeinen unterschiedliche Potenzen der $x_1$ und $x_2$ auf.

  Die Floquet-Moden $\Phi_{n,l}(x_+,x_-,t)=\Phi_{+,n}(x_+,t)\Phi_{-,l}(x_-,t)$ sind in den neuen Koordinaten das Produkt aus den einzelnen Floquet-Moden (\ref{floquet_moden_einzelner}).
  Die Quasienergien sind dadurch die Summe der einzelnen Quasienergien (\ref{ep}):\\ $\epsilon_{n,l}=\epsilon_{+,n}+\epsilon_{-,l}$.


\section{Berechnung von Erwartungswerten}
  \label{erwartungswerte_gekoppelt}
  Um die Erwartungswerte für die Observablen $x_1,x_2,p_1,p_2$ zu berechnen, verwenden wir die benutzten Tranformationen für den Ort (\ref{koord_trafo_x}) und den Impuls (\ref{koord_trafo_p}), mit welchen wir die Berechnung auf die schon bekannten Erwartungswerte des einzelnen getriebenen Oszillators zurückführen können.
  %Denn wir berechnen die Erwartungswerte für die neuen Observablen  $x_1,x_2,p_1,p_2$
  Es werden alle Erwartungswerte für eine allgemeine $T$-periodische Treibkraft $S(t)$ berechnet und einige Erwartungswerte für die sinusförmige Treibkraft aus Kapitel (\ref{epsilon_sinuskraft}), für einfache Konstanten, visualisiert (\ref{ferg}).

  Die „$+$" und „$-$" Operatoren wirken auf verschiedene Variablen, das heißt, sie operieren auf den verschiedenen Räumen $\cal H_+,\cal H_-$.
  Aus diesem Grund gilt für den Erwartungswert eines Operators $U_+:\cal H_+ \rightarrow \cal H_+$ bezüglich einer Wellenfunktion der Form $\Psi_{n,l}=\Psi_{+,n}\Psi_{-,l}$ mit $\Psi_\pm \in \cal H_\pm$, wegen der Normierung nach Formel (\ref{norm_gekoppelt}):
  \begin{align}
    \braket{U_+}_{n,l} &= \int_{-\infty}^{\infty}\int_{-\infty}^{\infty} \Psi^*U_+\Psi \: \text dx_+ \text dx_-
    = \int_{-\infty}^{\infty} |\Psi_-|^2 \: \text dx_- \int_{-\infty}^{\infty} \Psi_{+,n}^*U_+\Psi_{+,n} \: \text dx_+ \notag\\
    &= 1 \cdot \braket{U_+}_n^+ \; .
    \label{erwartungswert_gekoppelt}
  \end{align}
  Für einen Operator $U_+$ ist der Erwartungswert der Gesamtwellenfunktion $\Psi_{n,l}$ der gekoppelten getriebenen Oszillatoren also identisch zum Erwartungswert der Wellenfunktion des einzelnen getriebenen Oszillators, in den „+"-Koordinaten.
  Für $U_-$ ergibt sich dies analog.

  Demnach folgt für den Erwartungswert $\braket{U_+U_-}_{n,l}$
  \begin{equation}
    \braket{U_+U_-}_{n,l}=\braket{U_-U_+}_{n,l} = \braket{U_+}_n^+\braket{U_-}_l^- \; .
    \label{erwartungswert_gekoppelt_produkt}
  \end{equation}
\iffalse
  \textbf{Graphische Darstellung}\\
  F"ur die graphische Darstellung der Erwartungswerte werden dimensionslose Gr"o"sen eingef"uhrt, indem $E_0=\hbar  \omega_0/2$ als charachteristische Energie des Systems betrachtet wird:
\iffalse
  \begin{align}
    \frac{H}{\hbar\omega_0}&=H_S(t)=\frac{2p^2}{2m\hbar\omega_0}+\frac{1}{2}\frac{m}{\hbar\omega_0}\omega_0^2x^2-\frac{S(t)}{\hbar\omega_0}x\notag\\
    =
  \end{align}
\fi
  \begin{align}
    &\frac{H(t)}{E_0}\frac{H'(t)}=\frac{p^2}{p_0}+\frac{x^2}{x_0}-\frac{A}{A_0}\sin(\omega t)\frac{x}{x_0}
    =p'+x'-A'\sin(\omega t)x' \;, %\notag\\
    \text{mit}\;p_0=m\omega s \;,\;x_0=s\;,\;A_0=\frac{\hbar \omega_0}{s} \notag\\
    &\text{d.\,h.}\;
  \end{align}
  wobei $s$ wieder die charakteristische L"ange nach (\ref{charak_laenge}) ist.
\fi

  \subsection{Zeitabhängige Erwartungswerte des Ortes}
    Mit der Transformation (\ref{koord_trafo_x}), der Überlegung zum gekoppelten Erwartungswert (\ref{erwartungswert_gekoppelt}) und dem Erwartungswert des einzelnen getriebenen Oszillators (\ref{erwartungswert_x_einzelner}) aus Kapitel \ref{3}, sind nun die Erwartungswerte
    $\braket{x_1}_{n,l}/\braket{x_2}_{n,l}$ bekannt:
    \begin{align}
      \braket{x_{1 \atop 2}}_{n,l}=\frac{1}{\sqrt{2}}\left(\braket{x_+}_{n,l}\mp\braket{x_-}_{n,l}\right)
      =\frac{1}{\sqrt{2}}\left(\braket{x_+}_n^+\mp\braket{x_-}_l^-\right)
      =\frac{1}{\sqrt{2}}(\zeta_+(t)\mp\zeta_-(t)) \; .
    \end{align}
    Der Mittelwert des Ortes für einen der beiden Oszillatoren $x_1$ oder $x_2$ des gekoppelten Systems ist somit die quantenzahlunabhängige Differenz oder die Summe der beiden klassischen Lösungen $\zeta_+(t)$ und $\zeta_-(t)$ der entkoppelten Oszillatoren nach der Variablentransformation, multipliziert mit $1/\sqrt 2$.

    Der Erwartungswert für $x_{1}^2/x_2^2$ ergibt sich darüber hinaus mit Formel (\ref{erwartungswert_gekoppelt_produkt}) und zusätzlich dem Erwartungswert (\ref{erwartungswert_x^2_einzelner}) zu
    \begin{align}
      \braket{x_{1\atop 2}^2}_{n,l}&=\frac{1}{2}\braket{(x_+\mp x_-)^2}_{n,l}
      =\frac{1}{2}\left(\braket{x_+^2}_n^+ \mp 2\braket{x_+}_n^+\braket{x_-}_l^- + \braket{x_-^2}_l^+\right) \notag\\
      &=\frac 1 2 (s_+^2(2n+1)+\zeta_+^2(t) \mp 2\zeta_+(t)\zeta_-(t) + s_-^2(2l+1)+\zeta_-^2(t)) \; ,
    \end{align}
    mit $s_\pm$ als charakteristische Länge des ungetriebenen Oszillators nach Formel (\ref{charak_laenge}), mit der jeweiligen Eigenfrequenz $\omega_\pm$.

    \textbf{Sinusoidale Treibkraft}

    Für eine sinusoidale Treibkraft $S(t)=A\sin(\omega t)$ und entsprechenden $S_\pm(t)$ (\ref{gekoppelte_H_entkoppelt}) sind $\zeta_\pm(t)$ gegeben durch (\ref{zeta_sinuskraft})
    \begin{align}
      \zeta_\pm(t) = \frac{\pm\frac{1}{\sqrt 2} A\sin(\omega t)}{m(\omega_\pm^2 - \omega^2)} \; .
    \end{align}
    %Betrachten wir, wann
    Hiermit werden im Folgenden die Erwartungswerte $\braket{x_1}_{n,l}$ und $\braket{x_2}_{n,l}$ für die Konstanten
    \begin{equation}
      \omega=\SI{2}{Hz} \;,\; A=\SI{1}{\newton} \;,\; k=\SI{1}{\newton \meter^{-1}} \;,\; m=\SI{1}{\kilo \gram} \;,\; \hbar=\SI{1}{\joule \second} \;
    \end{equation}
    dargestellt.
    Hierbei untersuchen wir, für welche Kopplungskonstanten $\kappa$ die beiden Erwartungswerte gleichphasig oder gegenphasig bzw. mit gleichem oder unterschiedlichem Vorzeichen der Amplitude schwingen, also bei welchem $\kappa$ die Amplitude $\braket{x_1}_{n,l}$ ihr Vorzeichen ändert.
    Wir betrachten somit die Situation einer schwachen (gegenphasigen) und starken (gleichphasigen) Kopplung.

    Die Amplitude von $\braket{x_2}_{n,l}$ kann ihr Vorzeichen nicht ändern, da negative physikalisch Kopplungskonstanten nicht sinnvoll sind. Der Erwartungswert verschwindet genau für $\kappa=\SI{0}{\newton \meter^{-1}}$, wie in Abb.(\ref{fig:x2_null}) veranschaulicht.
    Dies entspricht der Erwartung, da es sich bei $x_2$ ohne Kopplung um einen einzelnen ungetriebenen Oszillator handelt, da nur $x_1$ getrieben ist.

    Der Erwartungswert von $x_1$ wechselt sein Vorzeichen für $\kappa=\omega^2m-k=\SI{3}{\newton \meter^{-1}}$, wie in Abb.(\ref{fig:x1_null}) gezeigt.
    Die gegenphasige (schwache) Kopplung ist für $\kappa=\SI{1}{\newton \meter^{-1}}<\SI{3}{\newton \meter^{-1}}$ in Abb.(\ref{fig:schwach}) dargestellt. Wie klassisch anzunehmen, ist die Amplitude des Erwartungswertes $\braket{x_2}_{n,l}$ des angekoppelten Oszillators in diesem Fall geringer als die des getriebenen Oszillator-Erwartungswertes $\braket{x_1}_{n,l}$, da weniger Energie auf $x_2$ übetragen wird.
    Die gleichphasige (starke) Kopplung ist für $\kappa=\SI{15}{\newton \meter^{-1}}>\SI{3}{\newton \meter^{-1}}$ in Abb.(\ref{fig:stark}) dargestellt.
    Für eine starke Kopplung werden die Erwartungswerte identisch und die Amplitude geringer, was der klassischen Anschauung zweier starr gekoppelter Oszillatoren entspricht.
\iffalse
    \begin{figure}
      \begin{subfigure}[t]{0.5\textwidth}
        \centering
        \includegraphics[width=\textwidth]{plots/<x2>nl0.png}
        \caption{$\kappa=0$.}
        \label{fig:x2_null}
      \end{subfigure}
      %\quad
      \begin{subfigure}[t]{0.5\textwidth}
          \centering
          \includegraphics[width=\textwidth]{plots/<x1>nl0.png}
          \caption{$\kappa=\omega^2m-k=3$.}
          \label{fig:x1_null}
      \end{subfigure}
      %\quad
      \begin{subfigure}[t]{0.5\textwidth}
        \centering
        \includegraphics[width=\textwidth]{plots/<x12>nlschwach.png}
        \caption{$\kappa=1$.}
        \label{fig:schwach}
      \end{subfigure}
      %\quad
      \begin{subfigure}[t]{0.5\textwidth}sinusoidal
          \centering
          \includegraphics[width=\textwidth]{plots/<x12>nlstark.png}
          \caption{$\kappa=15$.}
          \label{fig:stark}
      \end{subfigure}
      %\quad
      \begin{subfigure}[t]{0.5\textwidth}
        \centering
        \includegraphics[width=\textwidth]{plots/<H>00_kappa0.png}
        \caption{$\kappa=0$.}
        \label{fig:H_kappa0}
      \end{subfigure}
      %\quad
      \begin{subfigure}[t]{0.5\textwidth}
          \centering
          \includegraphics[width=\textwidth]{plots/<H>00_kappa15.png}
          \caption{$\kappa=15$.}
          \label{fig:H_kappa15}
      \end{subfigure}
      \caption{Erwartungswerte der gekoppelten Oszillatoren $\braket{x_1}_{n,l}$ und $\braket{x_2}_{n,l}$ für beliebigen Zustand $\Psi_{n,l}$, außerdem der Erwartungswert der Gesamtenergie $\braket{H(t)}_{0,0}$ und zeitliches Mittel $\overline{H}_{0,0}$ für den Grundzustand $\Psi_{0,0}$. Mit $\omega=2 , A=1 , k=1 , m=1 , \hbar=1$, für verschiedene Kopplungskonstanten $\kappa$.}
    \end{figure}
\fi

  \subsection{Zeitabhängige Erwartungswerte des Impulses}
    Die Transformation der Impulsoperatoren (\ref{koord_trafo_p}) hat die gleiche Form wie bei den Ortsoperatoren (\ref{koord_trafo_x}), die Berechnung der Erwartungswerte $\braket{p_{1,2}}_{n,l}$ und $\braket{p_{1,2}^2}_{n,l}$ erfolgt darum analog, mit den Impuls-Erwartungswerten (\ref{erwartungswert_p_einzelner}) und (\ref{erwartungswert_p^2_einzelner}).
    Es gilt:
    \begin{align}
      \braket{p_{1 \atop 2}}_{n,l}=\frac{1}{\sqrt{2}}\left(\braket{p_+}_{n,l}\mp\braket{p_-}_{n,l}\right)
      =\frac{1}{\sqrt{2}}\left(\braket{p_+}_n^+\mp\braket{p_-}_l^-\right)
      =\frac{m}{\sqrt{2}}(\dot\zeta_+(t)\mp\dot\zeta_-(t)) \;
    \end{align}
    und
    \begin{align}
      \braket{p^2_{1\atop 2}}_{n,l}&=\frac{1}{2}\braket{(p_+\mp p_-)^2}_{n,l} %\notag\\
      %=\frac{1}{2}\left(\braket{p_+^2}_n^+ \mp 2\braket{p_+}_n^+\braket{x_-}_l^- - \braket{x_-^2}_n^+\right) \notag\\
      =\frac {m^2} 2 (\omega_+^2s_+^2(2n+1)+\dot\zeta_+^2(t) \mp 2\dot\zeta_+(t)\dot\zeta_-(t) + \omega_-^2s_-^2(2l+1)+\dot\zeta_-^2(t)) \; .
    \end{align}
    Die Erwartungswerte sind ähnlich zu denen des Ortes.
    %Wie beim einzelnen Oszillator
    Es kommt nur ein Faktor hinzu und die $\zeta_\pm(t)$ werden zu ihren Ableitungen.

    Für eine sinusförmige Treibkraft, sehen die Erwartungswerte fur $p_{1,2}$ demnach so aus wie die $\braket{x_{1,2}}_{n,l}$ nur mit skalierter Amplitude und phasenverschoben um $\pi$.
    Das Verhalten für die verschiedenen Kopplungskonstanten $\kappa$ ist ebenfalls identisch.

  \subsection{Erwartungswerte der Energie}
    Der zeitabhängige Erwartungswert $\braket{H(t)}_{n,l}$ des Hamilton-Operators für eine beliebige Treibkraft $S(t)$ ist auch mit dem Erwartungswert des einzelnen Oszillators berechenbar, indem wir wieder in den neuen Koordinaten arbeiten:
    \begin{align}
      \braket{H(t)}_{n,l} = \braket{H_+(t)}_n^+ + \braket{H_-(t)}_l^- %\notag\\
      = E_{+,n}+E_{-,l} -L_+-L_- +m\dot\zeta_+^2(t)+m\dot\zeta_-^2(t) \; .
    \end{align}
    Außerden wird der mittlere Erwartungswert $\overline{H}_{n,l}$ wieder, in den neuen Koordinaten, mit Formel (\ref{mittleres_H}), berechnet.
    Weil die Quasienergien $\epsilon_{n,l}$ die Summe von $\epsilon_{+,n}$ und $\epsilon_{-,l}$ ist, erhalten wir
    \begin{align}
      \overline{H}_{n,l}&=\overline{H}_{+,n}+\overline{H}_{-,l}  \notag\\
      &=E_{+,n}+E_{-,l} - \frac{\frac{1}{\sqrt 2} A}{4m(\omega_+^2-\omega^2)}\left(1-\frac{2\omega^2}{(\omega_+^2-\omega^2)}\right)
      - \frac{\frac{-1}{\sqrt 2} A}{4m(\omega_-^2-\omega^2)}\left(1-\frac{2\omega^2}{(\omega_-^2-\omega^2)}\right) \; .
      \label{mittleres_H_gekoppelt}
    \end{align}

    \textbf{Sinusoidale Treibkraft}

    Für das Beispiel der sinusoidalen Kraft folgt $\braket{H(t)}_{n,l}$ mit den ensprechenden Orts- und Impulserwartungswerten aus dem vorherigen Abschnitt und  $\overline{H}_{n,l}$ folgt aus Gleichung (\ref{mittleres_H_gekoppelt}).
    Der Erwartungswert der Energie $\braket H_{0,0}$ und dessen zeitliches Mittel $\overline{H}_{0,0}$ sind für die gleichen Konstanten wie zuvor
    \begin{equation}
      \omega=\SI{2}{Hz} \;,\; A=\SI{1}{\newton} \;,\; k=\SI{1}{\newton \meter^{-1}} \;,\; m=\SI{1}{\kilo \gram} \;,\; \hbar=\SI{1}{\joule \second}  \; ,
    \end{equation}
    in Abb.(\ref{ferg}) aufgetragen.
    Da beide Erwartungswerte wegen den $E_{+,n}/E_{-,l}$ abhängig von den Quantenzahlen sind, wird hier der einfachste Fall $n=l=0 \Rightarrow E_{\pm,0}=\hbar \omega_\pm/2$ betrachtet.
\iffalse
    \begin{figure}
      \begin{subfigure}[t]{0.5\textwidth}
        \centering
        \includegraphics[width=\textwidth]{plots/<H>00_kappa0.png}
        \caption{$\kappa=0$.}
        \label{fig:H_kappa0}
      \end{subfigure}
      \quad
      \begin{subfigure}[t]{0.5\textwidth}
          \centering
          \includegraphics[width=\textwidth]{plots/<H>00_kappa15.png}
          \caption{$\kappa=15$.}
          \label{fig:H_kappa15}
      \end{subfigure}
      \caption{Erwartungswert der Gesamtenergie $\braket{H(t)}_{0,0}$ und zeitliches Mittel $\overline{H}_{0,0}$ für den Grundzustand $\Psi_{0,0}$ mit $\omega=2 \;,\; A=1 \;,\; k=1 \;,\; m=1 \;,\; \hbar=1$, für verschiedene Kopplungskonstanten $\kappa$.}
    \end{figure}
\fi
    Ohne Kopplung, das heißt $\kappa=\SI{0}{\newton \meter^{-1}}$ (\ref{fig:H_kappa0}) ist die mittlere Energie am geringsten, weil nur $x_1$ schwingt.
    Für eine möglichst starke/starre Kopplung $\kappa=\SI{15}{\newton \meter^{-1}}$ wird die mittlere Energie im System maximal, weil die Treibkraft so möglichst viel Energie auf den angekoppelten Oszillator $x_2$ übertragen kann.
    Diese Situation ist in Abb. (\ref{fig:H_kappa15}) dargestellt.
    \begin{figure}
      \begin{subfigure}[t]{0.5\textwidth}
        \centering
        \includegraphics[width=\textwidth]{plots/<x2>nl0.png}
        \caption{$\kappa=\SI{0}{\newton \meter^{-1}}$.}
        \label{fig:x2_null}
      \end{subfigure}
      %\quad
      \begin{subfigure}[t]{0.5\textwidth}
          \centering
          \includegraphics[width=\textwidth]{plots/<x1>nl0.png}
          \caption{$\kappa=\omega^2m-k=\SI{3}{\newton \meter^{-1}}$.}
          \label{fig:x1_null}
      \end{subfigure}
      %\quad
      \begin{subfigure}[t]{0.5\textwidth}
        \centering
        \includegraphics[width=\textwidth]{plots/<x12>nlschwach.png}
        \caption{$\kappa=\SI{1}{\newton \meter^{-1}}$.}
        \label{fig:schwach}
      \end{subfigure}
      %\quad
      \begin{subfigure}[t]{0.5\textwidth}
          \centering
          \includegraphics[width=\textwidth]{plots/<x12>nlstark.png}
          \caption{$\kappa=\SI{15}{\newton \meter^{-1}}$.}
          \label{fig:stark}
      \end{subfigure}
      %\quad
      \begin{subfigure}[t]{0.5\textwidth}
        \centering
        \includegraphics[width=\textwidth]{plots/<H>00_kappa0.png}
        \caption{$\kappa=\SI{0}{\newton \meter^{-1}}$.}
        \label{fig:H_kappa0}
      \end{subfigure}
      %\quad
      \begin{subfigure}[t]{0.5\textwidth}
          \centering
          \includegraphics[width=\textwidth]{plots/<H>00_kappa15.png}
          \caption{$\kappa=\SI{15}{\newton \meter^{-1}}$.}
          \label{fig:H_kappa15}
      \end{subfigure}
      \caption{Erwartungswerte der gekoppelten Oszillatoren $\braket{x_1}_{n,l}$ und $\braket{x_2}_{n,l}$ für einen beliebigen Zustand $\Psi_{n,l}$, außerdem der Erwartungswert der Gesamtenergie $\braket{H(t)}_{0,0}$ und zeitliches Mittel $\overline{H}_{0,0}$ für den Grundzustand $\Psi_{0,0}$. Mit $\omega=\SI{2}{Hz}\;,\; A=\SI{1}{\newton} \;,\; k=\SI{1}{\newton\meter^{-1}} \;,\; m=\SI{1}{\kilo \gram} \;,\; \hbar=\SI{1}{\joule \second}$, für verschiedene Kopplungskonstanten $\kappa$.}
      \label{ferg}
    \end{figure}
