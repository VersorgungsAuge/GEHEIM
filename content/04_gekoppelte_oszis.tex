\chapter{Zwei gekoppelte getriebene harmonische Oszillatoren in der Quantenmechanik}


In diesem Teil der Arbeit wird mit Hilfe der aus (\ref{lsg_einzelner}) bekannten Lösung des einzelnen getrieben Oszillators, die Wellenfunktionen für ein System hergeleitet, dass aus zwei gekoppelten Oszillatoren $x_1$ und $x_2$ der gleichen Masse $m$ besteht, von denen einer mit der periodischen Kraft $S(t) = S(t+T)$ angetrieben wird.
Die Potentialkonstanten $k$ der beiden Oszillatoren sind ebenfalls identisch, die Kopplungskonstante $\kappa$ zwischen den Oszillatoren ist allerdings anders.
Der Hamilton-Operator dieses Systems kann direkt aus der klassichen Mechanik übernommen werden:
\begin{equation}
  H(t) = H(t+T) = \frac{p_1^2}{2m} + \frac{p_2^2}{2m} + \frac 1 2 kx_1^2 + \frac 1 2 kx_2^2 + \frac 1 2 \kappa(x_2-x_1)^2 - S(t)x_1 \; .
  \label{H_gekoppelt}
\end{equation}
Es werden auch die Erwartungswerte für  den Ort $\braket{x_{1,2}}_{n,l}$ und den Impuls $\braket{p_{1,2}}_{n,l}$, genauso wie der Erwartungswert der Energie $\braket{H}_{n,l}$ und dessen zeitliches Mittel $\bar{H}_{n,l}$ berechnet, indem auf die bekannten Erwartungswerte des einzelnen getriebenen Oszillators zurückgeführt wird.
Die Erwartungswerte werden auch graphisch dargestellt.

Zur Lösung des Systems wird eine unitäre Koordinatentransformation eingeführt, welche den Hamilton-Operator $H(x_1,x_2,p_1,p_2,t)$ zu zwei in den neuen Koordinaten unabhängigen Hamilton-Operatoren $H_+(x_+,p_+,t)$ und $H_-(x_-,p_-,t)$ mit effektiven Potentialkonstanten $k_+,k_-$ entkoppelt, welche je einen einzelnen (getriebenen) Oszillator beschreiben.
Dann ergeben sich die Wellenfunktionen leicht aus denen des einzelnen Oszillators.


%war vorher subsection
\section{Die Schrödinger-Gleichung mit unabhängigen Hamilton-Operatoren}
  Liegt ein Hamilton-Operator der Form
  \begin{equation}
    H = \sum_i H_i(x_i,p_i,t) \;,\; H_i : \cal H_i \rightarrow H_i
    \label{unabh_H}
  \end{equation}
  vor, führt der Ansatz
  \begin{equation}
    \Psi = \prod_i \Psi(x_i,p_i,t) \; , \; \Psi_i \in \cal H_i
    \label{lsg_unabh_H}
  \end{equation}
  auf unabhängige Schrödinger-Gleichungen für die einzelnen Wellenfunktionen \\ $\Psi_i(x_i,p_i,t)$, sodass \cite{online quelle}
  \begin{equation}
    \text i \hbar \frac{\partial}{\partial t}\Psi_i = \delta_{i,j}  H_j \Psi_i
    \label{schroedglg_unabh_H}
  \end{equation}
  erfüllt ist.
  Um dies schnell zu zeigen, schauen wir uns den Fall von zwei unabhängigen Operatoren an:
  \begin{align}
    \begin{split}
      \text i \hbar \frac{\partial}{\partial t} \Psi = H\Psi \iff \Psi_2 H_1 \Psi_1 + \Psi_1 H_2 \Psi_2 = \text i \hbar \left(\frac{\partial}{\partial t} \Psi_1\Psi_2 + \Psi_1\frac{\partial}{\partial t} \Psi_2 \right) \; .
    \end{split}
  \end{align}
  Da die Operatoren nur auf Funktionen wirken, die auf dem selben Raum definiert sind, kann man sie an der jeweils anderen Funktion vorbei ziehen.
  Wenn wir weiterhin durch unseren Ansatz $\Psi_1\Psi_2$ teilen, wird die Gleichung zu:
  \begin{equation}
    \frac{1}{\Psi_1}H_1\Psi_1 + \frac{1}{\Psi_2}H_2\Psi_2 = \text i \hbar \left(\frac{\frac{\partial}{\partial t} \Psi_1}{\Psi_1} + \frac{\frac{\partial}{\partial t} \Psi_2}{\Psi_2} \right) \; .
  \end{equation}
  Weil die linke und rechte Seite für alle unabhängigen $x_1,p_1,x_2,p_2$ gleich sein müssen, folgen die einzelnen Schrödinger-Gleichungen für die Wellenfunktionen $\Psi_1$ und $\Psi_2$:
  \begin{equation}
    \frac{1}{\Psi_1}H_1\Psi_1 = \frac{\frac{\partial}{\partial t} \Psi_1}{\Psi_1} \; , \; \frac{1}{\Psi_2}H_2\Psi_2 = \frac{\frac{\partial}{\partial t} \Psi_2}{\Psi_2} \; .
  \end{equation}
  Wie an diesem Beispiel leicht zu sehen, gilt Gleiches auch für beliebig viele unabhängige Hamilton-Operatoren, womit (\ref{schroedglg_unabh_H}) erfüllt ist und wir wissen, dass die Lösung $\Psi$ die Form (\ref{lsg_unabh_H}) hat.
  %nur EINE moegliche lsg hat die form psi, wenn wir einschraenken, dass psi so aussehen soll, wer weis ob es noch andere lsgen nicht dieser form gibt.



\section{Unitäre Variablenformation und allgemeine Lösung der Schrödinger-Gleichung}
  Um für unseren Hamilton-Operator $H(x_1,x_2,p_1,p_2,t)$ (\ref{H_gekoppelt}) eine entkoppelte Form $H(x_+,x_-,p_+,p_-,t)=H_+(x_+,p_+,t)+H_-(x_-,p_-,t)$ nach (\ref{unabh_H}) in den neuen Koordinaten/Variablen $x_+,x_-,p_+,p_-$ zu erhalten, wählen wir die unitären Koordinationtransformationen \cite{arxiv}
  \begin{equation}
    x_+ = \frac{1}{\sqrt{2}}(x_2+x_1) \;,\; x_-=\frac{1}{\sqrt{2}}(x_2-x_1) \;,
  \end{equation}
  welche aus den Normalmoden des klassischen Problems folgen, worauf in Kapitel () genauer eingegangen wird.
  Durch einfaches Umstellen folgen $x_1$ und $x_2$ in Abhängigkeit der neuen Koordinaten $x_+$ und $x_-$:
  \begin{equation}
    x_1=\frac{1}{\sqrt{2}}(x_+-x_-) \;,\; x_2=\frac{1}{\sqrt{2}}(x_++x_-) \; .
  \end{equation}
  Indem wir $x_1$ und $x_2$ so im Hamilton-Operator (\ref{H_gekoppelt}) ersetzen, formen wir den ortsabhängigen Teil um zu
  \begin{align}
    &\frac{1}{2}kx_1^2+\frac{1}{2}kx_2^2+\frac{1}{2}\kappa(x_2-x_1)^2-S(t)x_1= \notag\\
    &\frac{1}{2}kx_+^2+\frac{1}{2}(k+2\kappa)x_-^2-S(t)\frac{1}{\sqrt{2}}(x_+-x_-) \; .
  \end{align}
  Es tauchen nun keine Kopplungsterme $x_+x_-$ mehr auf, wie es in den alten Koordinaten der Fall war.

  Jetzt betrachten wir die neuen Impulse und überprüfen, dass es durch den Variablenwechsel nicht zu neuen Kopplungstermen in den $p_+$,$p_-$ kommt.
  Mit der Kettenregel folgt für die Ableitungen der Impulsoperatoren
  \begin{align}
    \frac{\partial}{\partial x_{\pm}} = \frac{\partial}{\partial x_1}\frac{\partial x_1}{\partial x_{\pm}} + \frac{\partial}{\partial x_2}\frac{\partial x_2}{\partial x_{\pm}}
    =\frac{\partial}{\partial x_1}\left(\pm\frac{1}{\sqrt{2}}\right)
    + \frac{\partial}{\partial x_2}\frac{1}{\sqrt{2}}
    = \frac{1}{\sqrt{2}}\left(\frac{\partial}{\partial x_2}\pm\frac{\partial}{\partial  x_1}\right) \;,
  \end{align}
  weshalb für die Impulsoperatoren identisch zu den Ortsoperatoren gilt
  \begin{equation}
    p_+ = \frac{1}{\sqrt{2}}(p_2+p_1) \;,\; p_-=\frac{1}{\sqrt{2}}(p_2-p_1) \; .
  \end{equation}
  Wir setzen erneut in unseren Hamilton-Operator (\ref{H_gekoppelt}) ein und erhalten für den impulsabhängigen Teil
  \begin{equation}
    \frac{p_1^2}{2m} + \frac{p_2^2}{2m} = \frac{p_+^2}{2m} + \frac{p_-^2}{2m}
  \end{equation}

  Wie erwartet bleibt die Summe der quadrierten Impulsoperatoren unverändert.

  Der gesamte Hamilton-Operator der zwei gekoppelten getrieben Oszillatoren ist in den neuen Variablen folglich
  \begin{align}
    H(t) &= H(t+T) = \frac{p_1^2}{2m} + \frac{p_2^2}{2m} + \frac 1 2 kx_1^2 + \frac 1 2 kx_2^2 + \frac 1 2 \kappa(x_2-x_1)^2 - S(t)x_1 \notag\\
    &= \frac{p_+^2}{2m}+\frac{1}{2}kx_+^2-\frac{1}{\sqrt{2}}S(t)x_+ \quad + \quad
    \frac{p_-^2}{2m}+\frac{1}{2}(k+2\kappa)x_-^2+\frac{1}{\sqrt{2}}S(t)x_- \notag\\
    &= \frac{p_+^2}{2m}+\frac{1}{2}k_+x_+^2-S_+(t)x_+ \quad + \quad
    \frac{p_-^2}{2m}+\frac{1}{2}k_-x_-^2-S_-(t)x_- \notag\\
    &= H_+(x_+,p_+,t) + H_-(x_-,p_-,t) \; .
  \end{align}
  Der Hamilton-Operator in den neuen Variablen $x_{\pm},p_{\pm}$ beschreibt demnach ein System aus zwei unabhängigen getriebenen harmonischen Oszillatoren mit neuen Potentialkonstanten $k_1=k$ und $k_2=k+2\kappa$ bzw. neuen Eigenfrequenzen
  \begin{equation}
    w_+=\sqrt{\frac{k}{m}} \quad\text{und}\quad \omega_-=\sqrt{\frac{k+2\kappa}{m}} \; .
  \end{equation}
  Die beiden Oszillatoren werden, wegen dem verschiedenen Vorzeichen von $S_+(t)=S_+(t+T)$ und $S_-(t)=S_-(t+T)$, periodisch aber phasenversetzt um $\pi$ mit der ursprünglichen Treibkraft $S(t)$ angetrieben, wobei diese mit dem Faktor $1/\sqrt{2}$ skaliert wird.

  Betrachten wir den komplizierter geglaubten Fall, dass beide Oszillatoren $x_1$ und $x_2$ des Systems in den alten Koordinaten getrieben sind, und zwar genau wie durch die neuen Koordinaten $x_+$ und $x_-$ vorgegeben, das heißt
  \begin{align}
      &H(t) = \frac{p_1^2}{2m} + \frac{p_2^2}{2m} + \frac 1 2 kx_1^2 + \frac 1 2 kx_2^2 + \frac 1 2 \kappa(x_2-x_1)^2 - S(t)(x_2+x_1) \notag\\
      \text{oder} \; &H(t) = \frac{p_1^2}{2m} + \frac{p_2^2}{2m} + \frac 1 2 kx_1^2 + \frac 1 2 kx_2^2 + \frac 1 2 \kappa(x_2-x_1)^2 - S(t)(x_2-x_1) \;,
  \end{align}
  liegt in den neuen Variablen ein vereinfachtes System vor, bei dem nur ein Oszillator $x_+$ oder $x_-$ getrieben ist.
  Es muss also in den klassischen Normalmoden getrieben werden, damit nach dem Variablenwechsel ein möglichst einfaches System mit nur noch einem getriebenen Oszillator vorliegt.
