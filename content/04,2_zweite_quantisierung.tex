\chapter{Zweite Quantisierung des getriebenen harmonischen Oszillators}
\label{5}
In diesem Kapitel werden wir die Wellenfunktionen $\Psi_n(x,t)$ in der Ortsdarstellung (\ref{gesamtlsg_einzelner}) in die Besetzungszahldarstellung übersetzen, das heißt wir werden $\Psi(x,t)$ darstellen in der Orthonormalbasis der Eigenzustände $\ket{n}$ des Besetzungszahloperators $N$ des ungetriebenen Oszillators.
Die $\ket{n}$ sind auch Eigenzustände des Hamilton-Operators
\begin{equation}
   H_\text{ung}(x,p) = \frac{ p^2}{2m} + \frac 1 2 m\omega_0^2 x^2 = H_{\text{ung}}(N)= \hbar\omega_0\left(N+\frac 1 2\right)
\end{equation}
des ungetriebenen Systems.
Hierbei gilt
\begin{equation}
   N \ket{n} = n \ket{n} \;,\; n\in \mathbb{N} \;.
\end{equation}
\iffalse
\begin{equation}
  H=\frac{\hat p^2}{2m} + \frac 1 2 m\omega_0^2\hat x^2 = \hbar\eft(\hat n+\frac 1 2\right)
\end{equation}
\fi
Um das darstellungsunabhängige Ket $\ket{\Psi(t)}$ in der 2. Quantisierung zu erhalten, wird ein geeigneter Ansatz gewählt, welcher mit Hilfe der Lösungen in Ortsdarstellung gefunden wird.
Der Ansatz wird in die Schrödinger-Gleichung des getriebenen Oszillators eingesetzt, wonach mit Hilfe von Kommutatorrelationen Differentialgleichungen zur Bestimmung der Unbekannten hergeleitet werden, welche gelöst werden können.

Mit dem bekannten $\ket{\Psi(t)}$ werden im zweiten Teil erneut die Erwartungswerte $\braket{x^m}_n$ und $\braket{p^m}_n$ des einzelnen getriebenen Oszillators berechnet.


\section{Wellenfunktion}
  Da die $\ket{n}$ Eingenzustände des hermiteschen Besetzungszahloperators $N$ sind, kann jede Wellenfunktion $\ket{\Psi_n(t)}$ bzw. Gesamtwellenfunktion $\ket{\Psi(t)}$ auf dem selben Raum geschrieben werden als
  \begin{equation}
    \ket{\Psi(t)}=\sum_{n=0}^{\infty}\ket{\Psi_n(t)}=\sum_{n=0}^{\infty}c_n(t)\ket{n} \;,
  \end{equation}
  mit geeigneten Koeffizienten $c_n(t)$.

  Für die $\ket{\Psi_n(t)}$ des getriebenen harmonischen Oszillators wählen wir den Ansatz
  \begin{equation}
    \ket{\Psi_n(t)}=\exp\left(\frac{-\text i}{\hbar}\theta_n(t)\right)\exp\left(\frac{-\text i}{\hbar}\alpha(t)p\right)\exp\left(\frac{-\text i}{\hbar}\varphi(t)x\right)\ket{n}=K_nO_pO_x\ket{n} \;.
    \label{ansatz_quant}
  \end{equation}
  Dieser Ansatz folgt aus der Betrachtung der Lösung $\Psi_n(x,t)$ in Ortsdarstellung (\ref{gesamtlsg_einzelner}).
  Die ungetriebenen Eigenzustände $\ket{n}$ haben in der Ortsdarstellung die Form einer Gauß-Glocke.
  Da bei den $\Psi_n(x,t)$ zur (verschobenen) Gauß-Glocke eine zeit- und ortsabhängige Phase hinzu kommt und der Ortsoperator in Ortsdarstellung einfach der Multiplikation mit der Ortskoordinate enspricht, wählen wir
  \begin{equation}
    \exp\left(\frac{-\text i}{\hbar}\varphi(t)x\right)=O_x
  \end{equation}
  als Ansatz.
  Hier ist $x$ der Ortsoperator.
  Sowohl die zusätzliche orts- und zeitabhängige Phase als auch die Gauß-Glocke in $\Psi_n(x,t)$ sind verschoben (um $\zeta(t)$), daher wenden wir von links den Operator
  \begin{equation}
    \exp\left(\frac{-\text i}{\hbar}\alpha(t)p\right)=O_p
  \end{equation}
  an.
  Denn dieser Operator bewirkt bei der Transformation genau die Verschiebung des Ortsoperators (\ref{kommutator_1}), wie sie im Argument der Wellenfunktion $\Psi(x,t)$ vorliegt:
  \begin{equation}
    O_pxO_p^{-1}=
    \exp\left(\frac{-\text i}{\hbar}\alpha(t)p\right) x  \exp\left(\frac{+\text i}{\hbar}\alpha(t)p\right)
    =x-\alpha(t) \; .
    \label{verschiebung_x}
  \end{equation}
  %Dies wird noch mit Kommutatorrelationen gezeigt.
  Zuletzt wird die ausschließlich zeitabhängige Phase der Wellenfunktion $\Psi_n(x,t)$, in welche die einzige quantenzahlabhängige Größe $E_n$ eingeht, durch die Exponentialfunktion $K_n$ ohne Operator berücksichtigt.
  %durch die Funktion $\theta_n(t)$ in der Exponentialfunktion ganz links in (\ref{ansatz_quant}) berücksichtigt.

  Jetzt setzen wir diesen Ansatz (\ref{ansatz_quant}) in die Schrödinger-Gleichung (\ref{schroedinger}) ein.
  Das Ziel ist unseren Hamilton-Operator
  \begin{equation}
    H(x,p,t)=\frac{p^2}{2m}+\frac 1 2 m\omega_0 -S(t)x = H_{\text{ung}}(N)-S(t)x
  \end{equation}
  an den Operatoren im Ansatz vorbeizukommutieren, sodass $H_{\text{ung}}(N)$ auf die $\ket{n}$ angewendet wird.
  Währenddessen entstehen Resterme mit $p$ und $x$, die auch auf $\ket{n}$ wirken.
  Auf der anderen Seite der Gleichung müssen der $p$- und $x$-Operator, welche durch die Zeitableitung entstehen, ebenfalls bis zum $\ket{n}$ durchkommutiert werden.
  Dann erhalten wir die Differentialgleichungen zum Bestimmen von $\theta_n(t), \alpha(t)$ und $\varphi(t)$ durch Koeffizientenvergleich beider Seiten.

  Für dieses Vorgehen brauchen wir die Vertauschungsrelationen von $x/p$ mit $O_x/O_p$.
  Wenn beide Operatoren nur von $x$ oder $p$ abhängen, verschwindet der Kommutator, das heißt es gilt
  \begin{equation}
    [x^m,O_x]=[p^m,O_p]=0\iff x^mO_x=O_xx^m\quad\text{und}\quad p^mO_p=O_pp^m\;,\; m \in \mathbb{N} \; .
  \end{equation}
  Zusätzlich berechnen wir die gemischten Vertauschungsrelationen mit der Kommutatorrechenregel
  \begin{align}
    %[AB,C]=A[B,C]+[A,C]B\quad,\quad
    [A,f(B)]=\frac{\partial}{\partial B}f(B)[A,B]\quad (\text{für $[B,[A,B]]=0$})
  \end{align}
  und dem bekannten Kommutator $[x,p]=\text i \hbar$.
  Es ergibt sich:
  \begin{align}
    [x,O_p]&=\alpha(t)O_p\iff xO_p=O_p(x+\alpha(t))\quad\text{bzw.}\quad
    x^mO_p=O_p(x+\alpha(t))^m \label{kommutator_1}\\
    %[x^2,O_p]&=2\alpha(t)x+\alpha^2(t)\iff x^2O_p=O_p(x^2+2\alpha(t)x+\alpha^2(t))\\
    [p,O_x]&=-\varphi(t)O_x\iff pO_x=O_x(p-\varphi(t))\quad\text{bzw.}\quad
    p^mO_x=O_x(p-\varphi(t))^m \;, \label{kommutator_2} \\
    &\quad m \in \mathbb{N} \;. \notag
    %[p^2,O_x]&=-2\varphi(t)p+\varphi^2(t)\iff p^2O_x=O_x(p^2-2\varphi(t)p+\varphi^2(t)) \label{kommutator_2} \; .
  \end{align}

  Für die Anwendung des Hamilton-Operators $H(x,p,t)$ des getriebenen Oszillators auf den Zustand $\ket{\Psi(t)}$ folgt
  \begin{align}
    &H\ket{\Psi(t)}=K_nHO_pO_x\ket{n}
    =K_nO_p\left(H+\frac{1}{2m}(-2\varphi(t)p+\varphi^2(t))\right)O_x\ket{n} \notag\\
    &=K_nO_pO_x\left(H+\frac{1}{2m}(-2\varphi(t)p+\varphi^2(t))+\frac{1}{2}m\omega_0^2(2\alpha(t)x+\alpha^2(t))-\alpha(t)S(t)\right)\ket{n} \notag\\
    &=K_nO_pO_x
    \left(E_n-S(t)x+\frac{1}{2m}(-2\varphi(t)p+\varphi^2(t))+\frac{1}{2}m\omega_0^2(2\alpha(t)x+\alpha^2(t))-\alpha(t)S(t)\right)\ket{n} \;.
    \label{super1}
  \end{align}
  Weiterhin ist die Zeitableitung
  \begin{align}
    \text i\hbar\frac{\partial}{\partial t} \ket{\Psi(t)}
    &=K_n\left(\dot\theta(t)O_pO_x+O_p\dot\alpha(t)pO_x+O_pO_x\dot\varphi(t)x\right)\ket{n} \notag\\
    &=K_nO_pO_x\left(\dot\theta(t)+\dot\alpha(t)(p-\varphi(t))+\dot\varphi(t)x\right)\ket{n} \; .
    \label{super2}
  \end{align}
  %Indem auf beiden Seiten durch die Exponentialfunktion mit $\theta(t)$ dividiert wird
  %Der Koeffizientenvergleich der Operatoren von (\ref{super1}) und (\ref{super2}) liefert die Differentialgleichungen:
  Da alle Restterme auf beiden Seiten der Schrödinger-Gleichung auf die Besetzungzahlzustände $\ket{n}$ angewendet werden, wenden wir einen Koeffizientenvergleich von (\ref{super1}) und (\ref{super2}) an und setzen jeweils die Terme vor $x$, $p$ und die Terme ohne Operator gleich.
  Dies liefert die Differentialgleichungen zu Bestimmung der Zeitabhängigkeit der Phasen im Ansatz:
  \begin{align}
    &1)\; \dot\theta_n(t)=E_n+\frac 1 2 m\omega_0^2\alpha^2(t)+\frac{1}{2m}\varphi^2(t)+\dot\alpha(t)\varphi(t)-S(t)\alpha(t) \notag\\
    &2)\; \dot\varphi(t)=m\omega_0^2\alpha(t)-S(t) \quad,\quad 3) \;\dot\alpha(t)=\frac{-1}{m}\varphi(t) \;.
  \end{align}
  Einsetzen von Gleichung 2) in Gleichung 3) führt auf die klassische Bewegungsgleichung des getriebenen Oszillators (\ref{dgl_zeta}) für $\alpha(t)$
  \begin{equation}
    m\ddot\alpha(t)+m\omega_0\alpha(t)-S(t)=0 \;,
  \end{equation}
  weshalb $\alpha(t)=\zeta(t)$ ist, wie nach der Transformation (\ref{verschiebung_x}) erwartet, weil $x$ in der Lösung $\Psi(x,t)$ genau um $-\zeta(t)$ verschoben ist.
  Demnach ist $\varphi(t)=-m\dot\zeta(t)$ durch 3) gegeben.
  Weiterhin setzen wir das bekannte $\alpha(t)$ und $\varphi(t)$ in 1) ein und identifizieren die Lagrange-Funktion für $\zeta(t)$ (\ref{lagrange_zeta}). Somit erhalten wir
  \begin{align}
    &\dot\theta_n(t)=E_n+\frac{1}{2}m\omega_0^2\zeta^2(t)+\frac{1}{2m}\dot\zeta^2(t)-m\dot\zeta^2(t)-S(t)\zeta(t)
    =E_n-L(\dot\zeta,\zeta,t) \notag\\
    \Rightarrow &\theta_n(t)=E_nt-\int_0^tL(\dot\zeta,\zeta,t')\:\text dt' \;.
  \end{align}
  Daher ist $K_n$ genau die rein zeitabhängige Phase von $\Psi_n(x,t)$ (\ref{gesamtlsg_einzelner}) und $\ket{\Psi_n(t)}$ (\ref{ansatz_quant}) ist insgesamt
  \begin{align}
    \ket{\Psi_n(t)}&=K_nO_pO_x\notag\\
    &=\exp\left(\frac{\text i}{\hbar}\left(-E_nt+\int_0^tL(\dot\zeta,\zeta,t')\:\text dt'\right)\right)\exp\left(\frac{-\text i}{\hbar}\zeta(t)p\right)\exp\left(\frac{\text i}{\hbar}m\dot\zeta(t)x\right)\ket{n} \; .
    \label{psi_quant}
  \end{align}



\section{Berechnung von Erwartungswerten}
  Mit dem bekannten Wellenfunktion-Ket $\ket{\Psi(t)}$, sowie den Vertauschungsrelationen in (\ref{kommutator_1}-\ref{kommutator_2}) werden wir in diesem Teil die zeitabhängigen Erwartungswerte des Ortes $\braket{x^m}_n$ und des Impulses $\braket{x^m}_n$ des einzelnen getriebenen Oszillators in der Besetzungszahldarstellung bestimmen.

  Es wird verwendet, dass die Erwartungswerte der Zustände $\Psi_{n,\text{ung}}(y,t)$ des ungetriebenen Oszillators, gleich den Erwartungswerten der Besetzungszahlzustände $\ket{n}$ sind, da sich $\ket{n}$ und $\ket{\Psi_{n,\text{ung}}(t)}$ nur um eine komplexe Phase unterscheiden, welche bei den Erwartungswerten wegfällt.
  Außerdem kann beim Erwartungswert eines Operators $U$ für die Zustände $\ket{\Psi(t)}$ die komplexe Phase $K_n$ in $\ket{\Psi(t)}$ (\ref{psi_quant}) sofort am Operator vorbeigezogen werden und mit $K_n^{-1}$ zu 1 vereinfacht werden, weil $K_n$, im Gegensatz zu $O_x$ und $O_p$, kein Operator ist:
  %Der Erwartungswert eines Operators $A$ für der Zustände $\ket{\Psi(t)}$ ist gegeben durch:
  \begin{equation}
    \braket{U}_n=\braket{\Psi_n(t)|U|\psi_n(t)}=\braket{n|O_x^{-1}O_p^{-1}UO_pO_x|n} \;.
  \end{equation}
  Es liegt nun eine (Rück-)Transformation von $U$ \, bezüglich $O_p$ und dann $O_x$ vor.
  Für $U=x^m/p^m$ können wir das Ergebnis der Transformation direkt den Relationen (\ref{kommutator_1}-\ref{kommutator_2}) entnehmen, indem wir damit $U_{x/p}$ jeweils mit $p^m/x^m$ vertauschen und dann $U_{x/p}^{-1}U_{x/p}$ zu 1 zusammenfassen.
  Weil danach entweder nur $O_p$ und $p^m$ oder $O_x$ und $x^m$ verbleiben, und diese Operatoren kommutieren, kann man diesen letzten Operator ebenso mit seinem Inversen zu 1 zusammenfassen.

  \textbf{Zeitabhängige Erwartungswerte des Ortes}

  Mit den vorangegangenen Überlegungen lassen sich die Erwartungswerte des Ortsoperators $x$ berechnen:
  \begin{align}
    \braket{x^m}_n&=\braket{\Psi_n(t)|x^m|\Psi_n(t)}=\braket{n|O_x^{-1}O_p^{-1}x^mO_pO_x|n}=\braket{n|(x+\alpha(t))^m|n}\notag\\
    &=\braket{(x^m+\alpha(t))^m}_{n,\text{ung}}
    =\braket{(y+\zeta(t))^m}_{n,\text{ung}}
    %\braket{x^m}_n&=\braket{n|O_x^{-1}O_p^{-1}x^mO_pO_x|n}=\braket{n|(x+\alpha(t))^m|n}=\braket{n|x|n}+\alpha(t) =\braket{y}_{n,\text{ung}}+\zeta(t)=\zeta(t)
    %\braket{x^2}_n&=\braket{n|O_x^{-1}O_p^{-1}x^2O_pO_x|n}=\braket{n|x^2+2\alpha(t)x+\alpha^2(t)|n}
    %=\braket{n|x^2|n}+2\alpha(t)\braket{n|x|n}+\alpha^2(t)\notag\\
    %&=\braket{y^2}_{n,\text{ung}}+2\zeta(t)\braket{y}_{n,\text{ung}}+\zeta^2(t)=s^2(2n+1)+\zeta^2(t)
  \end{align}
  Dieses Ergebnis ist identisch zum entsprechenden Ergebnis (\ref{erwartungswert_x^m_einzelner}) aus Kapitel \ref{3}.

  \textbf{Zeitabhängige Erwartungswerte des Impulses}

  Für die Impuls-Erwartungswerte ergibt sich analog zu den Ort-Erwartungswerten
  \begin{align}
    \braket{p^m}_n&=\braket{\Psi_n(t)|p^m|\Psi_n(t)}=\braket{n|O_x^{-1}O_p^{-1}p^mO_pO_x|n}=\braket{n|(p-\varphi(t))^m|n}\notag\\
    &=\braket{(p_y^m-\varphi(t))^m}_{n,\text{ung}}
    =\braket{(p_y+m\dot\zeta(t))^m}_{n,\text{ung}} \;.
    %\braket{p}_n&=\braket{n|O_x^{-1}O_p^{-1}pO_pO_x|n}=\braket{n|(p-\varphi(t))|n}=\braket{n|p|n}-\varphi(t) =\braket{p_y}_{n,\text{ung}}+m\dot\varphi(t)=\dot\varphi(t)\\
    %\braket{p^2}_n&=\braket{n|O_x^{-1}O_p^{-1}p^2O_pO_x|n}=\braket{n|p^2-2\varphi(t)x+\varphi^2(t)|n}
    %=\braket{n|p^2|n}-2\varphi(t)\braket{n|p|n}+\dot\varphi^2(t)\notag\\
    %&=\braket{p_y^2}_{n,\text{ung}}+2m\dot\zeta(t)\braket{p_y}_{n,\text{ung}}+(m\zeta(t))^2=m^2\omega_0^2s^2(2n+1)+m^2\dot\zeta^2(t) \; .
  \end{align}
  Auch für den Impuls bekommen wir die gleichen Erwartungswerte wie bei der ersten Berechnung in Gleichung (\ref{erwartungswert_p^m_einzelner}) aus Kapitel \ref{3}.

  Schließlich erhalten wir für den Ort und den Impuls des einzelnen getriebenen Oszillators und folglich auch für die Energie in der Besetzungszahldarstellung die Erwartungswerte, wie wir sie schon in Kapitel \ref{3} in der Ortsdarstellung erhalten haben.
\iffalse
  Nach diesem Ergebnis kann davon ausgegangen werden, dass die Erwartungswerte für alle $x^m/p^m$ übereinstimmen, weshalb wir die Relationen für die Anwendung der $O_{x/p}$ aud $p,x$ aus (\ref{kommutator_1}-\ref{kommutator_2}) allgemein hinschreiben können, weil die Erwartungswerte in Ortsdarstellung für alle $m\in \mathbb{N}$ in (\ref{erwartungswert_x^m_einzelner}) und (\ref{erwartungswert_x^m_einzelner}) berechnet haben.
  Es folgt
  \begin{align}
    x^mO_p=O_P(x+\alpha(t))^m\sum_{j=0}^m \begin{pmatrix} m \\ j \\ \end{pmatrix} \braket{y^{m-j}}_{n,\text{ung}}\zeta^j(t) \; .
\fi
