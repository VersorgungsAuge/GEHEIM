\chapter{Zusammenfassung und Ausblick}
\label{6}
In dieser Arbeit wurde das quantenmechanische Problem eines periodisch getriebenen harmonischen Oszillators näher betrachtet.
Nachdem die Floquet-Theorie in Grundzügen erklärt wurde, konnte das Floquet-Theorem anhand der Wellenfunktionen des getriebenen Oszillators verifiziert werden.
Mit einem Fourier-Reihen-Ansatz konnten die Quasienergien der Floquet-Theorie für eine beliebige periodische Treibkraft im System, in Abhängigkeit der Fourier-Koeffizienten der Kraft bestimmt werden.
Mit diesen kann unter anderem das zeitliche Mittel des Energie-Erwartungswertes berechnet werden, ohne diesen selbst zu kennen, wodurch ein einfacher erster Eindruck des Systems möglich ist.
Wir konnten z.\,B. frühzeitig feststellen, dass die Energie divergieren wird, wenn sich die Treibfrequenz nah der Oszillator-Eigenfrequenz befindet.

Nachdem die Erwartungswerte für den Ort, den Impuls und die Energie berechnet wurden, haben wir das System um einen weiteren angekoppelten Oszillator erweitert.
Die Lösung der Schrödinger-Gleichung gelang hier mit einer Koordinatentransformation, welche den Hamilton-Operator in eine Summe aus zwei unabhängigen Hamilton-Operatoren überführte.
Damit wussten wir, dass die Lösung durch ein Produkt der Wellenfunktionen in den neuen Koordinaten gegeben ist, was für beliebig viele Hamilton-Operatoren ebenso korrekt ist.
Die Form der Koordinaten-Transformation kann hierbei aus den klassischen Normalmoden gewonnen werden, welche wiederum durch das Diagonalisieren einer Matrix bei der Lösung des klassichen Problems erhalten werden.
Dies sollte ebenso für beliebig viele gekoppelte Oszillatoren funktionieren, womit die Wellenfunktionen für ein System aus $n$ gekoppelten Oszilaltoren genauso aufgestellt werden können.
Für ungleiche Massen und Potentialkonstanten ergeben sich im Allgemeinen aber kompliziertere Normalmoden und Transformationen.

Für die zwei gekoppelten getriebenen Oszillatoren haben wir, unter Ausnutzung der neuen Variablen und der Erwartungswerte des einzelnen Oszillators, ebenfalls die Erwartungswerte berechnet und zudem visualisiert.
Es ergab sich an vielen Stellen ein Verhalten der Erwatungswerte wie es aus der klassichen Mechanik zu erwarten war.

Zum Schluss der Arbeit wurde ein weiteres Mal der einzelne getriebene Oszillator behandelt.
Diesmal wurde das Problem in der 2. Quantisierung betrachtet.
Mit einem geeigneten Ansatz, welchen wir anhand der Lösung in Ortsdarstellung motiviert haben, haben wir das darstellungsunabhängige Wellenfuktion-Ket aufstellen können.
In der 2. Quantisierung haben wir damit die gleichen Erwartungswerte wie in der Ortsdarstellung erhalten.
Um mehrere gekoppelte Oszillatoren in der Besetzungszahldarstellung zu betrachten, kann die gleiche Transformation für die Operatoren gewählt werden wie zuvor in der Ortsdarstellung.
Weil die Operatoren beim Wellenfunktion-Ket nur im Exponenten vorkommen, führt dies sofort auf die Produktform.
%Der Pduktansatz funktioniert für beliebig viele unabhängige Operatoren.
%Die Koordinatentranformation
