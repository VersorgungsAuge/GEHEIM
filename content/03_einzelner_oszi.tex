\chapter{Getriebener harmonischer Oszillator in der Quantenmechanik}
  Der Hamilton-Operator eines harmonischen Oszillators der Masse $m$, welcher mit einer beliebigen aber periodischen äußeren Kraft $S(t)=S(t+T)$ getrieben wird, hat die Form
  \begin{equation}
    H(t) = H(t+T) = \frac{p^2}{2m} + \frac{1}{2}m\omega_0^2x^2-S(t)x \; ,
  \end{equation}
  mit $\omega_0=\sqrt{k/m}$ und der Potentialkonstante $k$.
  Im Folgenden wird vorerst die Schr"odinger-Gleichung f"ur eine beliebige Treibkraft $S(t)$ gel"ost.
  Weiterhin werden die Floquet-Moden $\Phi_n(x,t)$ identifiziert, genauso wie die Quasienergien $\epsilon_n$, welche f"ur eine beliebige periodische Treibkraft und explizit f"ur eine sinusf"ormige Treibkraft bestimmt werden.

  Danach werden wir die Ewartungswerte $\braket{x}_n,\braket{x^2}_n,\braket{p}_n,\braket{p^2}_n$ und damit die Unsch"arfe berechnen, indem wir die bekannten Erwartungswerte des ungetriebenen Oszillators benutzen.
  Ebenso werden wir den zeitabh"angigen und gemittelten Erwartungswert der Energie $\braket{H}_n$ und $\bar H_n$ berechnen.



\section{Allgemeine Lösung der Schrödinger-Gleichung}
  \label{lsg_einzelner}
  Dieses System kann exakt gelöst werden, indem die Schrödinger-Gleichung durch einen Variablenwechsel und zwei unitäre Transformationen auf die bekannte Form des ungetriebenen Oszillators reduziert wird \cite{haenggi}.

  \textbf{1) Unit"arer Variablenwechsel}\\
  Für den neuen Ortsoperator bzw. die Ortsvariable wird eine zeitabhängige Verschiebung angesetzt:
  \begin{equation}
    x \rightarrow y=x-\zeta(t) \; ;
  \end{equation}
  Wie zu erwarten verändert sich der Impuls(operator) durch die Translation im Ort nicht, da $\zeta(t)$ bei der Ortsableitung wegfällt.
  %wobei der Impuls(operator) unverändert bleibt.
  %Der Impulsoperator bleibt dadurch unverändert.

  Mit der neuen Zeitableitung der Wellenfunktion
  \begin{equation}
    \text{i}\hbar \frac{\partial}{\partial t} \Psi(y(t),t) = \text{i}\hbar \dot{\Psi} -\dot{\zeta}\frac{\partial}{\partial y}\Psi(y(t),t)
  \end{equation}
  wird die Schrödinger-Gleichung zu:
  \begin{equation}
    \text i \hbar \dot{\Psi}(y,t) = \left[\text i \hbar \dot{\zeta}\frac{\partial}{\partial y}-\frac{\hbar^2}{2m}\frac{\partial^2}{\partial y^2}+\frac{1}{2}m\omega_0^2(y+\zeta)^2-(y+\zeta)S(t)\right]\Psi(y,t) \; .
    \label{schroedinger_einzeln_getrieben}
  \end{equation}

  \textbf{2) Unitäre Tranformation für $\Psi(y,t)$}\\
  Im Weiteren wählen wir die unitäre Transformation
  \begin{equation}
    \Psi(y,t) = \text e^{\frac{\text i}{\hbar}m\dot \zeta y}\Lambda(y,t) \; .
  \end{equation}
  Durch Einsetzen in die Schrödinger-Gleichung (\ref{schroedinger_einzeln_getrieben}) und Ausrechnen der Ableitungen erhalten wir
  \begin{align}
    \begin{split}
      &\text e^{\frac{\text i}{\hbar}m\dot \zeta y}(\text i \hbar \dot \Lambda(y,t) - my \ddot{\zeta}\Lambda(y,t)) =\\
       &\text e^{\frac{\text i}{\hbar}m\dot \zeta y} \left[\left(-m\dot \zeta^2 \Lambda(y,t) + \text i \hbar \dot \zeta \frac{\partial}{\partial y} \Lambda(y,t) \right) \right. \\
       &\left. + \left(\frac{1}{2}m\dot \zeta^2 \Lambda(y,t) - \text{i} \hbar \dot \zeta \frac{\partial}{\partial y} \Lambda(y,t) - \frac{\hbar}{2m}\frac{\partial^2}{\partial y^2} \Lambda(y,t)  \right)
      + \frac{1}{2}m\omega_0^2(y+\zeta)^2  + (y+\zeta)S(t)  \right] \; .
    \end{split}
  \end{align}
  \iffalse
  Durch Einsetzen in die Schrödinger-Gleichung und Ausrechnen der Ableitungen erhalten wir für die linke Seite von (\ref{schroedinger_einzeln_getrieben})
  \text e^{\frac{\text i}{\hbar}m\dot \zeta y}(\text i \hbar \dot \Lambda(y,t) - my \ddot{\zeta}\Lambda(y,t))
  \begin{equation}
  \end{equation}
  und für die rechte Seite
  \begin{align}
    \begin{split}
    \text e^{\frac{\text i}{\hbar}m\dot \zeta y} \left[\left(-m\dot \zeta^2 \Lambda(y,t) + \text i \hbar \dot \zeta \frac{\partial}{\partial y} \Lambda(y,t) \right) + \left(\frac{1}{2}m\dot \zeta^2 \Lambda(y,t) - \text{i} \hbar \dot \zeta \frac{\partial}{\partial y} \Lambda(y,t) - \frac{\hbar}{2m}\frac{\partial^2}{\partial y^2} \Lambda(y,t)  \right) \right. \\
    \left. + \left(\frac{1}{2}m\omega_0^2y^2\Lambda(y,t) + m\omega_0^2y\zeta\Lambda(y,t) + \frac{1}{2}m\omega_0^2 \zeta^2\Lambda(y,t)) \right) + \left(-yS(t)\Lambda(y,t) - \zeta S(t)\Lambda(y,t) \right) \right]
  \end{split}
  \end{align}
  \fi

  %\newpage

  Indem auf beiden Seiten durch die Exponentialfunktion geteilt wird, bekommen wir eine Differentialgleichung für $\Lambda(y,t)$.
  Außerdem können durch geschicktes Umsortieren der Terme die Lagrange-Funktion $L(\zeta,\dot \zeta, t)$
  \begin{equation}
    L(\zeta,\dot \zeta, t) = \frac{1}{2}m\dot \zeta^2 - \frac{1}{2}m\omega_0^2\zeta^2 + S(t)\zeta
    \label{lagrange_zeta}
  \end{equation}
    sowie die Bewegungsgleichung des klassischen getriebenen harmonischen  Oszillators \cite{husimi}
    \begin{equation}
      m\ddot \zeta + m\omega_0^2\zeta - S(t) = 0
      \label{dgl_zeta}
    \end{equation}
  für die Verschiebung $\zeta(t)$ identifiziert werden.
  Die Differentialgleichung für $\Lambda(y,t)$ ist folglich
  \begin{equation}
    \text i \hbar \dot \Lambda(y,t) = \left[ \frac{-\hbar^2}{2m}\frac{\partial^2}{\partial y^2} + \frac{1}{2}m\omega_0^2y^2 + (m\ddot \zeta + m\omega_0^2y\zeta - S(t))y - L(\zeta,\dot \zeta, t) \right]\Lambda(y,t) \; .
    \label{dgl_lambda}
  \end{equation}
  Um die Gleichung zu vereinfachen, wählen wir $\zeta(t)$ nun so, dass es gerade die klassische Bewegungsgleichung erfüllt, der entsprechende Term in (\ref{dgl_lambda}) also verschwindet.
  Nur noch die Lagrange-Funktion unterscheidet diese Differential-Gleichung von der des ungetriebenen Oszillators.

  \textbf{3) Unitäre Transformation für $\Lambda(y,t)$}\\
  Zuletzt wählen wir den Ansatz
  \begin{equation}
    \Lambda(y,t) = \text e^{\frac{\text i}{\hbar}\int_0^tL \: \text d t'}\chi(y,t)
  \end{equation}
  für $\Lambda(y,t)$, um die Lagrange-Funktion in (\ref{dgl_lambda}) zu eliminieren.
  Dadurch wird diese Differential-Gleichung für $\Lambda(y,t)$ bzw. die ursprüngliche Schrödinger-Gleichung auf einen ungetriebenen Oszillator für $\chi(y,t)$ zurückgeführt:
  \begin{equation}
    \text i \hbar \dot \chi(y,t) = \left[ \frac{-\hbar^2}{2m}\frac{\partial^2}{\partial y^2} + \frac{1}{2}m\omega_0^2y^2 \right]\chi(y,t) \; .
  \end{equation}
  Das bedeutet die $\chi_n(y,t)$ sind die Wellenfunktionen des ungetriebenen Oszillators und die Gesamtlösung der Schrödinger-Gleichung des getriebenen Oszillators ist damit gegeben durch
  \begin{align}
    \begin{split}
    \Psi_n(x,t) &= \Psi_n(y=x-\zeta(t),t) \\
    &= N_nO_n\left(\sqrt{\frac{m\omega_0}{\hbar}}(x-\zeta(t))\right) \text e^{\frac{-m\omega_0}{2\hbar}(x-\zeta(t))^2} \\
    &\quad \cdot \text e^{\frac{\text i}{\hbar}\left(m\dot \zeta(t)(x-\zeta(t))-E_nt+\int_0^tL(\dot \zeta,\zeta,t')\:\text dt'\right)} \; ,
    \; n \in \mathbb{N}_0 \; .
    \label{gesamtlsg_einzelner}
  \end{split}
  \end{align}
  Dabei sind $O_n$ die Hermit-Polynome, $E_n = \hbar \omega_0(n+1/2)$ die bekannten Eigenenergien der Zustände des ungetriebenen Oszillators und
  \begin{equation}
    N_n = \left(\frac{m\omega_0}{\pi \hbar}\right) \frac{1}{\sqrt{2^nn!}}
  \end{equation}
  dessen Normierungsfaktoren.
  Die Lösungen des getriebenen Oszillators sind im Raum normiert und somit ebenfalls bezüglich des Skalarproduktes (\ref{skalarprodukt_einzelner}).

  Die Lösung entspricht damit einem, um die klassiche Lösung $\zeta(t)$ verschobenen, ungetriebenen Oszillator, mit einer zusätzlichen zeit- und ortsabhängigen komplexen Phase.
  Die treibende Kraft $S(t)$ geht in die klassiche Lösung $\zeta(t)$ und direkt, über das Wirkungsintegral, in die komplexe Phase ein.



%war vorher subsection
  \section{\texorpdfstring{Identifizierung der Quasienergien $\epsilon_n$ und Floquet-Moden $\Phi_n(x,t)$}{Identifizierung der Quasienergien epsilon_n und Floquet-Moden Phi_n(x,t)}}
    Nach dem Floquet-Theorem für periodische Hamilton-Operatoren (\ref{floquet_theorem}) kann die Lösung des getriebenen Oszillators geschrieben werden als
    \begin{equation}
      \Psi_n(x,t) = \text e^{\frac{-\text i}{\hbar}\epsilon_nt}\Phi_n(x,t) \; ,
    \end{equation}
    mit $\Phi_n(x,t)=\Phi_n(x,t+T)$.
    Nun können wir alle $T$-periodischen Terme als $\Phi_n(x,t)$ identifizieren, und alle Terme im Exponenten, die linear in $t$ sind, als $-\text i\epsilon_n t/\hbar$ \cite{haenggi}.

    Alle Funktionen von $(x-\zeta(t))$ haben die Periode $T$, da $\zeta(T)$ als Lösung der klassichen Bewegungsgleichung mit $S(t)=S(t+T)$ die Periode $T$ hat. Das
    Ergebnis des Integrals über die Lagrange-Funktion kann nur $T$-periodisch oder linear in $t$ sein, deshalb sind die Quasienergien gegeben durch die $E_n$ und den linearen Teil des Integrals:
    \begin{equation}
      \epsilon_n = E_n - \frac{1}{T} \int_0^T L(\dot \zeta, \zeta, t) \: \text d t \; .
    \end{equation}
    Die Floquet-Moden sind demnach
    \begin{align}
      \begin{split}
        \Phi_n(x,t) &=
         N_nO_n\left(\sqrt{\frac{m\omega_0}{\hbar}}(x-\zeta(t))\right) \text e^{\frac{-m\omega_0}{2\hbar}(x-\zeta(t))^2} \\
        &\quad \cdot \text e^{\frac{\text i}{\hbar}\left(m\dot \zeta(t)(x-\zeta(t))+\int_0^tL(\dot \zeta,\zeta,t')\:\text d t-\frac{t}{T} \int_0^T L(\dot \zeta,\zeta,t)\:\text dt\: \right)} \; , \\
        &\quad \quad n \in \mathbb{N}_0 \; .
        \label{floquet-moden_einzelner}
      \end{split}
    \end{align}


  \subsection{Quasienergien für eine sinusoidiale Treibkraft}
    Hier wird ein Beispiel einer treibenden Kraft diskutiert \cite{haenggi}:
    \begin{equation}
      S(t) = S(t+T) = A\sin(\omega t),
    \end{equation}
    wobei $T=2\pi / \omega$ ist.
    Setzen wir die allgemeine homogene Lösung gleich null, wird die Lösung $\zeta(t)$ der klassischen Bewegungsgleichung (\ref{dgl_zeta}) zu \cite{mads}
    \begin{equation}
      \zeta(t) = \frac{A\sin(\omega t)}{m(\omega_0^2 - \omega^2)} \; .
    \end{equation}
    Das Berechnen des Wirkungsintegrals und anschließendes Identifizieren des linearen Anteils liefert die Quasienergien für die gegebene Kraft:
    \begin{equation}
      \epsilon_n  = \hbar \omega_0\left(n+\frac{1}{2}\right) - \frac{A}{4m(\omega_0^2-\omega^2)} \;.
    \end{equation}
    Wie zu erkennen streben die Quasienergien, und somit die mittlere Energie des Systems (\ref{mittleres_H}), gegen Unendlich, wenn sich die Treibfrequenz $\omega$ nahe der Eigenfrequenz des Oszillators $\omega_0 $ befindet.
    Da wir einen getriebenen Oszillator ohne Dämpfung betrachten, war dieses Ergebnis zu erwarten \cite{mads}.
    HIER KANN MAN NOCH AUSFUEHRLICH INSCHREIBEN MIT MADS

%\chapter{Ergebnisse}

%war vorher section
\subsection{Quasienergien für eine beliebige periodische Treibkraft $S(t)$}
  Um die Quasienergien bei einer beliebigen periodisch treibenden Kraft auf elegante Weise zu bestimmen, und ohne ein Wirkungsintegral berechnen zu müssen, setzen wir eine komplexe Fourier-Reihe an:
  \begin{equation}
    S(t) = \sum_{j=-\infty}^\infty c_j \text e^{\text ij\omega t}, \; c_j = \frac{1}{T} \int_0^T S(t) \text e^{-ij\omega t} \: \text d t \; .
  \end{equation}
  Die $c_j$ haben die Einheit einer Kraft.

  Für $\zeta(t)$ wählen wir jetzt ebenfalls einen Reihenansatz:
  \begin{equation}
    \zeta(t) = \sum_{j=-\infty}^\infty d_j \text e^{\text ij\omega t} \; .
  \end{equation}
  Wir betrachten wieder nur die inhomogene klassische Bewegungsgleichung und bestimmen die $d_j$, indem wir einsetzen und einen Koeffizientenvergleich der $d_j$ und $c_j$ machen.
  Es zeigt sich, dass
  \begin{equation}
    \zeta(t) = \sum_{j=-\infty}^\infty \frac{c_j}{m(-j^2\omega^2+\omega_0^2)} \text e^{ij\omega t}
  \end{equation}
  gilt, womit sich die Lagrange-Funktion (\ref{lagrange_zeta}) ergibt zu:
  \begin{align}
    \begin{split}
      L &= \frac{1}{2}m\dot \zeta^2 - \frac{1}{2}m\omega_0^2\zeta^2 + S(t)\zeta \notag\\
       &=\sum_j \sum_l \left[ \frac{-\omega^2}{2m} \frac{jc_j}{-j^2\omega^2+\omega_0^2} \frac{lc_l}{-l^2\omega^2+\omega_0^2}
       -\frac{\omega_0^2}{2m} \frac{c_j}{-j^2\omega^2+\omega_0^2} \frac{c_l}{-l^2\omega^2+\omega_0^2} \right. \notag\\
        &\left. \quad + \frac{1}{m} \frac{c_jc_l}{-j^2\omega^2+\omega_0^2} \right] \text e^{\text i(j+l)\omega t}  \; , \; j,l \in \mathbb{Z} \; .
     \end{split}
   \end{align}
  \iffalse
  \begin{align}
    \begin{split}
      L &= \frac{1}{2}m\dot \zeta^2 - \frac{1}{2}m\omega_0^2\zeta^2 + S(t)\zeta \\
       &=\frac{-\omega^2}{2m} \sum_j \sum_l \frac{jc_j}{-j^2\omega^2+\omega_0^2} \frac{lc_l}{-l^2\omega^2+\omega_0^2} \text e^{\text i(j+l)\omega t}\\
       &\quad-\frac{\omega_0^2}{2m} \sum_j \sum_l \frac{c_j}{-j^2\omega^2+\omega_0^2} \frac{c_l}{-l^2\omega^2+\omega_0^2} \text e^{\text i(j+l)\omega t}\\
       &\quad + \frac{1}{m} \sum_j \sum_l \frac{c_jc_l}{-j^2\omega^2+\omega_0^2} \text e^{\text i(j+l)\omega t}\; , \; j,l \in \mathbb{Z} \; .
     \end{split}
   \end{align}
   \fi

   Nun identifizieren wir alle Terme der Lagrange-Funktion, welche nach der Ausführung des Wirkungsintegrals linear in der Zeit $t$ sind, ohne dieses explizit zu berechnen.
   Denn wir wissen, dass die Exponentialterme, welche sich beim Integrieren nach der Zeit reproduzieren, periodisch in der Zeit sind.
   Daher werden nur die konstanten Terme, welche entstehen wenn die Exponentialterme wegfallen, eine lineare Abhängigkeit aufweisen.
   Die Exponentialterme werden 1 wenn $j=-l$ ist.
   Infolgedessen wird die Doppelsumme, die bei der Quadrierung entstanden ist, wieder zur Einzelsumme.
   Wir fassen den linearen Teil des Wirkungsintegrals zusammen:
   \begin{align}
     \begin{split}
       \frac{1}{T} \int_0^T L \: \text d t
       &= \sum_j \left[\frac{\omega^2}{2m} \frac{j^2c_jc_{-j}}{(-j^2\omega^2+\omega_0^2)^2}
       - \frac{\omega_o^2}{2m} \frac{c_jc_{-j}}{(-j^2\omega^2+\omega_0^2)^2} \right.
       \left. +\frac{1}{m} \frac{c_jc_{-j}}{-j^2\omega^2+\omega_0^2}
       \right]\\
       &= \sum_j \left[\frac{1}{2m} \frac{c_jc_{-j}(j^2\omega^2-\omega_0^2)}{(-j^2\omega^2+\omega_0^2)^2} + \frac{1}{m} \frac{c_jc_{-j}}{-j^2\omega^2+\omega_0^2} \right] \\
       &= \sum_j \frac{c_jc_{-j}}{2m(-j^2\omega^2+\omega_0^2)} = \frac{c_0^2}{2m\omega_0^2} + \sum_{j=1}^\infty \frac{c_j^2}{m(-j^2\omega^2+\omega_0^2)}
     \end{split}
   \end{align}

   Die Quasienergien $\epsilon_n$ sind hiernach:
   \begin{align}
     \begin{split}
       \epsilon_n &= \hbar \omega_0\left(n+\frac{1}{2}\right) - \sum_j \frac{c_jc_{-j}}{2m(-j^2\omega^2+\omega_0^2)} \; .
     \end{split}
   \end{align}
   Interessanterweise kommt es in obiger Formel nicht nur bei $\omega = \omega_0$ zu einer Singularität, wie bei der Beispielkraft $S(t) = A\sin(\omega t)$, sondern bei allen $\omega = \omega_0 / j$.



   \subsection{Beispiele für S(t) dreieck rechteck delta}

\section{Berechnung von Erwartungswerten}
  In diesem Abschnitt werden die Erwartungswerte des periodisch getriebenen Oszillators f"ur den Impuls, den Ort und die Energie berechnet.
  Die Treibkraft $S(t)=S(t+T)$ ist nicht weiter festgelegt und geht stets "uber die klassiche L"osung $\zeta(t)$ (\ref{dgl_zeta}) in die Erwartungswerte ein.

  Um die Erwartungswerte des getriebenen Oszillators zu erhalten, werden wir dessen Berechnnung auf die Berechnung der bekannten Erwartungswerte des ungetriebenen Oszillators zur"uckf"uhren.
  Daf"ur nutzen wir aus, dass die Wellenfunktionen und die Floquet-Moden nach (\ref{gesamtlsg_einzelner}) und (\ref{floquet-moden-einzelner}) einem um $\zeta(t)$ verschobenen ungetriebnen Oszillator mit einer zus"atzlichen komplexen Phase entsprechen.
  Da eine komplexe Phase beim Bilden des Betrages wegf"allt, gilt
  \begin{equation}
    \Psi_n^*(x,t)\Psi_n(x,t) = |\Psi_n(x,t)|^2 = |\Phi_n(x,t)|^2 = |\Psi_{n,ung}(x-\zeta(t),t)|^2 = |\Psi_{n,ung}(y,t)|^2 \; ,
    \label{betrag_einzelner}
  \end{equation}
  wobei $\Psi_{n,ung}(y,t)$ die Zust"ande des ungetriebenen Oszillators bezeichnen.

  Bei dem Integral von $-\infty$ bis $\infty$ ist eine konstante Verschiebung der Variable irrelevant.
  Au"serdem sind die $\Psi_{n,ung}(y,t)$ auf dem Raum normiert, daher sind auch die $\Psi_n(x,t)$ auf $x \in (-\infty,\infty)$ normiert, wie bereits am Ende von Kapitel \ref{lsg_einzelner} erw"ahnt.

  Eine Visualisierung von Erwartungswerten folgt in Kapitel 4, weil es sich bei den Erwartungswerten der zwei gekoppelten Oszillatoren genau um die Summe aus den Erwartungswerten von zwei gesonderten Oszillatoren handelt, wie sich dort zeigen wird.

  \subsection{Erwartungswerte f"ur den Ort}
  INTERPRETATION ; SCHWINGT KLASSICH USW
    Zuerst betrachten wir den Erwartungswert $\braket{x}_n$ des Ortsoperators $x$ f"ur den $n$-ten Zustand $\Psi_n(x,t)$ des getriebenen Oszillators, welcher per Definition gegeben ist durch
    \begin{equation}
      \braket{x}_n = \int_{-\infty}^{\infty} \Psi_n^*(x,t)x\Psi_n(x,t) \: \text dx
      = \int_{-\infty}^{\infty} |\Psi_n(x,t)|^2 x \: \text dx \; .
    \end{equation}
    Mit der Substitution $y=x-\zeta(t)$ und unter Verwendung von (\ref{betrag_einzelner}) und der Normierung der ungetriebenen Oszillator-Funktionen gelangen wir zu
    \begin{align}
      \begin{split}
        \braket{x}_n &= \int_{-\infty}^{\infty} |\Psi_{n,ung}(x-\zeta(t),t)|^2 x \: \text dx
        = \int_{-\infty}^{\infty} |\Psi_{n,ung}(y,t)|^2 (y+\zeta(t)) \: \text dy \\
        &= \braket{y}_{n,ung} + \zeta(t) = \zeta(t) \; .
      \end{split}
    \end{align}
    Da die Erwartungswerte $\braket{y}_{n,ung}$ f"ur den Ort des ungetriebenen Oszillators verschwinden, ist der Erwartungswert jedes Zust"andes des getriebenen Oszillators genau $\zeta(t)$.

    Das bedeutet, der Erwartungswert $\braket{x}_n$ f"ur den Ort eines beliebigen Zustandes des quantenmechanischen getriebenen Oszillators entspricht genau der L"osung $\zeta(t)$ der Bewegungsgleichung des klassischen getriebenen Oszillators.
    Der Erwartungswert des ungetriebenen Oszillators, welcher im Ursprung liegt, ist somit um $\zeta(t)$ verschoben.
    Dies entspricht dem Ergebnis, wie es auch bei einer konstanten Verschiebung der Ortsvariable/-operator $x$ erhalten wird, z.B. beim Oszillator im zus"atzlichen konstanten elektrischem Feld, nur das die Verschiebung hier die zeitabh"angige klassiche L"osung ist.

    Da $\braket{x}_n=\zeta(t)$ logischerweise die klassiche Bewegungsgleichung
    \begin{equation}
      m\ddot{\braket{x}}_n + m\omega_0^2\braket{x}_n - S(t) = 0
    \end{equation}
    erf"ullt, ist erg"anzend das Ehrenfest-Theorem best"atigt.

    F"ur den Erwartungswert von $x^2$ ergibt sich komplett analog
    \begin{align}
      \braket{x^2}_n &= \int_{-\infty}^{\infty} |\Psi_n(x,t)|^2 x^2 \: \text dx
      = \int_{-\infty}^{\infty} |\Psi_{n,ung}(y,t)|^2 (y+\zeta(t))^2 \: \text dy \notag \\
      &= \braket{y^2}_{n,ung} + 2\braket{y}_{n,ung} + \zeta^2(t)
      = l^2(2n+1) + \zeta^2(t) \; ,
    \end{align}
    zusammen mit der charakteristischen L"ange des ungetriebenen Oszillators
    \begin{equation}
      l = \sqrt{\frac{\hbar}{2m\omega_0}} \; .
    \end{equation}
    Der Erwartungswert ist nun um $\zeta^2(t)$ verschoben.

    Insgesamt k"onnen alle Erwartungswerte $\braket{x^m}_n$ auf diese Art mit den bekannten ungetriebenen Erwartungswerten berechnet werden.
    Ab $\braket{x^3}_n$ kommt es wegen $(y+\zeta(t))^3$ aber im Allgemeinen nicht nur zu Verschiebungen bzw. zus"atlichen $\zeta(t)$-Termen, verglichen zu den ungetriebenen Erwartungswerten, mit dem selben Exponenten, weshalb wir den Ewartungswert $\braket{x^m}_n$ nicht einfacher schreiben k"onnen als den linearen Erwartungswert des Klammerausdruckes mit dem binomischen Lehrsatz \cite{binom} zu vereinfachen:
    \begin{equation}
      \braket{x^m}_n = \braket{(y+\zeta(t)^)m}_{n,ung} = \sum_{j=0}^m \begin{pmatrix} m \\ j \\ \end{pmatrix} \zeta^j(t)\braket{y^{m-j}}_{n,ung} \; .
    \end{equation}





  \subsection{Erwartungswerte f"ur den Impuls}
    Um den Erwartungswert $\braket{p}_n$ des Impulsoperators f"ur den Zustand $\Psi_n(x,t)$ des angetriebenen Oszillators zu ermitteln, nutzen wir gleicherma"sen die Beziehung (\ref{betrag_einzelner}) und unterdessen ein geschicktes Anwenden der Produktregel, um das Problem erneut auf den ungetriebenen Oszilaltor zu beschr"anken.

    Wieder w"ahlen wir den Variablenwechsel zur neuen Variable $y=x-\zeta(t)$, weil Impulsoperator $p=p_x$ in der neuen Koordinate identisch ist, wie schon am Anfang von kapitel \ref{lsg_einzelner} angedeutet, was sich leicht mit der Kettenregel zeigen l"asst:
    \begin{equation}
      -\text i\hbar \frac{\partial}{\partial x} = -\text i\hbar \frac{\partial}{\partial y}\frac{\partial y}{\partial x} = -\text i\hbar \frac{\partial}{\partial y} \cdot 1 \quad \text{d.h.} \quad p_x \equiv p_y \; .
    \end{equation}
    Weiterhin k"onnen wir den $y$-unabh"angigen Teil der komplexen Phasen in den Wellenfunktionen (\ref{gesamtlsg_einzelner}) sofort am Operator vorbeiziehen und zu 1 zusmmenfassen, daher schreiben wir $\braket{p}_n$ als
    \begin{align}
      \braket{p}_n &= \int_{-\infty}^{\infty} \Psi^*_n(x,t) p \Psi_n(x,t) \; \text dx
      = \int_{-\infty}^{\infty} \Psi^*_n(y+\zeta(t),t) p_y \Psi_n(y+\zeta(t),t) \: \text dy \notag\\
      &= \int_{-\infty}^{\infty} N_nO_n(y)\text e^{\frac{-m\omega_0}{2\hbar}y^2}\text e^{\frac{-\text i}{\hbar}m\dot{\zeta}(t)y} p_y N_nO_n(y)\text e^{\frac{-m\omega_0}{2\hbar}y^2}\text e^{\frac{+\text i}{\hbar}m\dot{\zeta}(t)y} \: \text dy \; .
      \label{spaste}
    \end{align}
    Nun spalten wir die Anwendung des Impulsoperators auf den rechten Teil durch Anwenden der Produktregel auf in die Anwendung auf die ungetriebene Oszillatorfunktion (in der neuen Koordinate $y$) und die Anwendung auf die verbleibende komplexe Phase:
    \begin{align}
      &p_y N_nO_n(y)\text e^{\frac{-m\omega_0}{2\hbar}y^2}\text e^{\frac{+\text i}{\hbar}m\dot{\zeta}(t)y} = \notag\\
      &\left(p_y N_nO_n(y)\text e^{\frac{-m\omega_0}{2\hbar}y^2}\right) \text e^{\frac{+\text i}{\hbar}m\dot{\zeta}(t)y}
      + N_nO_n(y)\text e^{\frac{-m\omega_0}{2\hbar}y^2} \left(p_y \text e^{\frac{+\text i}{\hbar}m\dot{\zeta}(t)y}\right) \notag\\
      &= \text e^{\frac{+\text i}{\hbar}m\dot{\zeta}(t)} \left[p_yN_nO_n(y)\text e^{\frac{-m\omega_0}{2\hbar}y^2} + N_nO_n(y)\text e^{\frac{m\omega_0}{2\hbar}y^2}m\dot{\zeta}(t) \right] \; .
      \label{spaste2}
    \end{align}
    Hiernach haben wir effektiv erreicht, dass wir den Impulsoperator an der $y$-abh"angigen Phase vorbei ziehen konnten.
    Damit ist es m"oglich die verbliebenen Phasen in (\ref{spaste}) ebenso zu 1 zu vereinfachen.
    Es verbleiben nun nur noch im ersten Teil der eckigen Klammer die Anwendung des Impulsoperators auf die Wellenfunktionen des ungetriebenen Oszillators und im zweiten Teil die Multiplikation dieser mit einem ortsunabh"angigen Term.
    Zusammen mit dem "ubrigen linken Teil aus (\ref{spaste2}), welcher nach dem Wegfallen der komplexen Phase jetzt auch der ungetriebenen Wellenfunktion entspricht, ergeben sich dadurch die Erwartungswerte f"ur $p_y$ und den Term $m\dot{\zeta}(t)$, welcher nur zeitabh"angig ist.
    %Zusammen mit dem, nun identischen, "ubrigen komplex konjugiertem Teil aus (\ref{spaste2}) verbleiben nur noch mit im ersten Teil der der eckigen Klammer die Anwendung des Impulsoperators auf die Wellenfunktion des ungetriebenen Oszillators bzw

    Schlie"slich k"onnen wir den Erwartungswert f"ur den Impuls in Abh"angigkeit der bekannten Erwartungswerte des Standart-Oszillators hinschreiben:
    \begin{align}
      \braket{p}_n = \braket{p_y}_{n,ung} + m\dot{\zeta}(t) = m\dot{\zeta}(t)
    \end{align}

    Wieder erf"ullt der Erwartungswert die Bewegungsgleichung des klassichen Oszillators:
    \begin{equation}
      \ddot{\braket{p}}_n + \omega_0^2\braket{p}_n - \dot{S}(t) = 0 \; .
    \end{equation}

    Wie an (\ref{spaste2}) zu sehen, ist es dar"uber hinaus einfach den Erwartungswert $\braket{p^m}_n$ f"ur einen beliebigen Exponenten $m \in \mathbb{N}$ zu evaluieren.
    Wir k"onnen auf gleichem Wege die Produktregel anwenden, und erhalten wieder den ungetrieben Erwartungswert $\braket{p^m_y}$ aber diesmal zus"atzlich die $m$-Fache Anwendung des Operators auf die komplexe Phase, was wir sofort verallgemeinert aufschreiben k"onnen, da sich die komplexe Phase beim Ableiten nur reproduziert, weil immer der ortsunabh"angige Term $m\dot{\zeta(t)}$ "uber bleibt.
    Es folgt:
    \begin{equation}
      \braket{p^m}_n = \braket{p^m_y}_{n,ung} + (m\dot{\zeta}(t))^m \; .
    \end{equation}
    Speziell f"ur $m=2$ gilt
    \begin{equation}
      \braket{p^2}_n = \braket{p^2_y}_{n,ung} + (m\dot{\zeta}(t))^2 = m^2\omega^2_0 l^2(2n+1) + m^2\dot{\zeta^2}(t) \; .
    \end{equation}


  \subsection{Heisenberg'sche Unsch"arferelation}
    Mit den zuvor errechneten Erwartungswerten $\braket{x}_n,\braket{x^2}_n,\braket{p}_n,\braket{p^2}_n$ wird hier die Unsch"arfe des periodisch getriebenen Oszillators aufgestellt.
    Diese ist
    \begin{align}
      \Delta x_n\Delta p_n &= \sqrt{\braket{x^2}_n-\braket{x}^2_n}\sqrt{\braket{p^2}_n-\braket{p}^2_n} \notag\\
      &= \sqrt{l^2(2n+1)+\zeta^2(t)-\zeta^2(t)}\sqrt{m^2\omega^2_0l^2(2n+1)+m^2\zeta^2(t)-m^2\zeta^2(t)} \notag\\
      &= \frac{\hbar}{2}(2n+1) \; .
    \end{align}
    Die Unsch"arfe ist also gleich der des ungetriebenen Oszillators, welche nur f"ur den Grundzustand $n=0$ minimal ist.
    
    Dieses Ergebnis macht Sinn in Anbetracht der Tasache, dass die (zeitabh"angige) Verschiebung der Wellenfunktionen im Raum $\zeta(t)$ (\ref{gesamtlsg_einzelner}) die Unsch"arfe nicht "andern sollte.
    Die Wellenfunktionen haben allerdings noch den weiteren Unterschied der orts- und zeitabh"angigen komplexen Phase, verglichen mit den Wellenfunktionen des Standart-Oszillators.
    Diese spielt zwar keine  Rolle bei den Erwartungswerten f"ur den Ort, bei den Impulserwartungswerten aber schon.
    Dennoch war zu erwarten, dass sich auch beim Impuls eine Standartebweichung $\Delta p_n$ wie beim ungetriebenen OSzillator ergibt, so wie bei $\Delta x_n$, da $\braket{p}_n$ im Wesentlichen die zeitliche Ableitung von $\braket{x}_n$ ist.
    Demnach folgte die gleiche Unsch"arfe.

    MEHR SCHREIEN ????????


  \subsection{Erwartungswert f"ur die Energie}
















%\section{\texorpdfstring{Zeitlich gemittelter Erwartungswert der Energie $\bar{H}_n$}{Zeitlich gemittelter Erwartungswert der Energie bar{H}_n}}




































   b
