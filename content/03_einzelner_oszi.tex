

\chapter{Berechnung von Erwartungswerten}
\label{3}
  In diesem Abschnitt werden die zeitabhängigen Erwartungswerte des periodisch getriebenen Oszillators für den Impuls, den Ort und die Energie, sowie das zeitliche Mittel des Erwartungswertes der Energie berechnet.
  Dabei werden die Erwartungswerte für alle Zustände $\Psi_n(x,t)$ auf die bekannten Erwartungswerte des Standard-Oszillators zurückgeführt, womit wir eine Art "2. Quantisierung für die Erwartungswerte" haben.

  Die Treibkraft $S(t)=S(t+T)$ ist nicht weiter festgelegt und geht stets über die klassische Lösung $\zeta(t)$ (\ref{dgl_zeta}) in die Erwartungswerte ein.

  Um die Erwartungswerte des getriebenen Oszillators zu erhalten, werden wir dessen Berechnnung auf die Berechnung der bekannten Erwartungswerte des ungetriebenen Oszillators zurückführen.
  Dafür nutzen wir aus, dass die Wellenfunktionen und die Floquet-Moden nach (\ref{gesamtlsg_einzelner}) und (\ref{floquet_moden_einzelner}) einem um $\zeta(t)$ verschobenen ungetriebenen Oszillator mit einer zusätzlichen komplexen Phase entsprechen.
  Da eine komplexe Phase beim Bilden des Betrages wegfällt, gilt
  \begin{equation}
    \Psi_n^*(x,t)\Psi_n(x,t) = |\Psi_n(x,t)|^2 = |\Phi_n(x,t)|^2 = |\Psi_{n,\text{ung}}(x-\zeta(t),t)|^2 = |\Psi_{n,\text{ung}}(y,t)|^2 \; ,
    \label{betrag_einzelner}
  \end{equation}
  wobei $\Psi_{n,\text{ung}}(y,t)$ die Zustände des ungetriebenen Oszillators bezeichnen.

  Bei dem Integral von $-\infty$ bis $\infty$ ist eine konstante Verschiebung der Variable irrelevant.
  Außerdem sind die $\Psi_{n,\text{ung}}(y,t)$ auf dem Raum normiert, daher sind auch die $\Psi_n(x,t)$ auf $x \in (-\infty,\infty)$ normiert, wie bereits am Ende von Kapitel \ref{lsg_einzelner} erwähnt.

  Eine Visualisierung von Erwartungswerten folgt in Kapitel (\ref{erwartungswerte_gekoppelt}).
  Dort werden die Erwartungswerte der gekoppelten Oszillatoren dargestellt, bei denen es sich genau um die Summe aus den Erwartungswerten von zwei gesonderten Oszillatoren handelt, wie sich dort zeigen wird.
  % weil es sich bei den Erwartungswerten der zwei gekoppelten Oszillatoren genau um die Summe aus den Erwartungswerten von zwei gesonderten Oszillatoren handelt, wie sich dort zeigen wird.

  \section{Zeitabhängige Erwartungswerte des Ortes}
  %INTERPRETATION ; SCHWINGT KLASSICH USW
    Zuerst betrachten wir den Erwartungswert $\braket{x}_n$ des Ortsoperators $x$ für den $n$-ten Zustand $\Psi_n(x,t)$ des getriebenen Oszillators, welcher per Definition gegeben ist durch
    \begin{equation}
      \braket{x}_n = \int_{-\infty}^{\infty} \Psi_n^*(x,t)x\Psi_n(x,t) \: \text dx
      = \int_{-\infty}^{\infty} |\Psi_n(x,t)|^2 x \: \text dx \; .
    \end{equation}
    Mit der Substitution $y=x-\zeta(t)$ und unter Verwendung von Gleichung (\ref{betrag_einzelner}) und der Normierung der ungetriebenen Oszillator-Funktionen gelangen wir zu
    \begin{align}
      \begin{split}
        \braket{x}_n &= \int_{-\infty}^{\infty} |\Psi_{n,\text{ung}}(x-\zeta(t),t)|^2 x \: \text dx
        = \int_{-\infty}^{\infty} |\Psi_{n,\text{ung}}(y,t)|^2 (y+\zeta(t)) \: \text dy \\
        &= \braket{y}_{n,\text{ung}} + \zeta(t) = \zeta(t) \; .
        \label{erwartungswert_x_einzelner}
      \end{split}
    \end{align}
    Da die Erwartungswerte $\braket{y}_{n,\text{ung}}$ für den Ort des ungetriebenen Oszillators verschwinden, ist der Erwartungswert jedes Zuständes des getriebenen Oszillators genau $\zeta(t)$.

    Das bedeutet, der Erwartungswert $\braket{x}_n$ für den Ort eines beliebigen Zustandes des quantenmechanischen getriebenen Oszillators entspricht genau der Lösung $\zeta(t)$ der Bewegungsgleichung des klassischen getriebenen Oszillators.
    Der Erwartungswert hat deshalb sehr große Amplituden, wenn die Treibfrequenz $\omega$ von $S(t)$ in der Nähe der Eigenfrequenz des Systems $\omega_0$ liegt (\ref{zeta_bel_kraft}), was der klassischen Anschauung entspricht, da wir ein ungedämpftes System betrachten.
    Der Erwartungswert des ungetriebenen Oszillators, welcher im Ursprung liegt, ist somit zeitabhängig periodisch um $\zeta(t)=\zeta(t+T)$ verschoben, mit $T=2\pi/\omega_0$.
    Dies entspricht dem Ergebnis, wie es auch bei einer konstanten Verschiebung der Ortsvariable/-operator $x$ erhalten wird, z.\,B. beim Oszillator im zusätzlichen konstanten elektrischen Feld, nur das die Verschiebung hier die zeitabhängige klassische Lösung ist.

    Da $\braket{x}_n=\zeta(t)$ die klassische Bewegungsgleichung
    \begin{equation}
      m\ddot{\braket{x}}_n + m\omega_0^2\braket{x}_n - S(t) = 0
    \end{equation}
    erfüllt, ist zudem das Ehrenfest-Theorem erfüllt.

    Für den Erwartungswert von $x^2$ ergibt sich komplett analog
    \begin{align}
      \braket{x^2}_n &= \int_{-\infty}^{\infty} |\Psi_n(x,t)|^2 x^2 \: \text dx
      = \int_{-\infty}^{\infty} |\Psi_{n,\text{ung}}(y,t)|^2 (y+\zeta(t))^2 \: \text dy \notag \\
      &= \braket{y^2}_{n,\text{ung}} + 2\zeta(t)\braket{y}_{n,\text{ung}} + \zeta^2(t)
      = s^2(2n+1) + \zeta^2(t) \; ,
      \label{erwartungswert_x^2_einzelner}
    \end{align}
    zusammen mit der charakteristischen Länge des ungetriebenen Oszillators
    \begin{equation}
      s = \sqrt{\frac{\hbar}{2m\omega_0}} \; .
      \label{charak_laenge}
    \end{equation}
    Der Erwartungswert ist um $\zeta^2(t)$ verschoben.

    Insgesamt können alle Erwartungswerte $\braket{x^m}_n$ auf diese Art mit den bekannten ungetriebenen Erwartungswerten berechnet werden.
    Ab $\braket{x^3}_n$ kommt es wegen $(y+\zeta(t))^3$ aber im Allgemeinen nicht nur zu Verschiebungen bzw. zusätzlichen $\zeta(t)$-Termen, mit demselben Exponenten wie die ungetriebenen Erwartungswerte, weshalb wir den Erwartungswert $\braket{x^m}_n$ nicht einfacher schreiben können als den linearen Erwartungswert des Klammerausdruckes mit dem binomischen Lehrsatz zu vereinfachen:
    \begin{equation}
      \braket{x^m}_n = \braket{(y+\zeta(t))^m}_{n,\text{ung}} = \sum_{j=0}^m \begin{pmatrix} m \\ j \\ \end{pmatrix} \braket{y^{m-j}}_{n,\text{ung}}\zeta^j(t) \; .
      \label{erwartungswert_x^m_einzelner}
    \end{equation}
    In obiger Formel ist
    \begin{equation}
      \begin{pmatrix} m \\ j \\ \end{pmatrix} = \frac{m!}{j!(m-j)!}
    \end{equation}
    der Binomial-Koeffizient.

    Hiernach setzt sich der Erwartungswert $\braket{x^m}_n$ aus allen $\zeta(t)...\zeta^m(t)$ zusammen, die mit dem Binomial-Koeffizienten und der jeweiligen Potenz von $\braket{y}_{n,\text{ung}}$ gewichtet werden.


    %\newpage





  \section{Zeitabhängige Erwartungswerte des Impulses}
    Um den Erwartungswert $\braket{p}_n$ des Impulsoperators für den Zustand $\Psi_n(x,t)$ des getriebenen Oszillators zu ermitteln, nutzen wir gleichermaßen die Beziehung (\ref{betrag_einzelner}).
    Durch Anwenden der Produktregel beschränken wir das Problem erneut auf den ungetriebenen Oszillator.
    %und unterdessen ein geschicktes Anwenden der Produktregel, um das Problem erneut auf den ungetriebenen Oszillator zu beschränken.

    Wieder wählen wir den Variablenwechsel zur neuen Variable $y=x-\zeta(t)$, weil der Impulsoperator $p=p_x$ in der neuen Koordinate identisch ist, wie schon am Anfang von Kapitel \ref{lsg_einzelner} angedeutet, was sich mit der Kettenregel zeigen lässt:
    \begin{equation}
      -\text i\hbar \frac{\partial}{\partial x} = -\text i\hbar \frac{\partial}{\partial y}\frac{\partial y}{\partial x} = -\text i\hbar \frac{\partial}{\partial y} \cdot 1 \; , \; \text{d.\,h.} \; p_x = p_y \; .
    \end{equation}
    Weiterhin können wir den $y$-unabhängigen Teil der komplexen Phasen in den Wellenfunktionen (\ref{gesamtlsg_einzelner}) sofort am Operator vorbeiziehen und zu 1 zusmmenfassen, daher schreiben wir $\braket{p}_n$ als
    \begin{align}
      &\braket{p}_n = \int_{-\infty}^{\infty} \Psi^*_n(x,t) p \Psi_n(x,t) \; \text dx
      = \int_{-\infty}^{\infty} \Psi^*_n(y+\zeta(t),t) p_y \Psi_n(y+\zeta(t),t) \: \text dy= \notag\\
      &\int_{-\infty}^{\infty} \!\! N_nH_n(ay)\exp\left(\frac{-m\omega_0}{2\hbar}y^2-\frac{\text i}{\hbar}m\dot{\zeta}(t)y\right) p_y N_nH_n(ay)\exp\left(\frac{-m\omega_0}{2\hbar}y^2+\frac{\text i}{\hbar}m\dot{\zeta}(t)y\right) \: \text dy \; ,
      \label{spaste}
    \end{align}
    wobei der Faktor im Argument der Hermit-Polynome (\ref{gesamtlsg_einzelner}) als $a$ abgekürtzt wird.
    Mit der Produktregel spalten wir die Anwendung des Impulsoperators auf in die Anwendung auf die ungetriebene Oszillatorfunktion (in der neuen Koordinate $y$) und die Anwendung auf die verbleibende komplexe Phase:
    \begin{align}
      &p_y N_nH_n(ay)\exp\left(\frac{-m\omega_0}{2\hbar}y^2+\frac{+\text i}{\hbar}m\dot{\zeta}(t)y\right)  \notag\\
      %&\left(p_y N_nH_n(y)\text e^{\frac{-m\omega_0}{2\hbar}y^2}\right) \text e^{\frac{+\text i}{\hbar}m\dot{\zeta}(t)y}
      %+ N_nH_n(y)\text e^{\frac{-m\omega_0}{2\hbar}y^2} \left(p_y \text e^{\frac{+\text i}{\hbar}m\dot{\zeta}(t)y}\right) \notag\\
      &= \exp\left(\frac{+\text i}{\hbar}m\dot{\zeta}(t)y\right) (p_y+m\dot \zeta(t))N_nH_n(ay)\exp\left(\frac{-m\omega_0}{2\hbar}y^2\right)
      %\left[p_yN_nH_n(y)\text e^{\frac{-m\omega_0}{2\hbar}y^2} + N_nH_n(y)\text e^{\frac{m\omega_0}{2\hbar}y^2}m\dot{\zeta}(t) \right] \; .
      \label{spaste2}
    \end{align}
    Hiernach haben wir effektiv den Impulsoperator an der $y$-abhängigen Phase vorbeigezogen.
    Damit ist es möglich diese verbliebenen Phasen in (\ref{spaste}) ebenso zu 1 zu vereinfachen.
    Es verbleiben noch die Anwendung des Impulsoperators auf die Wellenfunktionen des ungetriebenen Oszillators und die Multiplikation dieser mit dem ortsunabhängigen Term $m\dot \zeta(t)$.
    Zusammen mit dem übrigen linken Teil aus (\ref{spaste2}), welcher nach dem Wegfallen der komplexen Phase jetzt auch der ungetriebenen Wellenfunktion entspricht, ergeben sich dadurch die Erwartungswerte für $p_y$ und den Term $m\dot{\zeta}(t)$.
    %welcher zeitabhängig ist.
    %Zusammen mit dem, nun identischen, übrigen komplex konjugiertem Teil aus (\ref{spaste2}) verbleiben nur noch mit im ersten Teil der der eckigen Klammer die Anwendung des Impulsoperators auf die Wellenfunktion des ungetriebenen Oszillators bzw

    Schließlich können wir den Erwartungswert für den Impuls in Abhängigkeit der bekannten Erwartungswerte des Standard-Oszillators schreiben:
    \begin{align}
      \braket{p}_n = \braket{p_y}_{n,\text{ung}} + m\dot{\zeta}(t) = m\dot{\zeta}(t) \; .
      \label{erwartungswert_p_einzelner}
    \end{align}

    Wieder erfüllt der Erwartungswert die Bewegungsgleichung des klassischen Oszillators:
    \begin{equation}
      \ddot{\braket{p}}_n + \omega_0^2\braket{p}_n - \dot{S}(t) = 0 \; .
    \end{equation}
\iffalse
    Wie an (\ref{spaste2}) zu sehen, ist es darüber hinaus einfach den Erwartungswert $\braket{p^m}_n$ für einen beliebigen Exponenten $m \in \mathbb{N}$ zu evaluieren.
    Wir können auf dem gleichen Weg die Produktregel anwenden, und erhalten wieder den ungetrieben Erwartungswert $\braket{p^m_y}_n$ aber diesmal zusätzlich die $m$-fache Anwendung des Operators auf die komplexe Phase, was wir sofort verallgemeinert aufschreiben können, da sich die komplexe Phase beim Ableiten nur reproduziert, weil immer der ortsunabhängige Term $m\dot{\zeta(t)}$ übrig bleibt.
    Es folgt:
    \begin{equation}
      \braket{p^m}_n = \braket{p^m_y}_{n,\text{ung}} + (m\dot{\zeta}(t))^m \; .
    \end{equation}
    Speziell für $m=2$ gilt:
    \begin{equation}
      \braket{p^2}_n = \braket{p^2_y}_{n,\text{ung}} + (m\dot{\zeta}(t))^2 = m^2\omega^2_0 s^2(2n+1) + m^2\dot{\zeta}^2(t) \; .
      \label{erwartungswert_p^2_einzelner}
    \end{equation}
\fi
    Um darüber hinaus den Erwartungswert $\braket{p^m}_n$ für einen beliebigen Exponenten $m \in \mathbb{N}$ zu evaluieren, muss in Formel (\ref{spaste2})) $m$-mal die Produktregel angewandt werden.
    Da sich die komplexe Phase beim Ableiten reproduziert und jedesmal ein ortsunabhängiger Term übrig bleibt, gilt für diesen Erwartungswert analog zu dem Erwartungswert $\braket{x^m}_n$:
    \begin{equation}
        \braket{p^m}_n = \braket{(p_y+m\dot\zeta(t))^m}_{n,\text{ung}} = \sum_{j=0}^m\begin{pmatrix} m \\ j \\ \end{pmatrix} \braket{p_y^{m-j}}_{n,\text{ung}}(m\dot\zeta(t))^j \; .
        \label{erwartungswert_p^m_einzelner}
    \end{equation}

    Speziell für $\braket{p^2}_n$ gilt:
    \begin{equation}
      \braket{p^2}_n = \braket{p^2_y}_{n,\text{ung}} +2m\dot\zeta(t)\braket{p_y}_{n,\text{ung}}+(m\dot{\zeta}(t))^2 = m^2\omega^2_0 s^2(2n+1) + m^2\dot{\zeta}^2(t) \; .
      \label{erwartungswert_p^2_einzelner}
    \end{equation}



  \section{Heisenberg-Unschärferelation}
    Mit den zuvor berechneten Erwartungswerten $\braket{x}_n,\braket{x^2}_n,\braket{p}_n,\braket{p^2}_n$ wird hier die Unschärfe des periodisch getriebenen Oszillators aufgestellt.
    Diese ist
    \begin{align}
      \Delta x_n\Delta p_n &= \sqrt{\braket{x^2}_n-\braket{x}^2_n}\sqrt{\braket{p^2}_n-\braket{p}^2_n} \notag\\
      &= \sqrt{s^2(2n+1)+\zeta^2(t)-\zeta^2(t)}\sqrt{m^2\omega^2_0s^2(2n+1)+m^2\zeta^2(t)-m^2\zeta^2(t)} \notag\\
      &= \frac{\hbar}{2}(2n+1) \; .
    \end{align}
    Die Unschärfe ist also gleich der des ungetriebenen Oszillators, welche zeitunabhängig ist und für den Grundzustand $n=0$ minimal ist.

    Dieses Ergebnis ist sinvoll in Anbetracht der Tatsache, dass die (zeitabhängige) Verschiebung der Wellenfunktionen im Raum $\zeta(t)$ (\ref{gesamtlsg_einzelner}) die Unschärfe nicht ändern sollte.
    Die Wellenfunktionen haben allerdings noch den weiteren Unterschied der orts- und zeitabhängigen komplexen Phase, verglichen mit den Wellenfunktionen des Standard-Oszillators.
    Diese spielt zwar keine  Rolle bei den Erwartungswerten für den Ort, bei den Impulserwartungswerten aber schon.
    Dennoch war zu erwarten, dass sich auch beim Impuls eine Standardabweichung $\Delta p_n$ wie beim ungetriebenen Oszillator ergibt, so wie bei $\Delta x_n$, da $\braket{p}_n$ im Wesentlichen die zeitliche Ableitung von $\braket{x}_n$ ist.
    Demnach folgte die gleiche Unschärfe.

  %  MEHR SCHREIEN ????????

\iffalse
  \subsection{Erwartungswerte der Energie}
    Hier werden wir den zeitabhängigen Erwartungswert der Energie $\braket{H(t)}_n$ aufstellen.
    Weiterhin wird der zeitlich, über eine Periode $T$, gemittelten Erwartungswert $\overline{H}_n$ mittels Formel (\ref{mittleres_H}) für eine beliebige Treibkraft bestimmt, indem wir die Quasienergien $\epsilon_n$ benutzen, welche wir in Kapitel \ref{espsilon_bel_kraft} für eine allgemeine periodische Treibkraft $S(t)$, in Abhängigkeit deren  Fourier-Koeffizienten, bestimmt haben.
\fi

  \section{Zeitabhängiger Erwartungswert der Energie}
    Mit denselben Erwartungswerten für den Ort und den Impuls, wie sie bei der Heisenber-Unschärferelation von Nöten sind, wird im Folgenden der normale Erwartungswert der Energie bzw. des Hamilton-Operators $\braket{H(t)}_n$ aufgestellt, es muss nur die Linearität des Erwartungswertes verwendet werden:
    \begin{align}
      \braket{H(t)}_n &= \braket{\frac{p^2}{2m} + \frac{1}{2}m\omega_0^2x^2 - S(t)x}_n
      =\frac{1}{2m}\braket{p^2}_n + \frac{1}{2}m\omega_0^2\braket{x^2}_n - S(t)\braket{x}_n \notag\\
      &= \frac{1}{2m}\braket{p^2_y}_{n,\text{ung}} + \frac{1}{2}m\omega_0^2\braket{y}_{n,\text{ung}} + \frac{1}{2}m\dot{\zeta}^2(t) + \frac{1}{2}m\omega_0^2\zeta^2(t) - S(t)\zeta(t) \notag\\
      &= \hbar\omega_0\left(n+\frac{1}{2}\right) - L(\dot{\zeta},\zeta,t) + m\dot{\zeta}^2(t) = E_n - L(\dot{\zeta},\zeta,t) + m\dot{\zeta}^2(t) \; .
      %hier en in extra zeile mit satz definieren evt
    \end{align}
    Wir finden wieder den Erwartungswert des ungetriebenen Oszillators $E_n$, mit einer zeitabhängigen Verschiebung, in welcher die Lagrange-Funktion des klassischen Problems (\ref{lagrange_zeta}) mit dem Restterm identifiziert werden kann.
    %Die Amplitude des Erwartungswertes kann wie erwartet sher groß werden, wenn die Treibfrequenz $\omega$ von $S(t)$ nah an der Eigenfrequenz des Systems $\omega_0$ liegt,
  %  WELCHE PERIODE, SCHWANKT

  \section{Durchschnittlicher Erwartungswert der Energie}
    Im Nachfolgenden wird der zeitlich, über eine Periode $T$, gemittelte Erwartungswert $\overline{H}_n$ mittels Formel (\ref{mittleres_H}) für eine beliebige Treibkraft bestimmt, indem wir die Quasienergien $\epsilon_n$ benutzen, welche wir in Kapitel \ref{epsilon_bel_kraft} für eine allgemeine periodische Treibkraft $S(t)$, in Abhängigkeit deren Fourier-Koeffizienten, bestimmt haben.
    %Um den zeitlich gemittelten Erwartungswert der Energie $\overline{H}_n$ für eine beliebige periodisch Treibende Kraft $S(t)=S(T)$ zu berechnen,

    Unter der Annahme, dass die Summe in $\epsilon_n$ (\ref{epsilon_bel_kraft}) konvergiert, weil es sich im Wesentlichen um die Summe über $1/j^2$ handelt, leiten wir nach $\omega$ ab, indem wir die Ableitung und Summe vertauschen, dann haben wir:
    \begin{align}
      \overline{H}_n = \epsilon_n - \omega\frac{\partial}{\partial \omega}\epsilon_n
      = E_n - \sum_{j=-\infty}^{\infty} \left[ \frac{c_jc_{-j}}{2m(\omega_0^2-j^2\omega^2)}\left( 1-\frac{2j^2\omega^2}{(\omega_0^2-j^2\omega^2)}\right) \right] \; .
    \end{align}

    Für den expliziten Fall der Sinus-Treibkraft $S(t)=A\sin(\omega t)$ folgt:
    \begin{equation}
      \overline{H}_n = E_n - \frac{A}{4m(\omega_0^2-\omega^2)}\left(1-\frac{2\omega^2}{(\omega_0^2-\omega^2)}\right) \; .
      \label{mittleres_H}
    \end{equation}
    Für $\omega_0^2>\omega^2$ erhalten wir in diesem Beispiel demnach einen Anstieg des Energiemittelwertes gegenüber den Quasienergien, andernfalls eine Absenkung.



















%\section{\texorpdfstring{Zeitlich gemittelter Erwartungswert der Energie $\overline{H}_n$}{Zeitlich gemittelter Erwartungswert der Energie bar{H}_n}}
