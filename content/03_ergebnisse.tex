\chapter{Ergebnisse}


\section{Quasienergien des einzelnen getriebenen Oszillators für eine beliebige periodische Treibkraft $S(t)$}
  Um die Quasienergien bei einer beliebigen periodisch treibenden Kraft auf elegante Weise zu bestimmen, und ohne ein Wirkungsintegral berechnen zu müssen, setzen wir eine komplexe Fourier-Reihe an:
  \begin{equation}
    S(t) = \sum_{j=-\infty}^\infty c_j \text e^{\text ij\omega t}, \; c_j = \frac{1}{T} \int_0^T S(t) \text e^{-ij\omega t} \: \text d t \; .
  \end{equation}
  Die $c_j$ haben die Einheit einer Kraft.

  Für $\zeta(t)$ wählen wir jetzt ebenfalls einen Reihenansatz:
  \begin{equation}
    \zeta(t) = \sum_{j=-\infty}^\infty d_j \text e^{\text ij\omega t} \; .
  \end{equation}
  Wir betrachten wieder nur die inhomogene klassische Bewegungsgleichung und bestimmen die $d_j$, indem wir einsetzen und einen Koeffizientenvergleich der $d_j$ und $c_j$ machen.
  Es zeigt sich, dass
  \begin{equation}
    \zeta(t) = \sum_{j=-\infty}^\infty \frac{c_j}{m(-j^2\omega^2+\omega_0^2)} \text e^{ij\omega t}
  \end{equation}
  gilt, womit sich die Lagrange-Funktion (\ref{lagrange_zeta}) ergibt zu:
  \begin{align}
    \begin{split}
      L &= \frac{1}{2}m\dot \zeta^2 - \frac{1}{2}m\omega_0^2\zeta^2 + S(t)\zeta \\
       &=\sum_j \sum_l \left[ \frac{-\omega^2}{2m} \frac{jc_j}{-j^2\omega^2+\omega_0^2} \frac{lc_l}{-l^2\omega^2+\omega_0^2}
       -\frac{\omega_0^2}{2m} \frac{c_j}{-j^2\omega^2+\omega_0^2} \frac{c_l}{-l^2\omega^2+\omega_0^2} \right. \\
        &\left. \quad + \frac{1}{m} \frac{c_jc_l}{-j^2\omega^2+\omega_0^2} \right] \text e^{\text i(j+l)\omega t}  \; , \; j,l \in \mathbb{Z} \; .
     \end{split}
   \end{align}
  \iffalse
  \begin{align}
    \begin{split}
      L &= \frac{1}{2}m\dot \zeta^2 - \frac{1}{2}m\omega_0^2\zeta^2 + S(t)\zeta \\
       &=\frac{-\omega^2}{2m} \sum_j \sum_l \frac{jc_j}{-j^2\omega^2+\omega_0^2} \frac{lc_l}{-l^2\omega^2+\omega_0^2} \text e^{\text i(j+l)\omega t}\\
       &\quad-\frac{\omega_0^2}{2m} \sum_j \sum_l \frac{c_j}{-j^2\omega^2+\omega_0^2} \frac{c_l}{-l^2\omega^2+\omega_0^2} \text e^{\text i(j+l)\omega t}\\
       &\quad + \frac{1}{m} \sum_j \sum_l \frac{c_jc_l}{-j^2\omega^2+\omega_0^2} \text e^{\text i(j+l)\omega t}\; , \; j,l \in \mathbb{Z} \; .
     \end{split}
   \end{align}
   \fi

   Nun identifizieren wir alle Terme der Lagrange-Funktion, welche nach der Ausführung des Wirkungsintegrals linear in der Zeit $t$ sind, ohne dieses explizit zu berechnen.
   Denn wir wissen, dass die Exponentialterme, welche sich beim Integrieren nach der Zeit reproduzieren, periodisch in der Zeit sind.
   Daher werden nur die konstanten Terme, welche entstehen wenn die Exponentialterme wegfallen, eine lineare Abhängigkeit aufweisen.
   Die Exponentialterme werden 1 wenn $j=-l$ ist.
   Infolgedessen wird die Doppelsumme, die bei der Quadrierung entstanden ist, wieder zur Einzelsumme.
   Wir fassen den linearen Teil des Wirkungsintegrals zusammen:
   \begin{align}
     \begin{split}
       &\frac{1}{T} \int_0^T L \: \text d t \\
       &= \sum_j \left[\frac{\omega^2}{2m} \frac{j^2c_jc_{-j}}{(-j^2\omega^2+\omega_0^2)^2}
       - \frac{\omega_o^2}{2m} \frac{c_jc_{-j}}{(-j^2\omega^2+\omega_0^2)^2} \right.
       \left. +\frac{1}{m} \frac{c_jc_{-j}}{-j^2\omega^2+\omega_0^2}
       \right]\\
       &= \sum_j \left[\frac{1}{2m} \frac{c_jc_{-j}(j^2\omega^2-\omega_0^2)}{(-j^2\omega^2+\omega_0^2)^2} + \frac{1}{m} \frac{c_jc_{-j}}{-j^2\omega^2+\omega_0^2} \right] \\
       &= \sum_j \frac{c_jc_{-j}}{2m(-j^2\omega^2+\omega_0^2)} = \frac{c_0^2}{2m\omega_0^2} + \sum_{j=1}^\infty \frac{c_j^2}{m(-j^2\omega^2+\omega_0^2)}
     \end{split}
   \end{align}

   Die Quasienergien $\epsilon_n$ sind hiernach:
   \begin{align}
     \begin{split}
       \epsilon_n &= \hbar \omega_0\left(n+\frac{1}{2}\right) - \sum_j \frac{c_jc_{-j}}{2m(-j^2\omega^2+\omega_0^2)} \; .
     \end{split}
   \end{align}
   Interessanterweise kommt es in obiger Formel nicht nur bei $\omega = \omega_0$ zu einer Singularität, wie bei der Beispielkraft $S(t) = A\sin(\omega t)$, sondern bei allen $\omega = \omega_0 / j$.



   \subsection{Beispiele für S(t) dreieck rechteck delta}





\section{Zwei getriebene gekoppelte Oszillatoren}
  In diesem Teil der Arbeit wird mit Hilfe der aus (\ref{lsg_einzelner}) bekannten Lösung des einzelnen getrieben Oszillators, die Wellenfunktionen für ein System hergeleitet, dass aus zwei gekoppelten Oszillatoren $x_1$ und $x_2$ der gleichen Masse $m$ besteht, von denen einer mit der periodischen Kraft $S(t) = S(t+T)$ angetrieben wird.
  Die Potentialkonstanten $k$ der beiden Oszillatoren sind ebenfalls identisch, die Kopplungskonstante $\kappa$ zwischen den Oszillatoren ist allerdings anders.
  Der Hamilton-Operator dieses Systems kann direkt aus der klassichen Mechanik übernommen werden:
  \begin{equation}
    H(t) = H(t+T) = \frac{p_1^2}{2m} + \frac{p_2^2}{2m} + \frac 1 2 kx_1^2 + \frac 1 2 kx_2^2 + \frac 1 2 \kappa(x_2-x_1)^2 - S(t)x_1 \; .
  \end{equation}
  Es werden auch die Erwartungswerte für  den Ort $\braket{x_{1,2}}_{n,l}$ und den Impuls $\braket{p_{1,2}}_{n,l}$, genauso wie der Erwartungswert der Energie $\braket{H}_{n,l}$ und dessen zeitliches Mittel $\bar{H}_{n,l}$ berechnet und visualisiert.

  Zur Lösung des Systems wird eine unitäre Koordinatentransformation eingeführt, welche den Hamilton-Operator $H(x_1,x_2,p_1,p_2,t)$ zu zwei in den neuen Koordinaten unabhängigen Hamilton-Operatoren $H_+(x_+,p_+,t)$ und $H_-(x_-,p_-,t)$ mit effektiven Potentialkonstanten $k_+,k_-$ entkoppelt.
  Dann ergeben sich die Wellenfunktionen leicht aus denen des einzelnen getriebenen Oszillators.



  \subsection{Die Schrödinger-Gleichung mit unabhängigen Hamilton-Operatoren}
    Liegt ein Hamilton-Operator der Form
    \begin{equation}
      H = \sum_i H_i(x_i,p_i,t) \;,\; H_i : \cal H_i \rightarrow H_i
    \end{equation}
    vor, führt der Ansatz
    \begin{equation}
      \Psi = \prod_i \Psi(x_i,p_i,t) \; , \; \Psi_i \in \cal H_i
    \end{equation}
    auf unabhängige Schrödinger-Gleichungen für die einzelnen Wellenfunktionen \\ $\Psi_i(x_i,p_i,t)$, sodass \cite{online quelle}
    \begin{equation}
      \delta_{i,j} \text i \hbar \frac{\partial}{\partial t}\Psi_i = H_j \Psi_i
    \end{equation}
    erfüllt ist.
    Um dies schnell zu zeigen, schauen wir uns den Fall von zwei unabhängigen Operatoren an:
    \begin{align}
      \begin{split}
        \text i \hbar \frac{\partial}{\partial t} \Psi = H\Psi \iff \Psi_2 H_1 \Psi_1 + \Psi_1 H_2 \Psi_2 = \text i \hbar \left(\frac{\partial}{\partial t} \Psi_1\Psi_2 + \Psi_1\frac{\partial}{\partial t} \Psi_2 \right) \; .
      \end{split}
    \end{align}
    Da die Operatoren nur auf Funktionen wirken, die auf dem selben Raum definiert sind, kann man sie an der jeweils anderen Funktion vorbei ziehen.
    Wenn wir weiterhin durch den Ansatz $\Psi_1\Psi_2$ teilen, wird die Gleichung zu:
    \begin{equation}
      \frac{1}{\Psi_1}H_1\Psi_1 + \frac{1}{\Psi_2}H_2\Psi_2 = \text i \hbar \left(\frac{\frac{\partial}{\partial t} \Psi_1}{\Psi_1} + \frac{\frac{\partial}{\partial t} \Psi_2}{\Psi_2} \right) \; .
    \end{equation}
    Weil die linke und rechte Seite f"ur alle unabhängigen $x_1,p_1,x_2,p_2$ gleich sein müssen, folgen die einzelnen Schrödinger-Gleichungen für die Wellenfunktionen $\Psi_1$ und $\Psi_2$:
    \begin{equation}
      \frac{1}{\Psi_1}H_1\Psi_1 = \frac{\frac{\partial}{\partial t} \Psi_1}{\Psi_1} \; , \; \frac{1}{\Psi_2}H_2\Psi_2 = \frac{\frac{\partial}{\partial t} \Psi_2}{\Psi_2} \; .
    \end{equation}
































   b
