\chapter{Ergebnisse}


\section{Quasienergien des einzelnen getriebenen Oszillators f"ur eine beliebige periodische Treibkraft $S(t)$}
  Um die Quasienergien einer beliebigen periodisch treibenden Kraft auf elegante Weise zu bestimmen, ohne ein Wirkungsintegral berechnen zu m"ussen, setzen wir eine komplexe Furier-Reihe an:
  \begin{equation}
    S(t) = \sum_{j=-\infty}^\infty c_j \text e^{\text ij\omega t}, \; c_j = \frac{1}{T} \int_0^T S(t) \text e^{-ij\omega t} \: \text d t \; .
  \end{equation}
  Die $c_j$ haben die Einheit einer Kraft.

  F"ur $\zeta(t)$ w"ahlen wir jetzt ebenfalls einen Reihenansatz:
  \begin{equation}
    \zeta(t) = \sum_{j=-\infty}^\infty d_j \text e^{\text ij\omega t} \; .
  \end{equation}
  Wir betrachten wieder nur die inhomogene klassische Bewegungsgleichung und bestimmen die $d_j$, indem wir einsetzen und einen Koeffizientenvergleich der $d_j$ und $c_j$ machen.
  Es zeigt sich, dass
  \begin{equation}
    \zeta(t) = \sum_{j=-\infty}^\infty \frac{c_j}{m(-j^2\omega^2+\omega_0^2)} \text e^{ij\omega t}
  \end{equation}
  gilt, womit sich die Lagrange-Funktion (\ref{lagrange_zeta}) ergibt zu:
  \begin{align}
    \begin{split}
      L &= \frac{1}{2}m\dot \zeta^2 - \frac{1}{2}m\omega_0^2\zeta^2 + S(t)\zeta \\
       &=\sum_j \sum_l \left[ \frac{-\omega^2}{2m} \frac{jc_j}{-j^2\omega^2+\omega_0^2} \frac{lc_l}{-l^2\omega^2+\omega_0^2}
       -\frac{\omega_0^2}{2m} \frac{c_j}{-j^2\omega^2+\omega_0^2} \frac{c_l}{-l^2\omega^2+\omega_0^2} \right. \\
        &\left. \quad + \frac{1}{m} \frac{c_jc_l}{-j^2\omega^2+\omega_0^2} \text e^{\text i(j+l)\omega t} \right] \; , \; j,l \in \mathbb{Z} \; .
     \end{split}
   \end{align}
  \iffalse
  \begin{align}
    \begin{split}
      L &= \frac{1}{2}m\dot \zeta^2 - \frac{1}{2}m\omega_0^2\zeta^2 + S(t)\zeta \\
       &=\frac{-\omega^2}{2m} \sum_j \sum_l \frac{jc_j}{-j^2\omega^2+\omega_0^2} \frac{lc_l}{-l^2\omega^2+\omega_0^2} \text e^{\text i(j+l)\omega t}\\
       &\quad-\frac{\omega_0^2}{2m} \sum_j \sum_l \frac{c_j}{-j^2\omega^2+\omega_0^2} \frac{c_l}{-l^2\omega^2+\omega_0^2} \text e^{\text i(j+l)\omega t}\\
       &\quad + \frac{1}{m} \sum_j \sum_l \frac{c_jc_l}{-j^2\omega^2+\omega_0^2} \text e^{\text i(j+l)\omega t}\; , \; j,l \in \mathbb{Z} \; .
     \end{split}
   \end{align}
   \fi

   Nun identifizieren wir alle Terme der Lagrange-Funktion, welche nach der Ausf"uhrung des Wirkungsintegrals linear in der Zeit $t$ sind, ohne dieses explizit zu berechnen.
   Denn wir wissen, dass die Exponentialterme, welche sich beim Integrieren nach der Zeit reproduzieren, nicht linear in $t$ sein k"onnen.
   Daher werden nur die konstanten Terme, welche entstehen wenn die Exponentialterme wegfallen, eine lineare Abh"angigkeit aufweisen.
   Die Exponentialterme werden 1 wenn
   \begin{equation}
     j=-l
   \end{equation}
   ist.
   Infolgedessen werden die Doppelsummen, die bei der Quadrierung entstanden, wieder zur Einzelsumme.
   Wir fassen den linearen Teil des Wirkungsintegrals zusammen:
   \begin{align}
     \begin{split}
       &\frac{1}{T} \int_0^T L \: \text d t \\
       &= \sum_j \left[\frac{\omega^2}{2m} \frac{j^2c_jc_{-j}}{(-j^2\omega^2+\omega_0^2)^2}
       - \frac{\omega_o^2}{2m} \frac{c_jc_{-j}}{(-j^2\omega^2+\omega_0^2)^2} \right.
       \left. +\frac{1}{m} \frac{c_jc_{-j}}{-j^2\omega^2+\omega_0^2}
       \right]\\
       &= \sum_j \left[\frac{1}{2m} \frac{c_jc_{-j}(j^2\omega^2-\omega_0^2)}{(-j^2\omega^2+\omega_0^2)^2} + \frac{1}{m} \frac{c_jc_{-j}}{-j^2\omega^2+\omega_0^2} \right] \\
       &= \sum_j \frac{c_jc_{-j}}{2m(-j^2\omega^2+\omega_0^2)} = \frac{c_0^2}{2m\omega_0^2} + \sum_{j=1}^\infty \frac{c_j^2}{m(-j^2\omega^2+\omega_0^2)}
     \end{split}
   \end{align}

    und die Quasienergien $\epsilon_n$ sind hiernach:
   \begin{align}
     \begin{split}
       \epsilon_n &= \hbar \omega_0(n+\frac{1}{2}) - \sum_j \frac{c_jc_{-j}}{2m(-j^2\omega^2+\omega_0^2)}
     \end{split}
   \end{align}
   Interessanterweise kommt es in obiger Formel nicht nur bei $\omega = \omega_0$ zu einer Singularit"at, sondern bei allen $\omega = \omega_0 / j$.



   \subsection{Beispiele f"ur S(t) dreieck rechteck delta}





























   b
